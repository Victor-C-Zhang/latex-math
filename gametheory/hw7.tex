\documentclass{article}
\usepackage[utf8]{inputenc}

\title{482 - Homework 7}
\author{Victor Zhang}
\date{April 7, 2020}

\usepackage[utf8]{inputenc}
\usepackage{amsmath}
\usepackage{amsfonts}
\usepackage{natbib}
\usepackage{graphicx}
% \usepackage{changepage}
\usepackage{amssymb}
% \usepackage{bm}
% \usepackage{empheq}
\usepackage{tikz}
\usepackage{multirow}
\usepackage{subcaption}


\newcommand{\contra}{\raisebox{\depth}{\#}}

\newenvironment{myindentpar}[1]
  {\begin{list}{}
          {\setlength{\leftmargin}{#1}}
          \item[]
  }
  {\end{list}}

\pagestyle{empty}

\begin{document}

\maketitle
% \begin{center}
% {\huge Econ 482 \hspace{0.5cm} HW 3}\
% {\Large \textbf{Victor Zhang}}\
% {\Large February 18, 2020}
% \end{center}

\section{Tomatoes}
\subsection{}
Denote the action "harvest" by $H$, "don't harvest" by $D$. Since this is the subgame beginning at the fourth day, if the tomato is not picked, it disappears forever.
\begin{center}\begin{tabular}{ r c c c }
     & & \multicolumn{2}{c}{Player 2}\\
     & & H & D\\
    \cline{3-4}
    \multirow{2}{*}{Player 1} & H & \multicolumn{1}{|c}{$(5,5)$} & \multicolumn{1}{|c|}{$(10,0)$}\\
    \cline{3-4}
                              & D & \multicolumn{1}{|c}{$(0,10)$} & \multicolumn{1}{|c|}{$(0,0)$}\\
    \cline{3-4}
\end{tabular}\end{center}
Note $H$ strictly dominates $D$ for both players, so $(H,H)$ is the only pure-strategy Nash equilibrium. In particular, it is the only Nash equilibrium $\Box$

\subsection{}
We may put $(5,5)$ as the outcome for $(D,D)$ since it is the unique Nash equilibrium associated with the subgame starting at day 4. Then we get matrix
\begin{center}\begin{tabular}{ c c c }
     & H & D\\
    \cline{2-3}
    H & \multicolumn{1}{|c}{$(4,4)$} & \multicolumn{1}{|c|}{$(8,0)$}\\
    \cline{2-3}
    D & \multicolumn{1}{|c}{$(0,8)$} & \multicolumn{1}{|c|}{$(5,5)$}\\
    \cline{2-3}
\end{tabular}\end{center}
Again, $H$ strictly dominates $D$ for both players, so $(H,H)$ is again the only Nash equilibrium. The SPNE of this subgame is thus $(HH,HH)$ $\Box$

\subsection{}
Similarly, the subgame starting at day 2 has one Nash equilibrium $(H,H)$ yielding payoffs $(3,3)$ and the original game thus has one Nash equilibrium $(H,H)$ yielding payoffs $(2,2)$. The only SPNE is thus $(HHHH,HHHH)$ $\Box$

\section{Monopoly Money}
\subsection{}
Denote by 1 the incumbent firm and 2 the new firm. Note if the new firm does not enter, 1 will simply maximize its utility by choosing $q_1^* = 4$. The payoffs for the subgame in which 2 does not enter is thus $(16,0)$. If 2 enters the market, we may reduce the problem to a Stackelberg game with two identical firms, since the cost of entry does not depend on quantity supplied. The SPNE of this subgame is firm 1 picking $q_1 = 4$ and firm 2 always responding by picking $q_2 = \frac{8-q_1}{2}$. Analysis of the Stackelberg game gives payoffs of $(8,4)$ for this subgame. Thus the actual payoffs of this subgame are $(8,3)$ if we consider cost of entry.\\
For the whole game, firm 2 will obviously choose to enter the industry. The SPNE is given by: $(q_1 = 4, \text{ enter the market and respond with } q_2 = \frac{8-q_1}{2})$ $\Box$

\subsection{}
Let the quantity firm 1 chooses be $q_1$. In the subgame that firm 2 decides to enter the market, it will act as in part (a) and produce $q_2 = \frac{8 - q_1}{2}$, so the outcome is $(\frac{(8-q_1)q_1}{2}, \frac{\left(8-q_1\right)^2}{4} - 1)$. In the subgame that firm 2 decides not to enter, 1 will face monopoly demand, so the outcome is $((8-q_1)q_1,0)$. Thus, as long as $\frac{\left(8-q_1\right)^2}{4} > 1$, firm 2 will choose to enter the market. Solving yields $q_1 < 6$. Note 2 achieves payoff 0 if $q_1 = 6$ regardless of its choice to enter or not.\\
From the perspective of firm 1, it faces a piecewise utility which is discontinuous at $q_1 = 6$. If 2 chooses to enter when $q_1 = 6$, the function is right-discontinuous and thus does not attain a maximum. Consequently, 1 has no best response. If instead 2 chooses to not enter, the maximum is attained at $q_1 = 6$.\\
The problem is now solved. The SPNE consists of firm 1 choosing $q_1 = 6$, firm 2 entering and picking $q_2 = \frac{8-q_1}{2}$ when $q_1 < 6$ and not entering when $q_1 \geq 6$ $\Box$

\section{Rigging Elections}
\subsection{}
The tournament function $T: X^2 \rightarrow X$ is given by
\begin{equation*}
    \begin{cases}
    T(A,B) = B\\
    T(A,C) = C\\
    T(A,D) = D\\
    T(A,E) = E\\
    T(B,C) = C\\
    T(B,D) = D\\
    T(B,E) = B\\
    T(C,D) = C\\
    T(C,E) = E\\
    T(D,E) = D
    \end{cases}
\end{equation*}

\subsection{}
We may create a "dominating" sequence $(C,D,B,E,A)$ s.t. each alternative wins against the next in the sequence. Thus $C$ is in the top cycle. Since $T(C,E) = E$, $T(D,E) = D$, and $T(B,E) = B$ $\{B,C,D,E\}$ is the top cycle $\Box$

\subsection{}
Note for every $x$ in the top cycle, we can find a dominating sequence starting at $x$. Thus, the binary agenda consisting of single eliminations guarantees $x$ a win:\\

% Set the overall layout of the tree
\tikzstyle{level 1}=[level distance=0cm, sibling distance=4cm]
\tikzstyle{level 2}=[level distance=1cm, sibling distance=1.5cm]

% Define styles for bags and leafs
\tikzstyle{spacer} = [circle, draw = none]
\tikzstyle{internal} = [circle,inner sep = 0cm, fill=black, text width = 1.5mm]
\tikzstyle{leaf} = [text width=4em, text centered]
% \tikzstyle{leaf} = [circle, minimum width=3pt,fill, inner sep=0pt]


\hspace*{-2.9cm}\begin{tikzpicture}[grow=down]
\node[spacer]{}
    child {
    node[internal]{}
        child {
            node[leaf]{B}
        }
        child {
            node[internal]{}
                child {
                    node[leaf]{E}
                }
                child {
                    node[internal]{}
                        child {
                            node[leaf]{C}
                        }
                        child {
                            node[internal]{}
                                child {
                                    node[leaf]{D}
                                }
                                child {
                                    node[leaf]{A}
                                }
                        }
                }
        }
    edge from parent [draw = none]
    }
    child {
    node[internal]{}
        child {
            node[leaf]{C}
        }
        child {
            node[internal]{}
                child {
                    node[leaf]{D}
                }
                child {
                    node[internal]{}
                        child {
                            node[leaf]{B}
                        }
                        child {
                            node[internal]{}
                                child {
                                    node[leaf]{E}
                                }
                                child {
                                    node[leaf]{A}
                                }
                        }
                }
        }
    edge from parent [draw = none]
    }
    child {
    node[internal]{}
        child {
            node[leaf]{D}
        }
        child {
            node[internal]{}
                child {
                    node[leaf]{E}
                }
                child {
                    node[internal]{}
                        child {
                            node[leaf]{C}
                        }
                        child {
                            node[internal]{}
                                child {
                                    node[leaf]{B}
                                }
                                child {
                                    node[leaf]{A}
                                }
                        }
                }
        }
    edge from parent [draw = none]
    }
    child {
    node[internal]{}
        child {
            node[leaf]{E}
        }
        child {
            node[internal]{}
                child {
                    node[leaf]{C}
                }
                child {
                    node[internal]{}
                        child {
                            node[leaf]{B}
                        }
                        child {
                            node[internal]{}
                                child {
                                    node[leaf]{D}
                                }
                                child {
                                    node[leaf]{A}
                                }
                        }
                }
        }
    edge from parent [draw = none]
    };
\end{tikzpicture}

\end{document}