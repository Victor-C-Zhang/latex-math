\documentclass{article}
\usepackage[utf8]{inputenc}

\title{482 - Midterm 2}
\author{Victor Zhang}
\date{April 14, 2020}

\usepackage[utf8]{inputenc}
\usepackage{amsmath}
\usepackage{amsfonts}
\usepackage{natbib}
\usepackage{graphicx}
% \usepackage{changepage}
\usepackage{amssymb}
% \usepackage{bm}
% \usepackage{empheq}
\usepackage{tikz}
\usepackage{multirow}
\usepackage{subcaption}


\newcommand{\contra}{\raisebox{\depth}{\#}}

\newenvironment{myindentpar}[1]
  {\begin{list}{}
          {\setlength{\leftmargin}{#1}}
          \item[]
  }
  {\end{list}}

\pagestyle{empty}

\begin{document}

\maketitle
% \begin{center}
% {\huge Econ 482 \hspace{0.5cm} HW 3}\
% {\Large \textbf{Victor Zhang}}\
% {\Large February 18, 2020}
% \end{center}

\section{}
\subsection{}
If $q_1 \geq 16$, $q_2^{*} = 0$. Otherwise, we maximize $U_2 = (16-q_1-q_2-4)q_2$ by putting $q_2 = \frac{12-q_1}{2}$. Thus the Nash equilibrium when $q_1 < 16$ is $(q_1,\frac{12-q_1}{2})$.
\subsection{}
Firm 1's actions do not affect 2's payoff. So maximize $U_2 = (8-q_2-6)q_2$ by putting $q_2 = 1$. Then the Nash equilibrium is $(q_1,1)$.
\subsection{}
Firm 2 will choose to compete in country A if its payoff is greater, $\frac{12-q_1}{2} > 1$. In other words, when $q_1 < 10$. Similar reasoning shows firm 2 will choose to develop in country B if $q_1 > 10$ and is indifferent if $q_1=10$.
\subsection{}
In the subgame that firm 2 decides to compete, it will act as in part (a) and produce $q_2 = \frac{12 - q_1}{2}$, so the outcome is $(\frac{(12-q_1)q_1}{2}, \frac{\left(12-q_1\right)^2}{4})$. In the subgame that firm 2 decides not to compete, 1 will face monopoly demand, so the outcome is $((12-q_1)q_1,1)$.
From the perspective of firm 1, it faces a piecewise utility which is discontinuous at $q_1 = 10$. If 2 chooses to compete when $q_1 = 10$, the function is right-discontinuous and thus does not attain a maximum. Consequently, 1 has no best response. If instead 2 chooses to not compete, the maximum is attained at $q_1 = 10$.\\
The problem is now solved. The SPNE consists of firm 1 choosing $q_1 = 10$, firm 2 competing when $q_1 < 10$, developing in country B otherwise, and choosing $q_2$ based on the analysis in (a) and (b). In particular, this induces an outcome with payoffs $(20,1)$
\subsection{}
Note if firm 2 chooses to develop in country B, it will face payoff of 1. But if firm 2 chooses to compete in country A, this is the Stackelberg game and as we have derived in class, they will enjoy payoffs of $(18,9)$. Thus the SPNE outcome will include firm 2 choosing to compete and forcing firm 1 to share the market.

\section{}
\subsection{}
The tournament function $T: X^2 \rightarrow X$ is given by
\begin{equation*}
    \begin{cases}
    T(A,B) = B\\
    T(A,C) = A\\
    T(A,D) = A\\
    T(A,E) = E\\
    T(B,C) = C\\
    T(B,D) = B\\
    T(B,E) = E\\
    T(C,D) = C\\
    T(C,E) = C\\
    T(D,E) = E
    \end{cases}
\end{equation*}
We may create a "dominating" sequence $(B,A,C,E,D)$ s.t. each alternative wins against the next in the sequence. Thus $B$ is in the top cycle. Since $T(C,B) = C$, $T(A,C) = A$, and $T(A,E) = E$, $\{A,B,C,E\}$ is the top cycle $\Box$

\subsection{}
Note for every $x$ in the top cycle, we can find a dominating sequence starting at $x$. Thus, the binary agenda consisting of single eliminations guarantees $x$ a win:\\

% Set the overall layout of the tree
\tikzstyle{level 1}=[level distance=0cm, sibling distance=4cm]
\tikzstyle{level 2}=[level distance=1cm, sibling distance=1.5cm]

% Define styles for bags and leafs
\tikzstyle{spacer} = [circle, draw = none]
\tikzstyle{internal} = [circle,inner sep = 0cm, fill=black, text width = 1.5mm]
\tikzstyle{leaf} = [text width=4em, text centered]
% \tikzstyle{leaf} = [circle, minimum width=3pt,fill, inner sep=0pt]


\hspace*{-2.9cm}\begin{tikzpicture}[grow=down]
\node[spacer]{}
    child {
    node[internal]{}
        child {
            node[leaf]{A}
        }
        child {
            node[internal]{}
                child {
                    node[leaf]{C}
                }
                child {
                    node[internal]{}
                        child {
                            node[leaf]{E}
                        }
                        child {
                            node[internal]{}
                                child {
                                    node[leaf]{B}
                                }
                                child {
                                    node[leaf]{D}
                                }
                        }
                }
        }
    edge from parent [draw = none]
    }
    child {
    node[internal]{}
        child {
            node[leaf]{B}
        }
        child {
            node[internal]{}
                child {
                    node[leaf]{A}
                }
                child {
                    node[internal]{}
                        child {
                            node[leaf]{C}
                        }
                        child {
                            node[internal]{}
                                child {
                                    node[leaf]{E}
                                }
                                child {
                                    node[leaf]{D}
                                }
                        }
                }
        }
    edge from parent [draw = none]
    }
    child {
    node[internal]{}
        child {
            node[leaf]{C}
        }
        child {
            node[internal]{}
                child {
                    node[leaf]{E}
                }
                child {
                    node[internal]{}
                        child {
                            node[leaf]{B}
                        }
                        child {
                            node[internal]{}
                                child {
                                    node[leaf]{A}
                                }
                                child {
                                    node[leaf]{D}
                                }
                        }
                }
        }
    edge from parent [draw = none]
    }
    child {
    node[internal]{}
        child {
            node[leaf]{E}
        }
        child {
            node[internal]{}
                child {
                    node[leaf]{B}
                }
                child {
                    node[internal]{}
                        child {
                            node[leaf]{A}
                        }
                        child {
                            node[internal]{}
                                child {
                                    node[leaf]{C}
                                }
                                child {
                                    node[leaf]{D}
                                }
                        }
                }
        }
    edge from parent [draw = none]
    };
\end{tikzpicture}
These correspond to $A,B,C,E$ winning, respectively.

\section{}
\subsection{}
Note the worker will always reject the offer if the salary $x<30$, and always accept if $x>30$. In a similar vein to our analysis of the Ultimatum and Holdup games, the SPNE consists of the employer offering $x = 30$ and the worker accepting all offers where $x \geq 30$.
\subsection{}
Similarly, the worker will reject all $x < 35$, accept all $x>35$, and is indifferent to $x=35$. Thus, the SPNE consists of the employer offering $x = 35$ and the worker accepting all offers where $x \geq 35$.
\subsection{}
The subgame SPNEs induce outcomes with payoffs of $(70,5)$ and $(40,20)$ respectively, where the first number is the firm's payoff, and the second the worker's. Clearly, the worker will pursue training $G$ since $20>5$. The problem is now solved. The SPNE consists of the worker choosing training $G$, and the players playing the subgames in the manner described in (a) and (b), respectively.
\subsection{}
In the SPNE-induced outcome, the payoffs are $(40,20)$. But they can achieve payoffs of $(50,25)$ if they play the following strategy:\\
The worker chooses training $S$. No strategies are changed in the $G$-subgame. In the $S$-subgame, the employer offers $x=50$ and the worker accepts all offers.

\end{document}