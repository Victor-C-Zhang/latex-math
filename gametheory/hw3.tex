\documentclass{article}
\usepackage[utf8]{inputenc}

\title{482 - Homework 3}
\author{Victor Zhang }
\date{February 18, 2020}

\usepackage[utf8]{inputenc}
\usepackage{amsmath}
\usepackage{amsfonts}
\usepackage{natbib}
\usepackage{graphicx}
% \usepackage{changepage}
\usepackage{amssymb}
% \usepackage{bm}
% \usepackage{empheq}

\newcommand{\contra}{\raisebox{\depth}{\#}}

\newenvironment{myindentpar}[1]
  {\begin{list}{}
          {\setlength{\leftmargin}{#1}}
          \item[]
  }
  {\end{list}}

\pagestyle{empty}

\begin{document}

\maketitle
% \begin{center}
% {\huge Econ 482 \hspace{0.5cm} HW 3}\\
% {\Large \textbf{Victor Zhang}}\\
% {\Large February 18, 2020}
% \end{center}

\section{Lobbying}
\subsection{}
For $x_{-i} \leq 48$, the best response is not to lobby. Not lobbying gives a payoff of 0 and lobbying gives a payoff of -1.\\
For $x_{-i} = 49$, the best response is to lobby. Lobbying gives a payoff of 9 and not lobbying gives a payoff of 0.\\
For $x_{-i} \geq 50$, the best response is not to lobby. Not lobbying gives a payoff of 10 while lobbying gives a payoff of 9 $\Box$
\subsection{}
No. If more than 50 residents lobby, for each resident $i$, $x_{-i} \geq 50$. Then their best response is not to lobby. That means at least 50 people are not responding optimally $\contra$
\subsection{}
Yes. For each resident $i$ who lobbies, $x_{-i} = 49$, so the best response is to lobby. For each resident $j$ who does not lobby, $x_{-i} = 50$, so the best response is to not lobby. In either case, each resident is acting optimally $\Box$
\subsection{}
Clearly, the case where nobody lobbies is a Nash equilibrum. $x_{-i} = 0$ for all residents $i$, and the best response is to not lobby $\Box$
\subsection{}
Let the mixed strategy be $\sigma_i = (1-p,p)$, where $p$ is the probability of \textit{not} lobbying. Then the probability that at least one person lobbies is $1-p^{98}$. The payoff for $i$ is then
$$(1- p^{98}) \cdot (9(1-p) + 10p) + p^{98} \cdot (-(1-p))$$
Maximizing the payoff function gives appropriate $p$ $\Box$

\section{3-Party System}
\subsection{}
There are 4 outcomes: Adams wins, Buchanan wins, Cleveland wins, or there is a tie $\Box$
\subsection{}
If Adams wins, the only Nash equilibrium is the action profile $(A,A,A)$. No profile in which Adams gets two votes is in equilibrium:
\begin{myindentpar}{2em}
    If voter 1 doesn't vote for $A$, then she must be voting for another candidate. WLOG suppose she votes for $B$. Then voter 2 is not best responding. Otherwise if voter 1 votes for $A$, then either voter 2 or voter 3 does not vote for $A$. WLOG suppose voter 2 votes for $B$ or $C$. Then voter 3's best response is $C$, not $A$.
\end{myindentpar}
By symmetry, the only Nash equilibrium where Buchanan wins is $(B,B,B)$ and the only Nash equilibrium where Cleveland wins is $(C,C,C)$.\\
It is clear that the Nash equilibrium in case of a tie is $(A,B,C)$. If the votes are otherwise distributed, then somebody can vote for their preferred candidate and guarantee them a win $\Box$
\subsection{}
From our analysis in 2(b), we can see that voting for one's preferred candidate is a weakly dominant strategy. This is evident if we consider all of voter 3's best responses: If either voter 1 or voter 2 vote for $C$, guaranteeing $C$ a win is a best response. If voters 1 and 2 vote together, it doesn't matter what she votes. If voters 1 and 2 split their vote, voting for $C$ is the best response, since she prefers a tie more than a non-preferred candidate winning $\Box$
\subsection{}
Clearly the pure-win (e.g.: $(A,A,A)$) Nash equilibria remain equilibria.
There is a new 2-vote win Nash equilibrium $(A,B,A)$, since voter 3 prefers A winning over a tie.
There is now no Nash equilibrium in which the candidates tie, since the profile must be still $(A,B,C)$ for voters 1 and 2 to best respond, but now voter 3 is not best-responding $\Box$

\section{Two-Player Nash Equilibria}
\subsection{}
Note that if player 2 plays $L$, $C$, and $R$, the best responses are $D$, $M$, and $U$ respectively. Out of these combinations, $C$ is a best response to $M$. So the only pure strategy Nash equilibrium is $(M,C)$ $\Box$
\subsection{}
Let the mixed strategy for player 2 be $\sigma_2 = (x,y,1-x-y)$. We wish to find $x,y$ s.t. each action $U,M,D$ is a best response to $\sigma_2$.
\begin{equation*}
    \begin{cases}
        2x + y + 4(1-x-y) = x + 4y + 2(1-x-y)\\
        x + 4y + 2(1-x-y) = 5x + 2y\\
    \end{cases}
\end{equation*}
The system of equations has one solution, namely $x = y= \frac{1}{3}$, so $\sigma_2 = (\frac{1}{3},\frac{1}{3},\frac{1}{3})$.\\
Now let the mixed strategy for player 1 be $\sigma_1 = (x,y,1-x-y)$. Similarly, we wish to find $x,y$ s.t. each action $L,C,R$ is a best response to $\sigma_1$.
\begin{equation*}
    \begin{cases}
        7x + 2y + (1-x-y) = x + 6y + 3(1-x-y)\\
        x + 6y + 3(1-x-y) = 2x + 4(1-x-y)\\
    \end{cases}
\end{equation*}
This system has one solution as well, $x = \frac{2}{7}, y = \frac{1}{7}$, so $\sigma_1 = (\frac{2}{7},\frac{1}{7},\frac{4}{7})$. The full Nash equilibrium is then $(\sigma_1 = (\frac{2}{7},\frac{1}{7},\frac{4}{7}), \sigma_2 = (\frac{1}{3},\frac{1}{3},\frac{1}{3}))$ $\Box$

\end{document}