\documentclass{article}
\usepackage[utf8]{inputenc}
\usepackage[margin=1in]{geometry}

\title{485 - Homework 3}
\author{Victor Zhang}
\date{October 1, 2021}

\usepackage[utf8]{inputenc}
\usepackage{amsmath}
\usepackage{amsfonts}
\usepackage{natbib}
\usepackage{graphicx}
% \usepackage{changepage}
\usepackage{amssymb}
\usepackage{xfrac}
% \usepackage{bm}
% \usepackage{empheq}
\usepackage{listings}
\usepackage{xcolor}

\newcommand{\contra}{\raisebox{\depth}{\#}}

\newenvironment{myindentpar}[1]
  {\begin{list}{}
          {\setlength{\leftmargin}{#1}}
          \item[]
  }
  {\end{list}}

\pagestyle{empty}

\definecolor{codegreen}{rgb}{0,0.6,0}
\definecolor{codegray}{rgb}{0.5,0.5,0.5}
\definecolor{codepurple}{rgb}{0.58,0,0.82}
\definecolor{backcolour}{rgb}{0.95,0.95,0.92}

\lstdefinestyle{mystyle}{
    backgroundcolor=\color{backcolour},   
    commentstyle=\color{codegreen},
    keywordstyle=\color{magenta},
    numberstyle=\tiny\color{codegray},
    stringstyle=\color{codepurple},
    basicstyle=\ttfamily\footnotesize,
    breakatwhitespace=false,         
    breaklines=true,                 
    captionpos=b,                    
    keepspaces=true,                 
    numbers=left,                    
    numbersep=5pt,                  
    showspaces=false,                
    showstringspaces=false,
    showtabs=false,                  
    tabsize=2
}

\lstset{style=mystyle}

\begin{document}

\maketitle
% \begin{center}
% {\huge Econ 482 \hspace{0.5cm} HW 3}\
% {\Large \textbf{Victor Zhang}}\
% {\Large February 18, 2020}
% \end{center}

\section{}
\subsection{}
We use the backward induction process. At time step $n$ we may define probability measure $\widetilde{\mathbb{P}}_n$ as we do normally with $\widetilde{\mathbb{P}}$. We note $\frac{1}{1+r_n}\widetilde{\mathbb{E}}_n[S_{n+1}] = S_n$ and similarly that $\widetilde{S}_n = \frac{S_n}{\prod^{n-1}_{i=0} (1+r_i)}$ is a $\widetilde{P}$-martingale. Then by replication we may write
$$V_n = \frac{1}{1+r_n} \left[ \tilde{p}_n V_{n+1}(H) + \tilde{q}_n V_{n+1}(T) \right]$$
To get $V_0$ we carry out the backward induction to the end $\Box$

\subsection{}
We may similarly modify the delta hedging formula as
$$\Delta_n = \frac{V_{n+1}(H) - V_{n+1}(T)}{S_n(u_n - d_n)} \; \Box$$

\subsection{}
Given $S_n$, $u = \frac{S_n + 10}{S_n}$, $d = \frac{S_n - 10}{S_n}$. Then $\tilde{p} = \tilde{q} = \frac{1}{2}$. We may write $V_n = \frac{1}{2} V_{n+1}(H) + \frac{1}{2} V_{n+1}(T)$, $V_n = \frac{1}{4} V_{n+2}(HH) + \frac{1}{4} V_{n+2}(HT) + \frac{1}{4} V_{n+2}(TH) + \frac{1}{4} V_{n+2}(TT)$, and similarly for future time steps. Then
\begin{gather*}
V_0 = \frac{1}{32} ( 50 + 30 + 30 + 10 + 30 + 10 + 10 + 0 + 30 + 10 + 10 + 0 + 10 + 0 + 0 + 0\\
+ 30 + 10 + 10 + 0 + 10 + 0 + 0 + 0 + 10 + 0 + 0 + 0 + 0 + 0 + 0 + 0)\\
= 9.375 \; \Box
\end{gather*}

\section{}
\subsection{}
\begin{gather*}
\widetilde{\mathbb{P}}(H) = \frac{1 + \frac{1}{4} - \frac{1}{2}}{2 - \frac{1}{2}} = \frac{1}{2}, \widetilde{\mathbb{P}}(T) = \frac{2 - (1 + \frac{1}{4})}{2 - \frac{1}{2}} = \frac{1}{2}\\
\widetilde{\mathbb{P}}(HH | H) = \frac{1 + \frac{1}{4} - 1}{\frac{3}{2} - 1} = \frac{1}{2}, \widetilde{\mathbb{P}}(HT | H) = \frac{\frac{3}{2} - (1 + \frac{1}{4})}{\frac{3}{2} - 1} = \frac{1}{2}, \widetilde{\mathbb{P}}(TH | T) = \frac{1 + \frac{1}{2} - 1}{4 - 1} = \frac{1}{6}, \widetilde{\mathbb{P}}(TT | T) = \frac{4 - (1 + \frac{1}{2})}{4 - 1} = \frac{5}{6}\\
\widetilde{\mathbb{P}}(HH) = \frac{1}{2} \cdot \frac{1}{2} = \frac{1}{4}, \widetilde{\mathbb{P}}(HT) = \frac{1}{2} \cdot \frac{1}{2} = \frac{1}{4}, \widetilde{\mathbb{P}}(TH) = \frac{1}{2} \cdot \frac{1}{6} = \frac{1}{12}, \widetilde{\mathbb{P}}(TT) = \frac{1}{2} \cdot \frac{5}{6} = \frac{5}{12}
\end{gather*}

\subsection{}
$$V_1(H) = \frac{1}{1 + \frac{1}{4}} \left( \frac{1}{2} \cdot 5 + \frac{1}{2} \cdot 1 \right) = \frac{12}{5}, V_1(T) = \frac{1}{1 + \frac{1}{2}} \left( \frac{1}{6} \cdot 1 \right) = \frac{1}{9}$$
$$V_0 = \frac{1}{1 + \frac{1}{4}} \left( \frac{1}{2} \cdot \frac{12}{5} + \frac{1}{2} \cdot \frac{1}{9} \right) = \frac{226}{225}  \; \Box$$

\subsection{}
The variance in $V_1$ will be replicated by the variance in stock price.
$$\Delta_0 (8 - 2) = \frac{12}{5} - \frac{2}{15}$$
$$\Delta_0 = \frac{17}{45} \; \Box$$

\subsection{}
In this regime, $(S_2 - 7)^+$ varies unitarily in stock price. Thus $\Delta_1(H) = 1$ $\Box$

\section{}
\subsection{}
$M$ trivially satisfies the martingale property:
$$\widetilde{E}_n[M_{n+1}] = \widetilde{E}_n[(M_{n} + X_{n+1})] = \widetilde{E}_n[M_n] + \widetilde{E}_n[X_{n+1}] = M_n + 0 = M_n \; \Box$$

\subsection{}
\begin{equation*}
\begin{split}
\widetilde{E}_n[S_{n+1}] &= \widetilde{E}_n[e^{\sigma (M_n + X_{n+1})} \left(\frac{2}{e^\sigma + e^{-\sigma}}\right)^{n+1}]\\
&= \widetilde{E}_n[S_n e^{\sigma X_{n+1}} \frac{2}{e^\sigma + e^{-\sigma}}]\\
&= S_n \frac{2}{e^\sigma + e^{-\sigma}} \widetilde{E}_n[e^{\sigma X_{n+1}}]\\
&= S_n \frac{2}{e^\sigma + e^{-\sigma}} \cdot \frac{1}{2} \left( e^\sigma + e^{-\sigma} \right)\\
&= S_n
\end{split}
\end{equation*}
where the third line is due to $S_n$ being a $\mathcal{F}_n$-measurable function $\Box$

\section{}
We are given
$$V_N = \frac{1}{\sqrt{S_N}}$$
and recurrence
$$V_n = \frac{1}{1+r} \widetilde{E}_n[V_{n+1}] = \frac{1}{1+r} \left( \tilde{p}V_{n+1}(H) + \tilde{q}V_{n+1}(T) \right)$$
Thus we may write
$$V_{N-1} = \frac{1}{(1+r)} \left( \frac{\tilde{p}}{\sqrt{u}} + \frac{\tilde{q}}{\sqrt{d}} \right) \frac{1}{\sqrt{S_{N-1}}}$$
and by recursion
$$V_0 = \frac{1}{(1+r)^N} \left( \frac{\tilde{p}}{\sqrt{u}} + \frac{\tilde{q}}{\sqrt{d}} \right)^N \frac{1}{\sqrt{S_0}} \; \Box$$

\section{}
\subsection{}
$$V_0 = \frac{1}{(1+r)^2} \left( \tilde{p}^2 \cdot 11 + 2\tilde{p}\tilde{q} \cdot 0 + \tilde{q}^2 \cdot 0 \right) = \frac{44}{25} \; \Box$$

\subsection{}
Note $100 \cdot 1.02^{20} > 148$ but $100 \cdot 1.02^{19} \cdot 0.98 < 148$. Then in all cases but one, the option will expire worthless.
$$V_0 = \frac{1}{(1+r)^{20}} \tilde{p}^{20} (100 \cdot 1.02^{20} - 148) = 0.0026 \; \Box$$

\subsection{}
By the pricer, $V_0 = 0.380$ $\Box$

\subsection{}
By replication,
\begin{gather*}
\Delta_0(S_1(H) - S_1(T)) = V_1(H) - V_1(T)\\
\Delta_0 = \frac{V_1(H) - V_1(T)}{S_0(u - d)}
\end{gather*}
This value is calculated as part of the pricer output. For the option above, it is -0.109 $\Box$

\section*{pricer.cpp}
\lstinputlisting[language=C++]{pricer.cpp}

\end{document}

% List of tex snippets:
%   - tex-header (this)
%   - R      --> \mathbb{R}
%   - Z      --> \mathbb{Z}
%   - B      --> \mathcal{B}
%   - E      --> \mathcal{E}
%   - M      --> \mathcal{M}
%   - m      --> \mathfrak{m}({#1})
%   - normlp --> \norm{{#1}}_{L^{{#2}}}
