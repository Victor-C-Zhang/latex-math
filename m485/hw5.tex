\documentclass{article}
\usepackage[utf8]{inputenc}
\usepackage[margin=1in]{geometry}

\title{485 - Homework 5}
\author{Victor Zhang}
\date{November 5, 2021}

\usepackage[utf8]{inputenc}
\usepackage{amsmath}
\usepackage{amsfonts}
\usepackage{natbib}
\usepackage{graphicx}
% \usepackage{changepage}
\usepackage{amssymb}
\usepackage{xfrac}
% \usepackage{bm}
% \usepackage{empheq}

\newcommand{\contra}{\raisebox{\depth}{\#}}

\newenvironment{myindentpar}[1]
  {\begin{list}{}
          {
            \setlength{\leftmargin}{#1}
            \setlength{\rightmargin}{#1}
          }
          \item[]
  }
  {\end{list}}

\pagestyle{empty}

\begin{document}

\maketitle
% \begin{center}
% {\huge Econ 482 \hspace{0.5cm} HW 3}\
% {\Large \textbf{Victor Zhang}}\
% {\Large February 18, 2020}
% \end{center}

\section{}
\subsection{}
We replicate by holding $\Delta^0$ cash and $\Delta$ shares.
\begin{equation*}
\begin{cases}
\Delta^0 + 9 \Delta = 0\\
\Delta^0 + 14 \Delta = 5
\end{cases}
\end{equation*}
Then $\Delta = 1$ and $\Delta_0 = -9$. So AFP at time 0 of the call is $\Delta^0 + 10\Delta = 1$ $\Box$

\subsection{}
This is equivalent to holding a stock for 10 periods and a call with maturity 2. The price of the call is $1.6$ so the AFP at time 0 is $4 + 1.6 = 5.6$. At time 1 the call has either value 0 or 4 in the downside and upside cases respectively. Then
$$\Delta = \frac{4 - 0}{8 - 2} = \frac{2}{3}$$
and $\Delta^0 = 1.6 - \frac{2}{3} \cdot 4 = -1.07$ $\Box$

\subsection{}
\subsubsection{}
$\left\{\frac{S_n}{(1_r)^n}\right\}$ is a $\widetilde{\mathbb{P}}$-martingale so
$$\widetilde{\mathbb{E}}_2[S_5](HT) = (1+r)^5 \widetilde{\mathbb{E}}_2[\frac{S_5}{(1 + r)^5}](HT) = (1+r)^3 S_2(HT) = \frac{125}{16} \; \Box$$

\subsubsection{}
$$\widetilde{\mathbb{E}}_n[X_{n+1}] = \widetilde{\mathbb{E}}_n[\widetilde{\mathbb{E}}_{n+1}[Y]] = \widetilde{\mathbb{E}}_n[Y] = X_n$$
by tower property. So $X$ is a $\widetilde{\mathbb{P}}$-martingale $\Box$

\subsubsection{}
Yes, $\sigma$ is a stopping time since every set $\{\tau = t\}$ is measurable at time $t$ and thus so is $\{\min(\tau,1) = t\}$ $\Box$

\subsubsection{}
$$\mathbb{E}_n[f(X_{n+1})] = p f(X_n^3 u^3) + q f(X_n^3 d^3)$$
so put
$$g(x) = p f(u^3x) + q f(d^3x)$$
and $X$ is indeed a $\mathbb{P}$-Markov process $\Box$

\subsection{}
$$V_0 = \frac{1}{(1+r)^N} \widetilde{\mathbb{E}}[S_N^2 + \sqrt{S_N}] = \frac{1}{(1+r)^N}[S_0^2(\tilde{p}u^2 + \tilde{q}d^2)^N + \sqrt{S_0}(\tilde{p}\sqrt{u} + \tilde{q}\sqrt{d})^N]$$
At time 1,
$$V_1(H) = \frac{1}{(1+r)^{N-1}}[S_0^2u^2(\tilde{p}u^2 + \tilde{q}d^2)^{N-1} + \sqrt{S_0}\sqrt{u}(\tilde{p}\sqrt{u} + \tilde{q}\sqrt{d})^{N-1}]$$
and similarly for the downside case. So then
$$\Delta_0 = \frac{1}{S_0(u-d)}\frac{1}{(1+r)^{N-1}}[S_0^2(u^2-d^2)(\tilde{p}u^2 + \tilde{q}d^2)^{N-1} + \sqrt{S_0}(\sqrt{u}-\sqrt{d})(\tilde{p}\sqrt{u} + \tilde{q}\sqrt{d})^{N-1}] \; \Box$$

\subsection{}
\subsubsection{}
We are given $V_N = M_N$ so $f_N(s,m) = m$.\\
Now
$$V_n = \max\left\{M_n, \frac{1}{1+r} (\tilde{p}V_{n+1}(H) + \tilde{q}V_{n+1}(T)) \right\}$$
so put
$$f_n(s,m) = \max\left\{m, \frac{1}{1+r} (\tilde{p}f_{n+1}(su, \max(m, su)) + \tilde{q}f_{n+1}(sd, m)) \right\} \;\Box$$

\subsubsection{}
Since $r = 0$, there is no benefit in cashing out early. So it always advantageous to cash out as late as possible to absorb as much potential upside as possible. Then $\hat{\tau} = N$ $\Box$

\section{}
\subsection{}
By linearity of expectation
$$\mathbb{E}\left[\int_0^t B_s \;\mathrm{d}s \right] = \int_0^t \mathbb{E}[B_s] \;\mathrm{d}s = \int_0^t 0 \;\mathrm{d}s = 0 \; \Box$$

\subsection{}
For aribitrary $0 \leq x \leq t \leq T$ we have
\begin{equation*}
\begin{split}
\mathbb{E}_x \left[ \int_0^t B_s \;\mathrm{d}s \right] &= \mathbb{E}_x \left[ \int_0^x B_s \;\mathrm{d}s + \int_x^t B_s \;\mathrm{d}s \right]\\
&= \left( \int_0^x B_s \;\mathrm{d}s \right)+ \mathbb{E}_x \left[ \int_x^t B_s \;\mathrm{d}s \right]\\
&= \left( \int_0^x B_s \;\mathrm{d}s \right) + \int_x^t \mathbb{E}_x[B_s] \;\mathrm{d}s\\
&= \left( \int_0^x B_s \;\mathrm{d}s \right) + \int_x^t B_x \;\mathrm{d}s \neq \left( \int_0^x B_s \;\mathrm{d}s \right)
\end{split}
\end{equation*}
So $\left( \int_0^t B_s \;\mathrm{d}s \right)$ is not a martingale $\Box$

\section{}
\begin{equation*}
\begin{split}
\mathbb{P}(\exp (X) \geq K) &= \mathbb{P}(X \geq \ln K)\\
&= \mathbb{P}(\frac{X - \mu}{\sigma} \geq \frac{\ln K - \mu}{\sigma})\\
&= \mathbb{P}(- \frac{X- \mu}{\sigma} \leq \frac{-\ln K + \mu}{\sigma})\\
&= \Phi\left(\frac{\ln \frac{1}{K} + \mu}{\sigma}\right) \; \Box
\end{split}
\end{equation*}

\section{}
\subsection{}
$$\mathbb{E}[B_t^2] = \mathrm{var}(B_t) = t$$
\begin{equation*}
\begin{split}
\mathbb{E}[B_sB_t^2] &= \mathbb{E}[B_s(B_t - B_s + B_s)^2\\
&= \mathbb{E}[B_s(B_t - B_s)^2] + 2\mathbb{E}[B_s^2(B_t - B_s)] + \mathbb{E}[B_s^3]\\
&= \mathbb{E}[B_s]\mathbb{E}[(B_t - B_s)^2] + 2\mathbb{E}[B_s^2]\mathbb{E}[B_t - B_s] + \mathbb{E}[B_s^3]\\
&= 0 + 0 + 0 = 0 \; \Box
\end{split}
\end{equation*}

\subsection{}
$X$ is not a Brownian motion since $X_0 = \frac{B_0 + B_1}{2} \neq 0$.

\subsection{}
$$\mathbb{E}[X_t] = \frac{1}{2}(\mathbb{E}[B_t] + \mathbb{E}[B_{t+1}]) = 0$$
\begin{equation*}
\begin{split}
\mathbb{E}\left[ (X_t - \mathbb{E}[X_t])(X_s - \mathbb{E}[X_s]) \right] &= \mathbb{E}[X_tX_s]\\
&= \frac{1}{4}\mathbb{E}[(B_t + B_{t+1})(B_s + B_{s+1})]\\
&= \frac{1}{4}\mathbb{E}[B_tB_s + B_{t+1}{B_s} + B_tB_{s+1} + B_{t+1}B_{s+1}]\\
&= \frac{1}{4}(s + s + (s + 1) + (s + 1))\\
&= s + \frac{1}{2} \; \Box
\end{split}
\end{equation*}


\end{document}

% List of tex snippets:
%   - tex-header (this)
%   - R      --> \mathbb{R}
%   - Z      --> \mathbb{Z}
%   - B      --> \mathcal{B}
%   - E      --> \mathcal{E}
%   - M      --> \mathcal{M}
%   - m      --> \mathfrak{m}({#1})
%   - normlp --> \norm{{#1}}_{L^{{#2}}}
