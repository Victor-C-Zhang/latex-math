\documentclass{article}
\usepackage[utf8]{inputenc}
\usepackage[margin=1in]{geometry}

\title{485 - Homework 4}
\author{Victor Zhang}
\date{October 8, 2021}

\usepackage[utf8]{inputenc}
\usepackage{amsmath}
\usepackage{amsfonts}
\usepackage{natbib}
\usepackage{graphicx}
% \usepackage{changepage}
\usepackage{amssymb}
\usepackage{xfrac}
% \usepackage{bm}
% \usepackage{empheq}
\usepackage{bbm}

\usepackage{tikz}
\usetikzlibrary{trees}

\newcommand{\contra}{\raisebox{\depth}{\#}}

\newenvironment{myindentpar}[1]
  {\begin{list}{}
          {
            \setlength{\leftmargin}{#1}
            \setlength{\rightmargin}{#1}
          }
          \item[]
  }
  {\end{list}}

\pagestyle{empty}

\begin{document}

\maketitle
% \begin{center}
% {\huge Econ 482 \hspace{0.5cm} HW 3}\
% {\Large \textbf{Victor Zhang}}\
% {\Large February 18, 2020}
% \end{center}

\section{}
\subsection{}
Put $g(x) = p \cdot f(x + 1) + q \cdot f(x - 1)$.
$$\mathbb{E}_n[f(X_{n+1})] = p \cdot f(X_n + 1) + q \cdot f(X_n - 1) = g(X_n) \; \Box$$

\subsection{}
$$\mathbb{E}_n[f(|X_{n+1}|)] = \frac{1}{2} \cdot \left[ f(|X_n + 1|) + f(|X_n - 1|) \right]$$
If $X_n > 0$, the RHS may be written $\frac{1}{2} \cdot \left[ f(|X_n| + 1) + f(|X_n| - 1) \right]$.\\
If $X_n < 0$, the RHS may be written $\frac{1}{2} \cdot \left[ f(|X_n| - 1) + f(|X_n| + 1) \right]$.\\
If $X_n = 0$, the RHS is simply $f(|X_n| + 1)$.\\
So we may put $g(x) = \mathbbm{1}_{\{x = 0\}} \cdot f(x + 1) + \mathbbm{1}_{\{x \neq 0\}} \cdot \frac{1}{2} \left( f(x + 1) + f(x - 1) \right)$ and thus $|X_n|$ is a Markov process $\Box$

\subsection{}
If $p \neq q$, then the two cases $X_n > 0$, $X_n < 0$ would not have the same expression. Thus there would be no way to represent these cases as a function of $|X_n|$, since we cannot distinguish sign $\contra$

\section{}
\subsection{}
$$f_N(x) = F$$
$$f_n(x) = \mathbbm{1}_{\{x \notin [L,R]\}} \cdot F \cdot \frac{n}{N} + \mathbbm{1}_{\{x \in [L,R]\}} \cdot \frac{1}{1+r} \left( \tilde{p} f_{n+1}(ux) + \tilde{q} f_{n+1}(dx) \right)$$

\subsection{}
\tikzstyle{level 1}=[level distance=3.5cm, sibling distance=6cm]
\tikzstyle{level 2}=[level distance=3.5cm, sibling distance=3cm]
\tikzstyle{level 3}=[level distance=3.5cm, sibling distance=1.5cm]
\tikzstyle{bag} = [text width=2em, text centered]
\begin{center}
\begin{tikzpicture}[grow=right]
\node[bag] {4\\(24)}
child {
  node[bag] {2\\(0)}
  edge from parent
  child {
    node[bag] {1\\(-)}
    edge from parent
    child {
      node[bag] {$\frac{1}{2}$\\(-)}
      edge from parent
    }
    child {
      node[bag] {2\\(-)}
      edge from parent
    }
  }
  child {
    node[bag] {4\\(-)}
    edge from parent
    child {
      node[bag] {2\\(-)}
      edge from parent
    }
    child {
      node[bag] {8\\(-)}
      edge from parent
    }
  }
}
child {
  node[bag] {8\\(60)}
  edge from parent
  child {
    node[bag] {4\\(100)}
    edge from parent
    child {
      node[bag] {2\\(100)}
      edge from parent
    }
    child {
      node[bag] {8\\(100)}
      edge from parent
    }
  }
  child {
    node[bag] {16\\(50)}
    edge from parent
    child {
      node[bag] {8\\(-)}
      edge from parent
    }
    child {
      node[bag] {32\\(-)}
      edge from parent
    }
  }
};
\end{tikzpicture}
\end{center}
By backward induction, $V_0 = 24$. $\Delta_0 = \frac{60 - 0}{8 - 2} = 10$ $\Box$

\section{}
\subsection{}
$$V_N = \max(S_N - A_N, 0) = \max(S_N - S_NX_N, 0) = S_N \max(1 - X_N, 0) = S_N f_N(X_n)$$
Note we may write $X_{n+1}$ as a pure function of $X_n$ and one of $u,d$
$$X_{n+1}(H) = \frac{A_{n+1}}{S_{n+1}} = \frac{\frac{A_n(n+1) + uS_n}{n+2}}{uS_n} = X_n \frac{n+1}{u(n+2)} + \frac{1}{n+2}$$
and similarly for the tails case. Now take arbitary $n$ and assume we may write $V_i = S_i f_i(X_i)$ for $i > n$.
\begin{equation*}
\begin{split}
V_n &= \frac{1}{1+r} \left[ \tilde{p} V_{n+1}(H) + \tilde{q} V_{n+1}(T) \right]\\
&= \frac{1}{1+r} \left[ \tilde{p} u S_n f_{n+1}(X_n \frac{n+1}{u(n+2)} + \frac{1}{n+2}) + \tilde{q} d S_n f_{n+1}(X_n \frac{n+1}{d(n+2)} + \frac{1}{n+2}) \right]\\
&= S_n \frac{1}{1+r}\left[ \tilde{p} u f_{n+1}(X_n \frac{n+1}{u(n+2)} + \frac{1}{n+2}) + \tilde{q} d f_{n+1}(X_n \frac{n+1}{d(n+2)} + \frac{1}{n+2}) \right]\\
&= S_n f_n(X_n)
\end{split}
\end{equation*}
where $f_n$ is a function of $f_{n+1}$ as desired $\Box$

\subsection{}
We solve this by regular backward induction
\tikzstyle{level 1}=[level distance=3.5cm, sibling distance=4cm]
\tikzstyle{level 2}=[level distance=3.5cm, sibling distance=2cm]
\begin{center}
\begin{tikzpicture}[grow=right]
\node[bag]{4\\$\left(\frac{88}{75}\right)$}
child {
  node[bag]{2\\$\left(\frac{4}{15}\right)$}
  edge from parent
  child {
    node[bag]{1\\(0)}
    edge from parent
  }
  child {
    node[bag]{4\\$\left(\frac{2}{3}\right)$}
    edge from parent
  }
}
child {
  node[bag]{8\\$\left(\frac{40}{15}\right)$}
  edge from parent
  child {
    node[bag]{4\\(0)}
    edge from parent
  }
  child {
    node[bag]{16\\$\left(\frac{20}{3}\right)$}
  }
};
\end{tikzpicture}
\end{center}
$V_0 = \frac{88}{75}$, $\Delta_0 = \frac{\frac{40}{15} - \frac{4}{15}}{8 - 2} = \frac{2}{5}$ $\Box$

\section{}
Note $\tilde{p} = 0.6$, $\tilde{q} = 0.4$. We solve this by backward induction
\tikzstyle{bag} = [text width=3em, text centered]
\begin{center}
\begin{tikzpicture}[grow=right]
\node[bag]{100\\(3.636)}
child {
  node[bag]{80\\(5.818)}
  edge from parent
  child {
    node[bag]{64\\(16)}
    edge from parent
  }
  child {
    node[bag]{104\\(0)}
    edge from parent
  }
}
child {
  node[bag]{130\\(9.455)}
  edge from parent
  child {
    node[bag]{104\\(26)}
    edge from parent
  }
  child {
    node[bag]{169\\(0)}
  }
};
\end{tikzpicture}
\end{center}
$V_0 = 3.636$, $\Delta_0 = \frac{9.455 - 5.818}{130 - 80} = 0.0727$ $\Box$

\section{}
\subsection{}
This is not a stopping time. Take counterexample where $M_N = S_N > S_n$ for all other $n$.

\subsection{}
This is a stopping time. At time $n$, there is enough information to calculate $S_n$, $M_n$ so the set $\{\tau = n\}$ is $\mathcal{F}_n$-measurable for all $n$ $\Box$

\subsection{}
This is a stopping time, since $S_n$ is known at time $n$ and thus the set $\{\tau = n\}$ is $\mathcal{F}_n$-measurable for all $n$ $\Box$

\subsection{}
This is a stopping time, since $S_n$ and $M_n$ are both known at time $n$ and thus the set $\{\tau = n\}$ is $\mathcal{F}_n$-measurable for all $n$ $\Box$

\end{document}

% List of tex snippets:
%   - tex-header (this)
%   - R      --> \mathbb{R}
%   - Z      --> \mathbb{Z}
%   - B      --> \mathcal{B}
%   - E      --> \mathcal{E}
%   - M      --> \mathcal{M}
%   - m      --> \mathfrak{m}({#1})
%   - normlp --> \norm{{#1}}_{L^{{#2}}}
