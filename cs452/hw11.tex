\documentclass{article}
\usepackage[utf8]{inputenc}
\usepackage[margin=1in]{geometry}

\title{452 - Homework 11}
\author{Victor Zhang}
\date{April 23, 2021}

\usepackage[utf8]{inputenc}
\usepackage{amsmath}
\usepackage{amsfonts}
\usepackage{natbib}
\usepackage{graphicx}
% \usepackage{changepage}
\usepackage{amssymb}
\usepackage{xfrac}
% \usepackage{bm}
% \usepackage{empheq}

\newcommand{\contra}{\raisebox{\depth}{\#}}

\newenvironment{myindentpar}[1]
  {\begin{list}{}
          {\setlength{\leftmargin}{#1}
          \setlength{\rightmargin}{#1}}
          \item[]
  }
  {\end{list}}

\pagestyle{empty}

\begin{document}

\maketitle
% \begin{center}
% {\huge Econ 482 \hspace{0.5cm} HW 3}\
% {\Large \textbf{Victor Zhang}}\
% {\Large February 18, 2020}
% \end{center}

\section*{9.16}
According to the proof provided in the text, for any language $A \in DSPACE(n^k)$ we can find a reduction to TQBF that generates a formula of length $\mathcal{O}(n^{2k})$. By space hierarchy, there exists $A \in DSPACE(n^{5/4}) - DSPACE(n)$. If TQBF were solvable in space $n^{1/3}$, that would imply $A$ would be solvable in $\mathcal{O}(n^{2(5/4)/3} \in DSPACE(n)$, a contradiction $\contra$ 

\section*{9.19}
We provide an algorithm to solve $USAT$. Suppose the input formula $\phi$ has $n$ free variables $v_1, \dots v_n$. First run SAT on $\phi$. If it rejects, \textit{reject} immediately. Otherwise incrementally go through the set of variables. For $v_i$, run SAT on $\phi\big\vert_{v_i = 0}$ and $\phi\big\vert_{v_i = 1}$. If SAT accepts both these restricted formulae, \textit{reject}. If we go through all the variables and we haven't rejected, \textit{accept}.\\
This algorithm clearly runs in linear time on the input, given an oracle for SAT. Correctness is ensured because if SAT accepts for both $\phi\big\vert_{v_i = 0}$ and $\phi\big\vert_{v_i = 1}$, there is more than one satisfying assignment. And clearly if SAT rejects on the whole formula, there is no satisfying assignment.

\section*{9.20}
We use the same procedure as we did to find $B$ s.t. $P^B \neq NP^B$, except with a slight modification. In the proof, we took the machines $M_1, M_2 \dots \in P$. Instead, take $N_1, N_2, \dots \in \text{co-}NP$. Generate oracle $C$ in stages, as we did $B$. To build $C_i$, run $N_{i-1}^{C_{i-1}}(0^{n_i})$. If $N_{i-1}$ accepts, put $C_i = C_{i-1}$. If $N_{i-1}$ rejects, there must exist some computational path that rejects (since $N$ is co-NP). Simply add one of the strings along this rejecting path to $C_i$. For the same reasoning as in the original proof, $N_i^{C_i}(0^{n_i})$ does not accept $L(C_i)$. Then clearly $L(C) \notin \text{co-}NP^C$. However, $L(C) \in NP^C$, since we may simply guess an element of $C$ for each query. This completes the proof $\Box$

\section*{9.21.a}
Since we may reduce every NP language to $SAT$, we can solve every NP problem by doing this reduction then query $SAT$ once and outputting its result. Since we may reduce every co-NP language to $\overline{SAT}$, we can solve every co-NP problem by doing this reduction then query $SAT$ once and output the inverse of its result. Thus both NP, co-NP are $P^{SAT,1}$ $\Box$

\section*{9.21.b}
We generate language $A = \{w\;:\; w = 0w_1, w_1 \in SAT \text{ or } w = 1w_2, w_2 \in \overline{SAT}\}$. Clearly, we can decide this language with one query to $SAT$: simply query $SAT$ on the input excluding the first bit, and decide whether or not to flip the result of $SAT$. Hence, $A \in P^{SAT,1}$. It is also clear we may poly-time reduce both $SAT$ and $\overline{SAT}$ to $A$. Hence, $A$ is both NP-hard and co-NP-hard. Since $NP \neq \text{co-}NP$, this implies $A \notin NP$ and $A \notin \text{co-}NP$. Thus, $NP \cup \text{co-}NP \subsetneq P^{SAT,1}$, as desired $\Box$

\section*{10.19}
If $NP \subseteq BPP$ then there exists probabilistic polytime machine $M$ for $SAT$ that answers correctly with high constant probability. We generate machine $M'$ that answers incorrectly with probability $< \frac{1}{2^n}$ by simply running $M$ $n$ times. We may build a RP machine for $SAT$ as follows:
\begin{myindentpar}{1em}
Suppose the input $\phi$ consists of $n$ variables $v_1, \dots v_n$. Run $M'$ on $\phi\big\vert_{v_1 = 0}$. If $M'$ accepts, set $v_1 = 0$ and continue. If $M$ rejects, set $v_1 = 1$ and continue. For each variable $v_i$, similarly run $M'$ on $\phi\big\vert_{v_1, \dots v_{i-1}, v_i = 0}$ using the values of $v_j, j < i$ we have already set. At the end, test if our choices of $v_1,\dots v_n$ give a satisfying assignment. If it does, \textit{accept}. If it doesn't, \textit{reject}.
\end{myindentpar}
Clearly this machine never accepts if the formula is unsatisfiable. If the formula is satisfiable, at each step $i$ we will generate a variable in a satisfying assignment with probability $> 1 - \frac{1}{2^n}$. The probability that we arrive at the end having generated anything incorrectly is $< n \frac{1}{2^n} < \frac{1}{2}$, by union bound. Since $M$ is polytime, $M'$ is polytime as well since we just run $M$ $n$ times. And since our algorithm runs $M'$ $n$ times, the machine we built is indeed polytime. Thus, $SAT \in RP$. To finish, we note $RP \subseteq NP$ by definition. Now since $SAT$ is NP-complete, in fact $NP \subseteq RP$ and hence $NP = RP$ $\Box$

\end{document}

% List of tex snippets:
%   - tex-header (this)
%   - R      --> \mathbb{R}
%   - Z      --> \mathbb{Z}
%   - B      --> \mathcal{B}
%   - E      --> \mathcal{E}
%   - M      --> \mathcal{M}
%   - m      --> \mathfrak{m}({#1})
%   - normlp --> \norm{{#1}}_{L^{{#2}}}
