\documentclass{article}
\usepackage[utf8]{inputenc}
\usepackage[margin=1in]{geometry}

\title{452 - Homework 10}
\author{Victor Zhang}
\date{April 16, 2021}

\usepackage[utf8]{inputenc}
\usepackage{amsmath}
\usepackage{amsfonts}
\usepackage{natbib}
\usepackage{graphicx}
% \usepackage{changepage}
\usepackage{amssymb}
\usepackage{xfrac}
% \usepackage{bm}
% \usepackage{empheq}

\newcommand{\contra}{\raisebox{\depth}{\#}}

\newenvironment{myindentpar}[1]
  {\begin{list}{}
          {\setlength{\leftmargin}{#1}}
          \item[]
  }
  {\end{list}}

\pagestyle{empty}

\begin{document}

\maketitle
% \begin{center}
% {\huge Econ 482 \hspace{0.5cm} HW 3}\
% {\Large \textbf{Victor Zhang}}\
% {\Large February 18, 2020}
% \end{center}

\section*{8.30}
To show $E_{\mathrm{DFA}} \in NL$ we first show $\overline{E_{\mathrm{DFA}}} \in NL$. Then $E_{\mathrm{DFA}} \in \text{co-}NL = NL$. For given input DFA with $n$ states, nondeterministically guess successive letters of input and simulate the machine on this input. If we reach an accepting state, \textit{accept}. If we guess more than $n$ letters without accepting, \textit{reject}. Each letter of input requires constant space, and keeping a counter of input letters requires $\log n$ space. Simulating the machine can be done by keeping a marker with the current state, which can be encoded in $\log n$ space. Thus, this is indeed a logspace algorithm. Correctness is guaranteed, since if there is an accepting string with more than $n$ letters, there must be an accepting string with $\leqslant n$ letters, by the pumping lemma.\\
Now we show completeness by reducing PATH to $E_{\mathrm{DFA}}$. On input $\langle (V,E),s,t \rangle$ simply transform the input into a DFA. Create nodes $V \cup {r}$, where $r$ is an absorbing rejecting state. Generate a unique transition symbol $s_{x,y}$ for each edge $(x,y) \in E$. Have transitions $x \to y$ on symbol $s_{x,y}$ and transitions $x \to r$ on any other symbol. Finally put the start state at $s$ and the sole accepting state at $t$. This DFA has an accepting string iff there is a path $s \to t$. We have shown how to transform PATH-graphs to NFAs, and that transformation can be done in logspace. To label our edges, it suffices to count the number of edges, which we can do in logspace, and dynamically assign symbols of length $\mathcal{O}(\log n)$ to these edge-transitions. To label the edges to $r$, it suffices to loop through all the symbols and add these symbols as edge transitions. We may do this by simply keeping a counter, which also takes logspace. Thus we have indeed shown a logspace reduction, so we are done $\Box$

\section*{9.7.e}
$(0^*1)^{500}0^*$

\section*{9.7.f}
$(0^*1)^{500}(0\cup1)^*$

\section*{9.7.g}
$(0^* \cup 0^*1)^{500}0^*$

\section*{9.9}
Note $A \in P^{SAT}$ iff $\overline{A} \in P^{SAT}$, since we may simply invert the output of one to get the output of another. Then $A \in NP = P^{SAT}$ iff $\overline{A} \in P^{SAT} = NP$, in other words $A \in NP$ iff $A \in \text{co-}NP$. Thus $NP = \text{co-}NP$ $\Box$

\section*{9.12}
Every language would be poly-time reducible to SAT, but there's no guarantee the reduction will be quick. We cannot conclude $NP \subseteq TIME(n^k)$ for any $k$, since by time hierarchy, there exists some language which runs in time $n^{k^2}$ and thus cannot be reduced to SAT in less than that amount of time.

\section*{9.13}
Suppose we have an algorithm that solves $A$ in $n^6$ time. We may solve $pad(A,n^2)$ by simply checking that the provided string is indeed of the correct format ($m$ character input to $A$, followed by $m^2-m$ \verb|#| symbols) and running our algorithm for $A$ on the first $m$ characters. This algorithm runs in time $m^6$. But since the length of input to $pad(A,n^2)$ is $n = m^2$, the algorithm runs in time $n^3$ $\Box$

\section*{9.14}
Pick arbitrary language $A$ and suppose it runs in some time $t(n)$. Then consider the language $pad(A,t(n))$. A similar analysis to the previous problem shows this runs in time $t(m)$, where $m$ is the length of the input string for $A$. But since $n \geqslant t(m)$, $pad(A,t(n))$ runs in polynomial time, in particular linear time.\\
Now if $NEXPTIME \neq EXPTIME$ we can find language $A$ that is only in one complexity class. WLOG suppose it is NEXPTIME and runs in $t(n)$. Then $pad(A,t(n)) \in NP$. But since $A \notin EXPTIME$, there is no deterministic exptime algorithm for $A$. This implies there is no polytime algorithm for $pad(A,t(n))$, since $pad(A,t(n)) \leqslant_m^p A$. We have shown a language which is in one complexity class and not the other, so we are done $\Box$

\section*{Supplemental 1}
For all $B \in EXPTIME$, $B \leqslant_m^p A \in DTIME(s(n))$ so there is some $k$ s.t. $B \in DTIME(p(n) + s(n^k))$, where $p(n)$ is some polynomial representing the time spent on reduction. Assume $A \in \bigcap_\epsilon DSPACE(2^{n^\epsilon})$. We know there is some $B \in DTIME(2^{n^2}) - DTIME(2^{n})$ by time hierarchy. Then for all $\epsilon > 0$, $B \in DTIME(p(n) + 2^{(n^k)^\epsilon})$. But if we pick $\epsilon = \frac{1}{k+1}$ this implies $B \in DTIME(p(n)+2^{n^{\sfrac{k}{(k+1)}}}) \subseteq DTIME(2^{n^k})$, a contradiction $\contra$

\section*{Supplemental 2}
Suppose $B$ is complete for EXPTIME and runs in time $2^{n^k}$. There exist complete languages for EXPTIME so pick $B$ to be any of these. If $\epsilon > k$, we are done. Otherwise, generate language $pad(B,n^{k/\epsilon})$ in polytime. We may solve this language by running the algorithm for $B$ on the beginning $m$-length input. This runs in time $2^{m^k} = 2^{n^\epsilon}$. We have thus shown $B \leqslant_m^p pad(B,n^{k/\epsilon})$. So since $B$ is complete under poly-time reductions, we may reduce every language to $pad(B,n^{k/\epsilon})$ by composing the reduction to $B$ with the reduction from $B$ to $pad(B,n^{k/\epsilon})$, which is a polytime reduction. Thus, $A = pad(B,n^{k/\epsilon})$ is EXPTIME-complete $\Box$

\section*{Supplemental 3}
We reduce $A$ to PATH by first checking that $y \in \overline{PATH}$. If $y \in \overline{PATH}$, pass $x$ to PATH. Otherwise, pass a simple non-accepting case to PATH. For instance, we can take $x$ to be the disconnected graph with 2 nodes and $s,t$ to be these two nodes. Since $NL = \text{co-}NL$, we may check if $y \in \overline{PATH}$ in logspace. Storing the non-accepting case requires constant space, so indeed this is a logspace reduction.\\
We reduce PATH to $A$ by simply adding a degenerate case $y \notin PATH$. Formally, suppose the input to PATH is $x$. Pass $\langle x,y \rangle$ to $A$. For instance, we can take $y$ to be the (same) disconnected graph with 2 nodes. We can store this configuration in constant space, so indeed $PATH \leqslant_m^{\log} A$.\\
$A$ is NL-hard and we may reduce $A \leqslant_m^{\log} PATH$, so in fact $A \in NL$ and is thus NL-complete $\Box$

\end{document}

% List of tex snippets:
%   - tex-header (this)
%   - R      --> \mathbb{R}
%   - Z      --> \mathbb{Z}
%   - B      --> \mathcal{B}
%   - E      --> \mathcal{E}
%   - M      --> \mathcal{M}
%   - m      --> \mathfrak{m}({#1})
%   - normlp --> \norm{{#1}}_{L^{{#2}}}
