\documentclass{article}
\usepackage[utf8]{inputenc}
\usepackage[margin=1in]{geometry}

\title{452 - Homework 1}
\author{Victor Zhang}
\date{January 26, 2021}

\usepackage[utf8]{inputenc}
\usepackage{amsmath}
\usepackage{amsfonts}
\usepackage{natbib}
\usepackage{graphicx}
% \usepackage{changepage}
\usepackage{amssymb}
\usepackage{xfrac}
% \usepackage{bm}
% \usepackage{empheq}

\newcommand{\contra}{\raisebox{\depth}{\#}}

\newenvironment{myindentpar}[1]
  {\begin{list}{}
          {\setlength{\leftmargin}{#1}}
          \item[]
  }
  {\end{list}}

\pagestyle{empty}

\begin{document}

\maketitle
% \begin{center}
% {\huge Econ 482 \hspace{0.5cm} HW 3}\
% {\Large \textbf{Victor Zhang}}\
% {\Large February 18, 2020}
% \end{center}

\section*{0.7.a}
Consider the relation $R$ on $\mathbb{N}$ given by $\{xRy : \gcd(x,y) \neq 1\}$. Clearly $R$ is reflexive and symmetric. $4R6$ and $6R9$ but 4 and 9 are coprime, so $R$ is not transitive.
\section*{0.7.b}
The relation $\leq$ on $\mathbb{Z}$ is clearly reflexive and transitive, but not symmetric for nonequal integers.
\section*{0.7.c}
The relation $R = \{(0,0),(0,1),(1,0),(1,1)\}$ on the set $\{0,1,2\}$ is transitive and symmetric, but not reflexive, since $(2,2) \notin R$.

\section*{0.9}
$G = (A\cup B, A \times B)$

\section*{0.10}
Since $a = b$ the value $a - b$ is necessarily zero. Thus, it is an error to divide by $a - b$ $\Box$

\section*{0.12}
The induction step fails to account for the case where $H_1$ and $H_2$ are disjoint and thus may have different colors. This occurs when $|H| = 2$ $\Box$

\section*{0.13}
Let a graph $G(V,E)$ have $v$ vertices. Suppose every vertex has a different degree. The maximum degree possible is $v-1$, so the vertices must have degrees $0,1,\dots v-1$. But that means there is one vertex which is connected to every other vertex (degree $v-1$) and one vertex which is connected to no other vertices (degree 0) $\contra$

\end{document}

% List of tex snippets:
%   - tex-header (this)
%   - R      --> \mathbb{R}
%   - Z      --> \mathbb{Z}
%   - B      --> \mathcal{B}
%   - E      --> \mathcal{E}
%   - M      --> \mathcal{M}
%   - m      --> \mathfrak{m}({#1})
%   - normlp --> \norm{{#1}}_{L^{{#2}}}
