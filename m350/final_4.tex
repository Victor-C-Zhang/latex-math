\documentclass{article}
\usepackage[utf8]{inputenc}

\topmargin -.5in
\textheight 9in
\title{350H - Final}
\author{Victor Zhang}
\date{May 12, 2020}

\usepackage[utf8]{inputenc}
\usepackage{amsmath}
\usepackage{amsfonts}
\usepackage{natbib}
\usepackage{graphicx}
% \usepackage{changepage}
\usepackage{amssymb}
% \usepackage{bm}
% \usepackage{empheq}

\newcommand{\contra}{\raisebox{\depth}{\#}}

\newenvironment{myindentpar}[1]
  {\begin{list}{}
          {\setlength{\leftmargin}{#1}}
          \item[]
  }
  {\end{list}}

\pagestyle{empty}

\begin{document}

\maketitle
% \begin{center}
% {\Large 350H \hspace{0.5cm} MT2}\\
% {\Large \textbf{Victor Zhang}}\\
% {\Large April 10, 2020}
% \end{center}

\section*{4.}
First note that the composition of two self-adjoint matrices is itself self-adjoint, so both $TU$ and $UT$ are self-adjoint. Since $T,U$ are self-adjoint they admit an orthonormal basis of eigenvectors. Thus if we put $n = \dim V$, $V = E_{\lambda_1} \oplus \dots \oplus E_{\lambda_n} = E_{\mu_1} \oplus \dots \oplus E_{\mu_n}$ for eigenvalues (not necessarily distinct) $\lambda_i$ of $U$, $\mu_i$ of $T$. So for eigenvalue $\lambda$ of $U$, $v \in E_\lambda$, there is some $\mu$ of $T$ s.t. $Uv \in E_\mu$. Then $\mu\lambda$ is an eigenvalue of $TU$. Let $\beta$ be the set of all eigenvalues $\mu_j\lambda_i$. We are guaranteed to have $n$ of these, since each $\lambda_i$ maps to some $\mu_j\lambda_i$ and $TU$ is diagonalizable. Then for particular $\mu$, $\lambda$, $E_{\mu\lambda} \subseteq E_\lambda$. By problem (3), since $\mu\lambda$ is an eigenvalue of $UT$ as well, $E_{\mu\lambda} \subseteq E_\mu$. Moreover, since there are the same number of eigenvectors (and thus eigenspaces) of $TU$, $UT$, $T$, and $U$, it follows that $E_{\lambda\mu} = E_\lambda = E_\mu$. That is, there is a bijection between the eigenvalues of $U$, $T$, and $TU$. Let $\beta$ be an orthonormal basis of eigenvectors of $TU$. Then $\beta$ is also an ONB of eigenvectors of $T$ and $U$ $\Box$

\end{document}