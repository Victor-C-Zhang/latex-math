\documentclass{article}
\usepackage[utf8]{inputenc}

\topmargin -.5in
\textheight 9in
\title{350H - Final}
\author{Victor Zhang}
\date{May 12, 2020}

\usepackage[utf8]{inputenc}
\usepackage{amsmath}
\usepackage{amsfonts}
\usepackage{natbib}
\usepackage{graphicx}
% \usepackage{changepage}
\usepackage{amssymb}
% \usepackage{bm}
% \usepackage{empheq}

\newcommand{\contra}{\raisebox{\depth}{\#}}

\newenvironment{myindentpar}[1]
  {\begin{list}{}
          {\setlength{\leftmargin}{#1}}
          \item[]
  }
  {\end{list}}

\pagestyle{empty}

\begin{document}

\maketitle
% \begin{center}
% {\Large 350H \hspace{0.5cm} MT2}\\
% {\Large \textbf{Victor Zhang}}\\
% {\Large April 10, 2020}
% \end{center}

\section*{2.}
Clearly, every eigenvalue must have geometric multiplicity at least 1. And since each $E_\lambda$ is pairwise disjoint (except for the 0 vector) it follows that for the eigenvalues $\lambda_1, \dots \lambda_n$ we can find eigenvectors $v_1, \dots v_n$ respectively s.t. $\beta = \{ v_1,\dots v_n\}$ is linearly independent and thus is an eigenbasis for $V$. Diagonalizability is then clear $\Box$

\end{document}