\documentclass{article}
\usepackage[utf8]{inputenc}

\topmargin -.5in
\textheight 9in
\title{350H - MT2}
\author{Victor Zhang}
\date{April 10, 2020}

\usepackage[utf8]{inputenc}
\usepackage{amsmath}
\usepackage{amsfonts}
\usepackage{natbib}
\usepackage{graphicx}
% \usepackage{changepage}
\usepackage{amssymb}
% \usepackage{bm}
% \usepackage{empheq}

\newcommand{\contra}{\raisebox{\depth}{\#}}

\newenvironment{myindentpar}[2]
  {\begin{list}{}
          {\setlength{\leftmargin}{#1}
          \setlength{\rightmargin}{#2}}
          \item[]
  }
  {\end{list}}

\pagestyle{empty}

\begin{document}

\maketitle
% \begin{center}
% {\huge Math 350H \hspace{0.5cm} MT2}\\
% {\Large \textbf{Victor Zhang}}\\
% {\Large April 9, 2020}
% \end{center}

\section*{4.}
We show the following lemma:
\begin{myindentpar}{1em}{0em}
    Lemma. If $T: V\to V$ and $p_1, p_2, \dots p_n$ are pairwise coprime polynomials, that is, they share no roots, then $N(\left(\prod\limits_{j=1}^n p_j \right)(T)) = \bigotimes\limits_{j=1}^n N(p_j(T))$
\end{myindentpar}
\begin{myindentpar}{2em}{2em}
    Proof. Note it suffices to show the statement for $n = 2$, since pairwise coprime implies disjoint compositions are also coprime.\\
    By Bezout's identity we can find polynomials $a,b$ s.t. $ap_1(x) + bp_2(x) \equiv 1$. This holds over all $\mathbb{F}[x]$, so in particular if we take $x = T$ then
    $$(a(T))(p_1(T)) + (b(T))(p_2(T)) = I$$ 
    So put $x \in N(p_1p_2)$, then notice $x = (a(T))(p_1(T))x + (b(T))(p_2(T))x$. Put $x_1 = (a(T))(p_1(T))x$ and $x_2 = (b(T))(p_2(T))x$. Then observe $(p_2(T))x_1 = (p_2(T))(a(T))(p_1(T))x_1 = (a(T))(p_1p_2(T))x_1 = 0$, so $x_1 \in N(p_2(T))$. Similarly, $x_2 \in N(p_1(T))$. Thus $N(p_1p_2(T)) = N(p_1(T)) + N(p_2(T))$. But if $x \in N(p_1(T)) \cap N(p_2(T))$ then $x = x_1 + x_2 = 0$. So in fact the sum is direct, i.e. $N(p_1p_2(T)) = N(p_1(T)) \otimes N(p_2(T))$ $\Box$
\end{myindentpar}
Now by Cayley-Hamilton $p_T(T) = \prod (T-\lambda_j I)^{m_j} \equiv 0$ so applying the lemma,
$$N(p_T(T)) = K_{\lambda_1} \otimes K_{\lambda_2} \otimes \dots \otimes K_{\lambda_n}$$
But $N(p_T(T) = N(0) = V$, so we are done $\Box$


\end{document}