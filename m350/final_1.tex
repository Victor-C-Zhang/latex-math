\documentclass{article}
\usepackage[utf8]{inputenc}

\topmargin -.5in
\textheight 9in
\title{350H - Final}
\author{Victor Zhang}
\date{May 12, 2020}

\usepackage[utf8]{inputenc}
\usepackage{amsmath}
\usepackage{amsfonts}
\usepackage{natbib}
\usepackage{graphicx}
% \usepackage{changepage}
\usepackage{amssymb}
% \usepackage{bm}
% \usepackage{empheq}

\newcommand{\contra}{\raisebox{\depth}{\#}}

\newenvironment{myindentpar}[1]
  {\begin{list}{}
          {\setlength{\leftmargin}{#1}}
          \item[]
  }
  {\end{list}}

\pagestyle{empty}

\begin{document}

\maketitle
% \begin{center}
% {\Large 350H \hspace{0.5cm} MT2}\\
% {\Large \textbf{Victor Zhang}}\\
% {\Large April 10, 2020}
% \end{center}

\section*{1.}
The characteristic polynomial $p_A(\lambda) = (2-\lambda)(1-\lambda)^2$. If we examine $\lambda = 1$, we get two eigenvectors $\left( \begin{matrix} 1\\0\\1 \end{matrix}\right), \left( \begin{matrix} 1\\-1\\1 \end{matrix}\right)$. If we examine $\lambda = 2$, we get one eigenvector $\left( \begin{matrix} 1\\-2\\1 \end{matrix}\right)$. Thus we may diagonalize
$$A = B^{-1}\left( \begin{matrix} 1 & 0 & 0 \\ 0 & 1 & 0 \\ 0 & 0 & 2 \end{matrix}\right)B$$
for some $B$. Conveniently, 
$$A^n = B^{-1} \left( \begin{matrix} 1^n & 0 & 0 \\ 0 & 1^n & 0 \\ 0 & 0 & 2^n \end{matrix}\right) B$$
so $tr(A^n) = 2 + 2^n$ $\Box$

\end{document}