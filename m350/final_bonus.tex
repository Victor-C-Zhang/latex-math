\documentclass{article}
\usepackage[utf8]{inputenc}

\topmargin -.5in
\textheight 9in
\title{350H - Final}
\author{Victor Zhang}
\date{May 12, 2020}

\usepackage[utf8]{inputenc}
\usepackage{amsmath}
\usepackage{amsfonts}
\usepackage{natbib}
\usepackage{graphicx}
% \usepackage{changepage}
\usepackage{amssymb}
% \usepackage{bm}
% \usepackage{empheq}

\newcommand{\contra}{\raisebox{\depth}{\#}}

\newcommand{\innerprod}[2]{\left\langle #1 , #2 \right\rangle}
\newcommand{\norm}[1]{\left\lVert#1\right\rVert}

\newenvironment{myindentpar}[1]
  {\begin{list}{}
          {\setlength{\leftmargin}{#1}}
          \item[]
  }
  {\end{list}}

\pagestyle{empty}

\begin{document}

\maketitle
% \begin{center}
% {\Large 350H \hspace{0.5cm} MT2}\\
% {\Large \textbf{Victor Zhang}}\\
% {\Large April 10, 2020}
% \end{center}

\section*{Bonus.}
First recall that if $T$, $U$ are operators for which the adjoint is defined,
$$(AB)^* = B^*A^*$$
This is easily seen since for any $x,y \in V$
$$\innerprod{x}{ABy} = \innerprod{A^*x}{By} = \innerprod{B^*A^*x}{y}$$
Then $(TT)^* = T^*T^* = I^* = I$, so $(T^*)^2 = I$. Since $TT^* = T^*T$, it is easy to see that $(T^*T)^2 = I$.\\
Put $S = T^*T$, let $v \in V$, and define $w = v - Sv$. We may see
$$Sw = Sv - S^2v = Sv - v = -w$$
If we take $\norm{\cdot}$ to be the norm induced by $\innerprod{\cdot}{\cdot}$,
\begin{equation*}
\begin{split}
    0 \geq -\norm{w}^2 &= \innerprod{Sw}{w}\\
    &= \innerprod{T^*Tw}{w}\\
    &= \innerprod{Tw}{Tw}\\
    &= \norm{Tw}^2 \geq 0
\end{split}
\end{equation*}
So $w = v - Sv = 0$. Since $v$ is arbitrary, $S = T^*T = I$ and thus $T =T^*$ as desired $\Box$
\end{document}