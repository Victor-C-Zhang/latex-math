\documentclass{article}
\usepackage[utf8]{inputenc}

\title{350 - Homework 5}
\author{Victor Zhang}
\date{March 9, 2020}

\usepackage[utf8]{inputenc}
\usepackage{amsmath}
\usepackage{amsfonts}
\usepackage{natbib}
\usepackage{graphicx}
% \usepackage{changepage}
\usepackage{amssymb}
% \usepackage{bm}
% \usepackage{empheq}

\newcommand{\contra}{\raisebox{\depth}{\#}}

\newenvironment{myindentpar}[1]
  {\begin{list}{}
          {\setlength{\leftmargin}{#1}}
          \item[]
  }
  {\end{list}}

\pagestyle{empty}

\begin{document}

\maketitle
% \begin{center}
% {\huge Econ 482 \hspace{0.5cm} HW 3}\\
% {\Large \textbf{Victor Zhang}}\\
% {\Large February 18, 2020}
% \end{center}

\section{}
Note the left-hand matrix can be obtained from the right-hand matrix through elementary row operations. In particular, let the left-hand matrix be $B$ and the right-hand matrix be $A$. Then
$$B = E_{2\rightleftarrows3}E_{3\rightarrow1}E_{2\rightarrow1}E_{1,-2}E_{1\rightarrow3}E_{1\rightarrow2}A$$
Adding rows to other rows does not change the determinant of a matrix. Multiplying a row by a scalar multiplies the determinant by that scalar. Switching rows negates the determinant. So $\det B = 2\det A$ $\Box$

% 1 2 3

% 1 2+1 3+1
% -2 2+1 3+1
% 2+3 2+1 3+1
% 2+3 1+3 1+2

\section{}
\subsection{}
Note for $x,y \in \mathbb{C}$, $\overline{x} + \overline{y} = \overline{x+y}$ and $\overline{x}\cdot\overline{y} = \overline{xy}$. Now apply induction on $n = \dim A$:\\
Take base case where $n = 2$. Put $A = \left( \begin{matrix} a & b \\ c & d \end{matrix} \right)$. Then $\det(\overline{A}) = \overline{a}\cdot\overline{d}-\overline{b}\cdot\overline{c} = \overline{ad-bc} = \overline{\det(A)}$. Thus the statement is proved for $n = 2$.\\
Now assume we have proved the statement for $i < n$. Then by cofactor expansion,
$$\det (\overline{A}) = \sum\limits_{j=1}^n (-1)^{i+j} \overline{a}_{ij} \det (\widehat{\overline{A}_{ij}}) = \sum\limits_{j=1}^n (-1)^{i+j} \overline{a}_{ij} \overline{\det (\widehat{A_{ij}})} = \overline{\det(A)}$$
and we are done $\Box$
\subsection{}
Since taking a transpose does not change the determinant,
\begin{equation*}
  \begin{split}
    \det (I) = 1 &= \det (QQ^*)\\
    &= \det (Q) \det (Q^*)\\
    &= \det (Q) \det (Q)\\
  \end{split}
\end{equation*}
Thus $|\det (Q)| = 1$ $\Box$

\section{}
First suppose $\beta$ is a basis for $\mathbb{F}^n$. Then note invertible change of basis $[T]_\alpha^\beta = B$, where $\alpha$ is the standard basis. Then clearly $\det B \neq 0$ by invertibility.\\
Now suppose $\det B \neq 0$. Since we can perform cofactor expansion along columns like along rows, column operations have the same effect on the determinant as row operations. This implies $\beta$ is a basis, since if there is some linearly dependent subset $\{b_k\} \subseteq \beta$ we may perform column operations to transform all the corresponding rows $k$ to zero and hence $\det B = 0$ $\Box$

\section{}
We prove by induction on $n = \dim C$:\\
Take base case $n = 1$. By cofactor expansion along the last row, $\det(M) = (-1)^{2k}c_{1,1}\det(A) = \det(A)\cdot\det(C)$, where $k = \dim(A)$.\\
Now assume the result holds for $i<n$. Then taking cofactor expansion along the last row,
$$\det(M) = \sum\limits_{j=1}^{n} c_{n,j} \det(\widehat{M_{n+k,j+k}}) = \sum\limits_{j=1}^{n} c_{n,j} \det(A)\cdot\det(\widehat{C_{n,j}}) = \det(A)\det(C)$$
as desired $\Box$.

\end{document}