\documentclass{article}
\usepackage[utf8]{inputenc}

\topmargin -.5in
\textheight 9in
\title{350H - Final}
\author{Victor Zhang}
\date{May 12, 2020}

\usepackage[utf8]{inputenc}
\usepackage{amsmath}
\usepackage{amsfonts}
\usepackage{natbib}
\usepackage{graphicx}
% \usepackage{changepage}
\usepackage{amssymb}
% \usepackage{bm}
% \usepackage{empheq}

\newcommand{\contra}{\raisebox{\depth}{\#}}

\newcommand{\innerprod}[2]{\left\langle #1 , #2 \right\rangle}
\newcommand{\norm}[1]{\left\lVert#1\right\rVert}

\newenvironment{myindentpar}[1]
  {\begin{list}{}
          {\setlength{\leftmargin}{#1}}
          \item[]
  }
  {\end{list}}

\pagestyle{empty}

\begin{document}

\maketitle
% \begin{center}
% {\Large 350H \hspace{0.5cm} MT2}\\
% {\Large \textbf{Victor Zhang}}\\
% {\Large April 10, 2020}
% \end{center}

\section*{5.1}
Suppose $A = \left( \begin{matrix} a_1 & a_2 \\ a_3 & a_4 \end{matrix}\right)$, $B = \left( \begin{matrix} b_1 & b_2 \\ b_3 & b_4 \end{matrix}\right)$. Then
$$\innerprod{A}{B} = tr[\left( \begin{matrix} b_1 & b_3 \\ b_2 & b_4 \end{matrix}\right)\left( \begin{matrix} a_1 & a_2 \\ a_3 & a_4 \end{matrix}\right)] = a_1b_1 + a_2b_2 + a_3b_3 + a_4b_4$$
But 
$$T(A) = \left( \begin{matrix} a_4 & a_1 \\ a_2 & a_3 \end{matrix}\right)$$
so it follows that
$$\innerprod{T(A)}{T(B)} = a_4b_4 + a_1b_1 + a_2b_2 + a_3b_3 = \innerprod{A}{B}$$
and thus $T$ is orthogonal $\Box$

\section*{5.2}
Note $\innerprod{T(A)}{B} = a_4b_1 + a_1b_2 + a_2b_3 + a_3b_4 = a_1b_2 + a_2b_3 + a_3b_4 + a_4b_1$. So if $\innerprod{T(A)}{B} = \innerprod{A}{T^*(B)}$ then $T^*(B) = \left( \begin{matrix} b_2 & b_3 \\ b_4 & b_1 \end{matrix}\right)$. Hence $T* = T^{-1}$. It is clear that the adjoint is defined by
$$T^*(A) = A\left( \begin{matrix} 0 & 0 \\ 1 & 0 \end{matrix}\right) + \left( \begin{matrix} 0 & 1 \\ 1 & 0 \end{matrix}\right) A \left( \begin{matrix} 0 & 1 \\ 0 & 0 \end{matrix}\right) \; \Box$$

\end{document}