\documentclass{article}
\usepackage[utf8]{inputenc}

\title{411 - Homework 7}
\author{Victor Zhang }
\date{November 1, 2019}

\usepackage[utf8]{inputenc}
\usepackage{amsmath}
\usepackage{amsfonts}
\usepackage{natbib}
\usepackage{graphicx}
\usepackage{changepage}
\usepackage{amssymb}

\newcommand{\contra}{\raisebox{\depth}{\#}}

\begin{document}

\maketitle

\section{}
\subsection{}
Since $x_n \rightarrow 0$ we can take $\frac{\epsilon}{2}>0$ and find $N$ s.t. $x_n < \frac{\epsilon}{2}$ for $n\geq N$. Then $y_n = \frac{\sum\limits_{i=0}^N x_i}{n} + \frac{\sum\limits_{i = N+1}^n x_i}{n} \leq \frac{\sum\limits_{i=0}^N x_i}{n} + \frac{n-N}{n}\frac{\epsilon}{2}$. Since $\frac{\sum\limits_{i=0}^N x_i}{n}$ is just a constant $c$, $(\frac{c}{n})$ converges to 0 and thus we can find $N_1$ s.t. $|\frac{c}{n}| < \frac{\epsilon}{2}$ for $n \geq N_1$. Then note $\frac{n-N}{n} < 1$ so for $n\geq N_1$, $y_n < \frac{\epsilon}{2} + \frac{\epsilon}{2} = \epsilon$. Thus, $y_n \rightarrow 0$ $\Box$

\subsection{}
No. Take sequence $(x_n)$ s.t. $x_n = (-1)^n$. $0\leq y_n \leq \frac{1}{n}$ so $y_n \rightarrow 0$ but $x_n$ does not converge.

\section{}
First we show $0<x_n<1$ for all $n$ by induction:\\
Take base case $n=1$. Trivially, $0<x_1<1$ so the base case is proved. Now take arbitrary $n\in\mathbb{N}$ and assume $0<x_{n-1}<1$. Then $x_n = x_{n-1}(1-x_{n-1})$. Since $x_n$ is the product of two numbers between 0 and 1, $0<x_n<1$.\\
Note also that since $0<x_{n-1} < 1$ for all $n$, $x_n = x_{n-1}(1-x_{n-1}) < x_{n-1}$. So $(x_n)$ is monotone decreasing and bounded below. Thus, it has a limit $L$. We can apply the polynomial limit property:
$$x_n = x_{n-1} - x_{n-1}^2$$
$$x_{n+1} = x_{n} - x_{n}^2$$
$$\lim\limits_{n\rightarrow\infty} x_{n+1} = \lim\limits_{n\rightarrow\infty} (x_{n} - x_{n}^2)$$
$$\lim\limits_{n\rightarrow\infty} x_{n+1} = \lim\limits_{n\rightarrow\infty} x_{n} - (\lim\limits_{n\rightarrow\infty} x_{n})^2$$
$$L = L - L^2$$
Note we can say $\lim\limits_{n\rightarrow\infty} x_{n+1} = \lim\limits_{n\rightarrow\infty} x_{n}$ since $(x_{n+1})$ is a subsequence of $(x_n)$ and so the limit $l$ must exist, $l\leq\lim\limits_{n\rightarrow\infty}\sup x_n$, $l\geq\lim\limits_{n\rightarrow\infty}\inf x_n$ so $l = L$. So $L = 0$ and thus $(x_n) \rightarrow 0$ $\Box$

\section{}
\subsection{}
We prove by induction on $n$:\\
Take base case $n=1$. $x_{2} - x_{1} = \frac{F_{3}}{F_{2}} - \frac{F_{2}}{F_{1}} = 2 - 1 = \frac{(-1)^{1+1}}{1\cdot1}$, proving the base case.\\
Now take arbitrary $n$ and assume $x_n - x_{n-1} = \frac{(-1)^n}{F_nF_{n-1}}$. In other words, $\frac{F_{n+1}F_{n-1} - F_n^2}{F_nF_{n-1}} = \frac{(-1)^n}{F_nF_{n-1}}$, or $F_{n-1}F_{n+1} - F_n^2 = (-1)^n$. Now $x_{n+1} - x_{n} = \frac{F_{n+2}}{F_{n+1}} - \frac{F_{n+1}}{F_n} = \frac{F_{n+2}F_n - F_{n+1}^2}{F_{n}F_{n+1}}$. So it remains just to show the numerator is $(-1)^{n+1}$.\\
Note by the Fibonacci recurrence,
\begin{equation*}
    \begin{split}
        F_{n+2}F_n - F_{n+1}^2 &= F_nF_{n+1} + F_n^2 - F_{n+1}^2 \\
        &= F_nF_{n+1} + F_n^2 - F_nF_{n+1} - F_{n-1}F_{n+1} \\
        &= - (F_{n-1}F_{n+1} - F_n^2) \\
        &= (-1)^{n+1}
    \end{split}
\end{equation*}
as desired $\Box$

\subsection{}
Note for all $n$, $F_{n+2} = F_{n+1} + F_n > 2F_n$. Then
$$\frac{1}{F_n} < \frac{1}{2}\frac{1}{F_{n-2}} < \frac{1}{4}\frac{1}{F_n-4} < \dots < \frac{1}{2^{\lfloor n/2 \rfloor}}$$
$$|x_{n+1} - x_n| = \frac{1}{F_nF_{n+1}} < \frac{1}{F_nF_n} \leq \frac{1}{2^{n-1}}$$
And so for all natural $n,\,k$, by Triangle inequality,
$$|x_{n+k} - x_n| \leq \sum\limits_{i=n}^{n+k-1} |x_{i+1} - x_i| < \sum\limits_{i=n}^{n+k-1} \frac{1}{2^{i-1}} < \frac{1}{2^{n-2}}$$
Thus for every $\epsilon > 0$ we can always find $N$ s.t. $\frac{1}{2^{N-2}} < \epsilon$ and $|x_n - x_m| < \epsilon$ for all $n,\,m \geq N$. Hence, $(x_n)$ is Cauchy $\Box$

\subsection{}
$$1 + \frac{1}{x_n} = 1 + \frac{F_n}{F_{n+1}} = \frac{F_{n+1}+F_n}{F_{n+1}} = \frac{F_{n+2}}{F_n} = x_{n+1} \; \Box$$
Note that since $x_n \rightarrow x$, by a similar argument to problem (2) we can say $\lim\limits_{n\rightarrow\infty} x_{n+1} = x$. By the polynomial limit property,
$$x_{n+1} = 1 + \frac{1}{x_n}$$
$$\lim\limits_{n\rightarrow\infty} x_{n+1} = \lim\limits_{n\rightarrow\infty} (1+\frac{1}{x_n})$$
$$x = 1 + \frac{1}{x} \; \Box$$

\subsection{}
Since $(x_n)$ is Cauchy in $\mathbb{R}$, it converges to some $x$. From part (c), if $(x_n)$ converges to $x$, $x = 1 + \frac{1}{x}$. So $x_n$ converges to $x = \frac{1 \pm \sqrt{5}}{2}$. Since $x_n > 0$ for all $n$, $x>0$ and so $x_n \rightarrow \frac{1+\sqrt{5}}{2}$ $\Box$

\section{}
\subsection{}
Take some $\epsilon > 0$ and corresponding $\sqrt{\epsilon} > 0$. Note $x_n \rightarrow x$ implies $d(x_n,x) \leq p^{-\delta} < \sqrt{\epsilon}$ for $n\geq N$, for some $\delta$ and corresponding $N$. That is, $x_n-x = p^k\frac{a}{b}$ for integral $k\geq \delta$ and $p\nmid a,b$. We can write $x = p^l\frac{c}{d}$ for integral $l$, $p \nmid c,d$.\\
Suppose $k\leq l$. Then $|x_n+x| = p^k(\frac{a}{b} + 2p^{l-k}\frac{c}{d}) = p^k(\frac{ad + 2p^{l-k}bc}{bd})$. Since all of $p\nmid b,d$, the denominator has no factors of $p$. So $|x_n+x| = p^m\frac{e}{f}$ where $m\geq k$, $p \nmid e,f$. A similar argument holds if $k\geq l$, just swapping the positions of $x_n$ and $x$.\\
So $\|x_n^2 - x^2\| = \|p^k\frac{a}{b}p^m\frac{e}{f}\| = \|p^{k+m}\frac{ae}{bf}\| = p^{-(k+m)}\leq p^{-2k} < \epsilon$. Thus, $x_n^2 \rightarrow x^2$ $\Box$

\subsection{}
Note for any $k_1,k_2,k_2\geq k_1$, $\|x_{k_2} - x_{k_1}\| = \|x_{k_1} - x_{k_2}\| = \|g_{k_1+1}p^{k_1+1} + g_{k_1+2}p^{k_1+2} + \dots g_{k_2}p^{k_2}\|$. Since all $g_i<p$ and are integral, $\|x_{k_1} - x_{k_2}\| = \|p^{k_1+1}r\|$ for some $p\nmid r$. So $\|x_{k_1} - x_{k_2}\| = p^{-(k_1+1)}$. Then for every $\epsilon > 0$ we can find $p^{-(k+1)} < \epsilon$ and thus $\|x_{k_1} - x_{k_2}\| \leq p^{-(k+1)}$ for all $k_1,k_2\geq k$. Thus, $(x_k)$ is Cauchy $\Box$

\subsection{}
We prove by induction:\\
First take base case $k=2$. Then $x_2 = 1+ \frac{p+1}{2}p^2$ by definition, so $x_2^2 = 1 + (p+1)p^2 + \left(\frac{p+1}{2}\right)^2p^4$ and $x_2^2 - 1 - p^2 = p^3 + \left(\frac{p+1}{2}\right)^2p^4$, which is divisible by $p^3$.\\
Now take arbitrary $n$ and assume we have constructed suitable $g_i$ for $i<n$. Then $x_{n-1}^2 = kp^n+1+p^2$ for some $p\nmid k$, and
\begin{equation*}
    \begin{split}
        x_n^2 &= (x_{n-1} + g_np^n)^2 \\
        &= x_{n-1}^2 + 2x_{n-1}g_np^n + g_n^2p^{2n} \\
        &= kp^n + 1 + p^2 + 2x_{n-1}g_np^n + g_n^2p^{2n} \\
        &= 1 + p^2 + p^n(k + 2mg_n)
    \end{split}
\end{equation*}
where $m = 2x_n + g_np^n$. Since $p \nmid x_n^2$, $p \nmid x_n$ and so $p \nmid m$, $p\nmid 2m$. Further, $p \nmid k$. So since $p$ is prime, we are guaranteed to find some $g_n < p$ s.t. $p \nmid k+2mg_n$, since all numbers $p\nmid s$ are primitive roots mod $p$. Thus we have constructed $g_n$ s.t. $p^{n+1} \mid x_n^2 - 1 -p$. By induction, we can construct suitable $g_n$ for all $n$ $\Box$

\subsection{}
First show $p^2+1$ is the limit for $(x_k^2)$:\\
For every $\epsilon>0$ we can find $p^{-(k+1)}<\epsilon$ for some $k$. From (c), for $n\geq k$, $\|x_n^2 - (p^2+1)\| = \|tp^{n+1}\| = p^{-(n+1)} < \epsilon$, where $t$ is some integer s.t. $p\nmid t$. So $p^2+1$ is the limit for $(x_k^2)$.\\
Now suppose a limit exists for $(x_k)$. It then must be $\sqrt{p^2+1}$, since by (a) if the limit were some other value $a$, the limit for $(x_k^2) = a^2$, and limits are unique. But $\sqrt{p^2+1}$ is irrational, so $(x_k)$ has no limit in $(\mathbb{Q},d_p)$ $\Box$

\end{document}
