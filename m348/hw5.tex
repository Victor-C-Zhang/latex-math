\documentclass{article}
\usepackage[utf8]{inputenc}
\usepackage[margin=1in]{geometry}

\title{348 - Homework 5}
\author{Victor Zhang}
\date{February 24, 2021}

\usepackage[utf8]{inputenc}
\usepackage{amsmath}
\usepackage{amsfonts}
\usepackage{natbib}
\usepackage{graphicx}
% \usepackage{changepage}
\usepackage{amssymb}
\usepackage{xfrac}
% \usepackage{bm}
% \usepackage{empheq}
\usepackage{listings}
\usepackage{xcolor}
\usepackage[title]{appendix}

\newcommand{\contra}{\raisebox{\depth}{\#}}

\definecolor{codegreen}{rgb}{0,0.6,0}
\definecolor{codegray}{rgb}{0.5,0.5,0.5}
\definecolor{codepurple}{rgb}{0.58,0,0.82}
\definecolor{backcolour}{rgb}{0.95,0.95,0.92}

\lstdefinestyle{mystyle}{
    backgroundcolor=\color{backcolour},   
    commentstyle=\color{codegreen},
    keywordstyle=\color{magenta},
    numberstyle=\tiny\color{codegray},
    stringstyle=\color{codepurple},
    basicstyle=\ttfamily\footnotesize,
    breakatwhitespace=false,         
    breaklines=true,                 
    captionpos=b,                    
    keepspaces=true,                 
    numbers=left,                    
    numbersep=5pt,                  
    showspaces=false,                
    showstringspaces=false,
    showtabs=false,                  
    tabsize=2
}

\lstset{style=mystyle}

\newenvironment{myindentpar}[1]
  {\begin{list}{}
          {\setlength{\leftmargin}{#1}}
          \item[]
  }
  {\end{list}}

\pagestyle{empty}

\begin{document}

\maketitle
% \begin{center}
% {\huge Econ 482 \hspace{0.5cm} HW 3}\
% {\Large \textbf{Victor Zhang}}\
% {\Large February 18, 2020}
% \end{center}

\section*{Hill cipher}
First write every encryption in $\mathbb{Z}_{26}^2$:\\
\begin{gather}
\begin{bmatrix}6 \\ 0\end{bmatrix} \mapsto \begin{bmatrix}24 \\ 9\end{bmatrix}\\
\begin{bmatrix}7 \\ 5\end{bmatrix} \mapsto \begin{bmatrix}1 \\ 2\end{bmatrix}\\
\begin{bmatrix}1 \\ 2\end{bmatrix} \mapsto \begin{bmatrix}9 \\ 1\end{bmatrix}
\end{gather}
Subtracting (1) from (2) and (2) from (3) yields
\begin{gather*}
k_1\begin{bmatrix}1 \\ 5\end{bmatrix} = \begin{bmatrix}3 \\ 19\end{bmatrix}\\
k_1\begin{bmatrix}20 \\ 23\end{bmatrix} = \begin{bmatrix}8 \\ 25\end{bmatrix}
\end{gather*}
Then $k_1 = \left(\begin{smallmatrix}3 & 0 \\ 0 & 9\end{smallmatrix}\right)$. Substituting this into (1) yields $k_2 = \left(\begin{smallmatrix}6 \\ 9\end{smallmatrix}\right)$ $\Box$

\section*{2.1}
The idea that backdoors are anything similar to wiretaps is dangerous and should be seen as a clear thrust by the government to destroy the very concept of personal privacy. The main difference between wiretaps and backdoors is the amount of information that can be gained by breaching security. A wiretap is good only for the time that it's active. There is no way to know anything about the person's prior or future conversations given only the wiretap. However, knowing a person's private key would allow the government to see everything the provider or host knows about the person. This is a far greater breach of privacy than any wiretap could be. In addition, a wiretap cannot be passed around to people across the world to listen in on. If a federal agent wants to share a wiretap, they can't. Unless you have a physical connection to the datastream, you cannot listen in. If a federal agent wanted to share everything private about a person into with anyone, it suffices to simply share the private key on some public forum and now the entire Internet has access to that person's private information into perpetuity.

\section*{2.3.a}
WLOG, let $a < b$. Then $g^{b-a} = 1$. Since $g$ is a primitive root, the only power $y < p$ s.t. $g^y = 1$ is $p-1$. Then $b - a \equiv 0 \mod p-1$ and the result follows $\Box$\\
This implies $\log_g$ is a well-defined map, since solutions to $g^x = h$ (and thus $\log_g(h)$) are unique in $\mathbb{Z}_{p-1}$.

\section*{2.3.b}
Put $x_1 = \log_g(h_1)$, $x_2 = \log_g(h_2)$. Then $g^{x_1+x_2} = h_1h_2$, so $\log_g(h_1h_2) = x_1 + x_2 = \log_g(h_1) + \log_g(h_2)$ $\Box$

\section*{2.3.c}
Put $x = \log_g(h)$. Then $g^{nx} = (g^x)^n = h^n$. So $\log_g(h^n) = nx$ $\Box$

\section*{2.4.c}
From table 2.1 we see $627^{18} \equiv 608 \mod 941$. So $\log_{627}608 = 18$ $\Box$
\section*{2.5}
The reverse direction is clear. If $\log_g(a) = 2k$, $g^k$ is a square root mod $p$. If $a$ has a square root, we may write $a = c^2$. Then by 2.3.c $\log_g(a) = \log_g(c^2) = 2\log_g(c)$ is even $\Box$

\section*{2.6}
Bob should send $g^b \mod p = 2^{871} \mod 1373 = 805$. Their shared secret value is $(g^a)^b = A^b = 974^{871} \mod 1373 = 397$. Alice's secret exponent is $\log_2(974) = 587$ by Shank's $\Box$

\section*{2.8.a}
Alice's public key is $g^a = 2^{947} = 117 \mod 1373$.
\section*{2.8.b}
Alice sends $(g^k, mB^k) = (2^{877}, 583 \cdot 469^{877}) = (719, 623)$.
\section*{2.8.c}
$m = c_1^{-a}c_2 = 661^{-299} \cdot 1325 = 113 \cdot 1325 = 68$.
\section*{2.8.d}
Using Shank's, we get $\log_2(893) = 219$. $m = c_1^{-b}c_2 = 693^{-219} \cdot 793 = 532 \cdot 793 = 365$.

\section*{2.17}
Using the following implementation of Shank's, we calculate $\log_{11}(21) \mod 71 = 37$, $\log_{156}(116) \mod 593 = 59$, $\log_{650}(2213) \mod 3571 = 319$.
\lstinputlisting[language=C++]{shanks.cpp}

\end{document}

% List of tex snippets:
%   - tex-header (this)
%   - R      --> \mathbb{R}
%   - Z      --> \mathbb{Z}
%   - B      --> \mathcal{B}
%   - E      --> \mathcal{E}
%   - M      --> \mathcal{M}
%   - m      --> \mathfrak{m}({#1})
%   - normlp --> \norm{{#1}}_{L^{{#2}}}
