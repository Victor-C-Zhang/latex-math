\documentclass{article}
\usepackage[utf8]{inputenc}
\usepackage[margin=1in]{geometry}

\title{348 - Homework 11}
\author{Victor Zhang}
\date{April 22, 2021}

\usepackage[utf8]{inputenc}
\usepackage{amsmath}
\usepackage{amsfonts}
\usepackage{natbib}
\usepackage{graphicx}
% \usepackage{changepage}
\usepackage{amssymb}
\usepackage{xfrac}
% \usepackage{bm}
% \usepackage{empheq}
\usepackage{listings}
\usepackage{xcolor}
\usepackage[title]{appendix}
\usepackage{dirtytalk}

\newcommand{\contra}{\raisebox{\depth}{\#}}

\definecolor{codegreen}{rgb}{0,0.6,0}
\definecolor{codegray}{rgb}{0.5,0.5,0.5}
\definecolor{codepurple}{rgb}{0.58,0,0.82}
\definecolor{backcolour}{rgb}{0.95,0.95,0.92}

\lstdefinestyle{mystyle}{
    backgroundcolor=\color{backcolour},   
    commentstyle=\color{codegreen},
    keywordstyle=\color{magenta},
    numberstyle=\tiny\color{codegray},
    stringstyle=\color{codepurple},
    basicstyle=\ttfamily\footnotesize,
    breakatwhitespace=false,         
    breaklines=true,                 
    captionpos=b,                    
    keepspaces=true,                 
    numbers=left,                    
    numbersep=5pt,                  
    showspaces=false,                
    showstringspaces=false,
    showtabs=false,                  
    tabsize=2
}

\lstset{style=mystyle}

\newenvironment{myindentpar}[1]
  {\begin{list}{}
          {\setlength{\leftmargin}{#1}}
          \item[]
  }
  {\end{list}}

\pagestyle{empty}

\begin{document}

\maketitle
% \begin{center}
% {\huge Econ 482 \hspace{0.5cm} HW 3}\
% {\Large \textbf{Victor Zhang}}\
% {\Large February 18, 2020}
% \end{center}

\section*{5.36.a}
The problem is solved for 23 and 40 people in the text. The probabilities are 0.507 and 0.891, respectively. The event at least one person in a group of 200 has your birthday is the complement of the event none of them have your birthday. The probability is thus
$$p = 1 - \left(\frac{364}{365}\right)^{200} \approx 0.422 \; \Box$$

\section*{5.36.b}
The event at least two people share a birthday is the complement of the event nobody has the same birthday. For $n \leq N$ this probability is
$$p = 1 - \prod_{i=1}^{n-1} \frac{N-i}{N}$$

\section*{5.37.a}
It suffices to count the number of choices of 8 cards and the number of 8 cards if one of them is the king of hearts.
$$p = \frac{\binom{51}{7}}{\binom{52}{8}} \approx 0.154 \; \Box$$

\section*{5.37.b}
This is analagous to marking 8 cards and calculating the probability we randomly choose at least one of them. This event is the complement of picking 8 cards and having none of them be marked.
$$p = 1 - \frac{44}{52}\frac{43}{51}\frac{42}{50}\frac{41}{49}\frac{40}{48}\frac{39}{47}\frac{38}{46}\frac{37}{45} \approx 0.764 \; \Box$$

\section*{5.39}
From the table we see $g^{234} = h \cdot g^{399} = 304$. Then $h = g^{234-399} = g^{-165} = g^{645}$, where the last equality is a result of Fermat little. So $x = 645$ $\Box$

\section*{5.40}
We write $(\delta_i-\beta_i)\log_g h = \gamma_i - \alpha_i$. $\gcd(\delta_i-\beta_i, p-1) = 6$, so we find $s$ by extended Euclid to be $81340$. Multiplying the equation by $s$ yields
$$6 \log_g h = 61950$$
Now we know $\log_g h \in \{\frac{w}{d} + k\frac{p-1}{d}\} = \{10325 + 13633k\}$. By trial and error, we find $\log_g h = 10325 + 13633 \cdot 4 = 64857$ $\Box$

\section*{5.41}
We find a match at $i = 93$. $x_i = y_i = 1217$, $\alpha_i = 6741$, $\beta_i = 7648$, $\gamma_i = 4647$, $\delta_i = 7656$. $\gcd(\delta_i-\beta_i, p-1) = 2$. We find $s = 6967$. Then we get the equation
$$2 \log_g h = 1257$$
and $\log_g h \in \{1257 + 3981 \cdot k\}$. Testing both gives $\log_g h = 5238$ $\Box$

\section*{5.44.a}
This problem essentially reduces to asking how fast we can find two numbers $x_i,x_j$ that are congruent mod $p$. We can formulate this in terms of the birthday problem. We pick random numbers from $[0,p)$ and hope for a match. From an analysis of the birthday problem, it suffices to pick $\mathcal{O}(\sqrt{p})$ numbers to have a match w.h.p. And since the $\rho$ method will loop on indices $[i,j]$, it takes $\mathcal{O}(\sqrt{p})$ time for the \say{tortoise and hare} method to realize this $\Box$

\section*{5.44.b}
Using the code in the appendix, we find
\begin{gather*}
k = 4, k/\sqrt{N} = 0.085\\
k = 34, k/\sqrt{N} = 0.011\\
k = 126, k/\sqrt{N} = 0.003
\end{gather*}

\section*{5.44.c}
We get the values
\begin{gather*}
k = 6, k/\sqrt{N} = 0.128\\
k = 6, k/\sqrt{N} = 0.002\\
k = 68, k/\sqrt{N} = 0.001
\end{gather*}
Runtimes are much better for the larger values of $N$.

\section*{5.44.d}
If $N$ is prime, we will find $p = N$ in $\mathcal{O}(\sqrt{N})$. The problem collapses into the birthday problem for $N$ $\Box$

\section*{5.44.e}
This function is not random enough, since it will loop with length $\mathrm{ord}(x_0)$. There is no guarantee that we will find a collision, since the order may be an arbitrary divisor of $N-1$ $\Box$

\section*{5.44.f}
This function is not random for $x_0 = 2$. It has the property $f^2(0) = f^2(2) = 2$. Thus, after 2 iterations, we always find $x_2 = y_2 = 2$ $\Box$ 

\newpage
\begin{appendices}
\section{Pollard's $\rho$ Method}
\lstinputlisting[language=C++]{pollard_rho.cpp}
\end{appendices}

\end{document}

% List of tex snippets:
%   - tex-header (this)
%   - R      --> \mathbb{R}
%   - Z      --> \mathbb{Z}
%   - B      --> \mathcal{B}
%   - E      --> \mathcal{E}
%   - M      --> \mathcal{M}
%   - m      --> \mathfrak{m}({#1})
%   - normlp --> \norm{{#1}}_{L^{{#2}}}
