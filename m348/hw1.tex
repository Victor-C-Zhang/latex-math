\documentclass{article}
\usepackage[utf8]{inputenc}
\usepackage[margin=1in]{geometry}

\title{348 - Homework 1}
\author{Victor Zhang}
\date{January 28, 2020}

\usepackage[utf8]{inputenc}
\usepackage{amsmath}
\usepackage{amsfonts}
\usepackage{natbib}
\usepackage{graphicx}
% \usepackage{changepage}
\usepackage{amssymb}
\usepackage{xfrac}
% \usepackage{bm}
% \usepackage{empheq}
\usepackage{array}

\newcommand{\contra}{\raisebox{\depth}{\#}}

\newenvironment{myindentpar}[1]
  {\begin{list}{}
          {\setlength{\leftmargin}{#1}}
          \item[]
  }
  {\end{list}}

\pagestyle{empty}

\begin{document}

\maketitle
% \begin{center}
% {\huge Econ 482 \hspace{0.5cm} HW 3}\
% {\Large \textbf{Victor Zhang}}\
% {\Large February 18, 2020}
% \end{center}

\section*{Caesar Cipher}
Plaintext name: \verb|VICTORZHANG|\\
Encrypted name: \verb|YLFWRUCKDQJ|\\
Encryption of \verb|CRYPTOGRAPHY| with shift $k = 21$: \verb|XMTKOJBMVKCT|

\section*{1.1.a}
\verb|LALRPZQSTDEZCJTDHZCESLGZWFXPZQWZRTN|
\section*{1.1.b}
\verb|There are no secrets better than the secrets that everybody guesses|
\section*{1.1.c}
\verb|When angry count ten before you speak, if very angry an hundred|

\section*{1.3.a}
\verb|IBX FEPA QL BQAAXW QW IBX FSVAXW|
\section*{1.3.b}
\begin{tabular}{|c|c|c|c|c|c|c|c|c|c|c|c|c|c|c|c|c|c|c|c|c|c|c|c|c|c|}
\hline
\verb|A| & \verb|B| & \verb|C| & \verb|D| & \verb|E| & \verb|F| & \verb|G| & \verb|H| & \verb|I| & \verb|J| & \verb|K| & \verb|L| & \verb|M| & \verb|N| & \verb|O| & \verb|P| & \verb|Q| & \verb|R| & \verb|S| & \verb|T| & \verb|U| & \verb|V| & \verb|W| & \verb|X| & \verb|Y| & \verb|Z|\\
\hline
\verb|d| & \verb|h| & \verb|b| & \verb|w| & \verb|o| & \verb|g| & \verb|u| & \verb|q| & \verb|t| & \verb|c| & \verb|j| & \verb|s| & \verb|y| & \verb|x| & \verb|z| & \verb|l| & \verb|i| & \verb|m| & \verb|a| & \verb|k| & \verb|f| & \verb|r| & \verb|n| & \verb|e| & \verb|v| & \verb|p|\\
\hline
\end{tabular}
\section*{1.3.c}
\verb|The secret password is swordfish|

\section*{1.4.a}
From frequency analysis, we see the bigrams \verb|JN| and \verb|NR| most likely correspond with \verb|th| and \verb|he|. Using this, we may deduce the first five characters are \verb|these| and \verb|I| most likely corresponds to \verb|a|. From there, trial and error yields the passage
\begin{myindentpar}{1em}
\verb|These characters, as one might readily guess, form a cipher.|\\
\verb|That is to say they convey a meaning. But then, from what is known of Captain Kidd,|\\
\verb|I could not suppose him capable of constructing any of the more abstruse cryptographs.|\\
\verb|I made up in my mind at once that this was of a simple species,|\\
\verb|such however as would appear to the crude intellect of the sailor absolutely|\\
\verb|insoluble without the key.|
\end{myindentpar}

\section*{1.4.b}
From frequency analysis, we see \verb|B| is \verb|e|, \verb|R| is \verb|a|, and \verb|G| is \verb|t|. From bigram analysis, we may deduce \verb|RI| corresponds with \verb|an|. Trial and error yields
\begin{myindentpar}{1em}
\verb|I was, I think, well educated for the standard of the day.|\\
\verb|My sister and I had a German governess, a very sentimental creature.|\\
\verb|She taught us the language of flowers, a forgotten study nowadays but most charming.|\\
\verb|A yellow tulip, for instance, means "hopeless love" while a China aster means|\\
\verb|"I die of jealousy at your feet."|
\end{myindentpar}

\section*{1.6.a}
By definition, we may write $b = qa$ and $c = tb$ for some integral $q,t$. Then $c = tqa$ and thus $a|c$ $\Box$
\section*{1.6.b}
We may write $b = qa$ and $a = tb$. But then $a = tqa$, so both $q,t$ have magnitude 1. This is only possible if $q,t = \pm 1$, or in other words, $a = \pm b$ $\Box$
\section*{1.6.c}
Suppose $b = qa$, $c = ta$. Then $b+c = (q+t)a$, $b-c = (q-t)a$. We note $q+t$ and $q-t$ are both integral, from which the result follows $\Box$

\end{document}

% List of tex snippets:
%   - tex-header (this)
%   - R      --> \mathbb{R}
%   - Z      --> \mathbb{Z}
%   - B      --> \mathcal{B}
%   - E      --> \mathcal{E}
%   - M      --> \mathcal{M}
%   - m      --> \mathfrak{m}({#1})
%   - normlp --> \norm{{#1}}_{L^{{#2}}}
