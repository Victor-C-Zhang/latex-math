\documentclass{article}
\usepackage[utf8]{inputenc}
\usepackage[margin=1in]{geometry}

\title{348 - Homework 4}
\author{Victor Zhang}
\date{February 18, 2021}

\usepackage[utf8]{inputenc}
\usepackage{amsmath}
\usepackage{amsfonts}
\usepackage{natbib}
\usepackage{graphicx}
% \usepackage{changepage}
\usepackage{amssymb}
\usepackage{xfrac}
% \usepackage{bm}
% \usepackage{empheq}

\newcommand{\contra}{\raisebox{\depth}{\#}}

\newenvironment{myindentpar}[1]
  {\begin{list}{}
          {\setlength{\leftmargin}{#1}}
          \item[]
  }
  {\end{list}}

\pagestyle{empty}

\begin{document}

\maketitle
% \begin{center}
% {\huge Econ 482 \hspace{0.5cm} HW 3}\
% {\Large \textbf{Victor Zhang}}\
% {\Large February 18, 2020}
% \end{center}


\section*{1.32.a}
\begin{equation*}
\begin{split}
47 = 4*11 + 3,\; & 1 = (-1)*11 + 4*(47 - 4*11) = 4*47 + (-17)*11\\
11 = 3*3 + 2,\; & 1 = 3 - (11- 3*3) = (-1)*11 +4*3\\
3 = 1*2 + 1,\; & 1 = 3 - 1*2\\
\end{split}
\end{equation*}
So $47^{-1} = 4$ $\Box$

$47^{11-2} = 47^{1001_2} = 5*3 = 4 \Box$

\section*{1.32.b}
\begin{equation*}
\begin{split}
587 = 345 + 242,\; & 1 = 47*345 + (-67)*(587 - 345) = (-67)*587 + 114*345\\
345 = 242 + 103,\; & 1 = (-20)*242 + 47*(345-242) = 47*345 + (-67)*242\\
242 = 2*103 + 36,\; & 1 = 7*103 + (-20)*(242-2*103) = (-20)*242 + 47*103\\
103 = 2*36 + 31,\; & 1 = (-6)*36 + 7*(103 - 2*36) = 7*103 + (-20)*36\\
36 = 31 + 5,\; & 1 = 31 + (-6)*(36 - 31) = (-6)*36 + 7*31\\
31 = 6*5 + 1,\; & 1 = 31 + (-6)*5\\
\end{split}
\end{equation*}
So $587^{-1} = -67 = 278$ $\Box$

345 is not prime (it is clearly a multiple of 3), so Fermat little will not give us anything $\Box$

\section*{1.33.a}
By Fermat little, $a^{(p-1)} = 1$. Then if $b = 1$, $b^q = 1$, so this is a possible solution. By prop. 1.29, the order of $b$ must be divisible by $q$. That is, it must be either $1$ or $q$. Order 1 is impossible since $b \neq 1$, so it must have order $q$ $\Box$
\section*{1.33.b}
$|\mathbb{F}_p^*| = p-1$. We note there are $\varphi(p-2)$ primitive roots, thus $\varphi(p-2)$ values $a$ s.t. $a^{(p-1)/q} \neq 1$. The desired probability is this $\frac{\varphi(p-2)}{p-1}$ $\Box$

\section*{1.34.a}
$2^3 \equiv 1 \mod 7$, $2^{11} \equiv 1 \mod 23$, so $2$ is not a primitive root of (i) or (iv). It can be verified by hand that 2 is a primitive root of the two other options.
\section*{1.34.c}
A bruteforce calculation shows that $5$ is a primitive root of $23$ and $3$ is a primitive root of $29$.
\section*{1.34.d}
The primitive roots of 11 are 2,6,7,8. $\varphi(10) = 4$, as predicted.

\section*{1.35}
$(p-1)$ has prime factors $2,q$. Thus, there must not exist $b \neq 2,q$ s.t. $g^b = 1$. So it suffices to check $g^2 \neq 1$. This is guaranteed by the initial conditions $g \neq \pm 1$. So in fact, $g^b \neq 1$ for all $b \neq p-1$ and we are done $\Box$

\section*{1.36.b}
By brute force computation, $3^2 = 4^2 = 2$, so the square roots of $2 \mod 7$ are 3,4.\\
Similarly, $4^2 = 7^2 = 5$, so the square roots of $2 \mod 7$ are 4,7.\\
The quadratic residues of 11 are $\{1,3,4,5,9\}$, so $7$ has no square roots.

\section*{1.43.a}
$E(204) = 34*204 + 71 \mod 541 = 515$. $D(431) = 34^{-1}(431-71) \mod 541 = 366(431-71) \mod 541 = 297$.
\section*{1.43.b}
We may use a plaintext attack with messages $m_1,m_2$ to determine the key. WLOG suppose $m_1 < m_2$. $E(m_2) - E(m_1) = k_1(m_2-m_1)$, and given $k_1$ we can calculate $k_2 = E(m_1) - k_1m_1$ $\Box$
\section*{1.43.c}
Using our formulas, $57 = 104k_1$, $k_1 = 57*104^{-1} = 41$. Then $k_2 = 324 - 41*387 = 83$ $\Box$
\section*{1.43.d}
We need at least 3 messages $m_1,m_2,m_3$. WLOG suppose $m_1<m_2<m_3$. Then we may calculate $a = k_1(m_2-m_1), b = k_1(m_3-m_2), c = k_1(m_3-m_1)$.\\
If $a < b < c$, we may determine $k_1$ and from that, $k_2$ and $p$ by comparing the values of $E(m_i) - k_1m_i$. We may determine $k_2$ directly, and a multiple of $p$. We may refine our guess for $p$ by seeing more messages and more values $E(m_i) - k_1m_i$.\\
Otherwise, if monotonicity of $a,b,c$ is not true, we may determine $p$ by computing $a + b = p + c$. Now that we have $p$, we may proceed as in 1.43.b.

\section*{1.44.a}
$E\left(\begin{smallmatrix}2\\1\end{smallmatrix}\right) =
\left(\begin{smallmatrix}1 & 3\\ 2 & 2\end{smallmatrix}\right)
\left(\begin{smallmatrix}2\\1\end{smallmatrix}\right) + 
\left(\begin{smallmatrix}5\\4\end{smallmatrix}\right) =
\left(\begin{smallmatrix}3\\3\end{smallmatrix}\right)$.\\
$k_1^{-1}$ is the matrix s.t. $k_1k^{-1} =
\left(\begin{smallmatrix}1 & 0\\ 0 & 1\\\end{smallmatrix}\right)$.
In this case, $k_1^{-1} = \left(\begin{smallmatrix}3 & 6\\ 4 & 5\\\end{smallmatrix}\right)$.\\
$E(\left(\begin{smallmatrix}3\\5\end{smallmatrix}\right) = 
\left(\begin{smallmatrix}3 & 6\\4 & 5\\\end{smallmatrix}\right)(
\left(\begin{smallmatrix}3\\5\\\end{smallmatrix}\right) - 
\left(\begin{smallmatrix}5\\4\\\end{smallmatrix}\right)) =
\left(\begin{smallmatrix}0\\4\end{smallmatrix}\right)$.
\section*{1.44.b}
Matrix inversion is a solved problem, so the Hill cipher is analagous to the affine cipher in all ways, including how to determine the key, given $p$.
\section*{1.44.c}
Subtracting pairs 1,2 and 2,3 we get $k_1\left(\begin{smallmatrix}3\\6\end{smallmatrix}\right) =
\left(\begin{smallmatrix}7\\8\end{smallmatrix}\right)$,
$k_1\left(\begin{smallmatrix}10\\2\end{smallmatrix}\right) =
\left(\begin{smallmatrix}0\\2\end{smallmatrix}\right)$. From algebra, we deduce $k_1 = \left(\begin{smallmatrix}3 & 7\\4 & 3\end{smallmatrix}\right)$. Plugging into $\left(\begin{smallmatrix}1\\8\end{smallmatrix}\right) = k_1\left(\begin{smallmatrix}5\\4\end{smallmatrix}\right) + k_2$ gives $k_2 = \left(\begin{smallmatrix}2\\9\end{smallmatrix}\right)$.
\section*{1.44.d}
There is a group isomorphism between permutations $\sigma$ of an alphabet of size $s$ and $M_s(\mathbb{Z}_s)$. Thus, every one-step substitution cipher may be encoded as $E(m) = km$ for some $k \in M_s(\mathbb{Z}_s)$ $\Box$

\section*{1.46.b}
We may use the same algorithm as for fast exponentiation to determine the binary representation. Denote $(b,x)$ as the current state of the number $b$ and the running binary representation $x$.
\begin{equation*}
\begin{split}
(37853,0) &\to (18926,1)\\
&\to (9463, 01)\\
&\to (4731,101)\\
&\to (2365,1101)\\
&\to (1182,11101)\\
&\to (591,011101)\\
&\to (295,1011101)\\
&\to (147,11011101)\\
&\to (73,111011101)\\
&\to (36,1111011101)\\
&\to (18,01111011101)\\
&\to (9,001111011101)\\
&\to (4,1001111011101)\\
&\to (2,01001111011101)\\
&\to (1,001001111011101)\\
&\to (0,1001001111011101) \; \Box
\end{split}
\end{equation*}

\section*{1.50}
$\gcd(12849217045006222,6485880443666222) = 174385766$. 174385766 has 2 factors: $2, 87192883$. The large prime is thus 87192883 $\Box$

\end{document}

% List of tex snippets:
%   - tex-header (this)
%   - R      --> \mathbb{R}
%   - Z      --> \mathbb{Z}
%   - B      --> \mathcal{B}
%   - E      --> \mathcal{E}
%   - M      --> \mathcal{M}
%   - m      --> \mathfrak{m}({#1})
%   - normlp --> \norm{{#1}}_{L^{{#2}}}
