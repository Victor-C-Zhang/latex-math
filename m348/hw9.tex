\documentclass{article}
\usepackage[utf8]{inputenc}
\usepackage[margin=1in]{geometry}

\title{348 - Homework 9}
\author{Victor Zhang}
\date{April 1, 2021}

\usepackage[utf8]{inputenc}
\usepackage{amsmath}
\usepackage{amsfonts}
\usepackage{natbib}
\usepackage{graphicx}
% \usepackage{changepage}
\usepackage{amssymb}
\usepackage{xfrac}
% \usepackage{bm}
% \usepackage{empheq}

\newcommand{\contra}{\raisebox{\depth}{\#}}

\newenvironment{myindentpar}[1]
  {\begin{list}{}
          {\setlength{\leftmargin}{#1}}
          \item[]
  }
  {\end{list}}

\pagestyle{empty}

\begin{document}

\maketitle
% \begin{center}
% {\huge Econ 482 \hspace{0.5cm} HW 3}\
% {\Large \textbf{Victor Zhang}}\
% {\Large February 18, 2020}
% \end{center}

\section*{3.36.a}
Through brute calculation, we write
$$g^{3030} = 14580 = 2^2 \cdot 3^6 \cdot 5$$
$$g^{6892} = 18432 =  2^{11} \cdot 3^2$$
$$g^{18312} = 6000 = 2^4 \cdot 3 \cdot 5^3$$
All of these powers are evidently 5-smooth $\Box$

\section*{3.36.b}
Let us write $x = \log 2, y = \log 3, z = \log 5$.
\begin{equation*}
\begin{split}
3030 &= 2x + 6y + z\\
6892 &= 11x + 2y\\
18312 &= 4x + y + 3z
\end{split}
\end{equation*}
Solving mod 2, mod 9539 gives
$$x \equiv 0 \!\!\!\mod 2, x \equiv 8195 \!\!\!\mod 9539$$
$$y \equiv 0 \!\!\!\mod 2, y \equiv 1299 \!\!\!\mod 9539$$
$$z \equiv 0 \!\!\!\mod 2, z \equiv 7463 \!\!\!\mod 9539$$
Computation with CRT yields $\log 2 = x = 17734, \log 3 = y = 10838, \log 5 = z = 17002$.

\section*{3.36.c}
We compute $19 \cdot 17^{-12400} = 384 = 2^7 \cdot 3$, which is 5-smooth.

\section*{3.36.d}
$$\log (19 \cdot 17^{-12400}) = \log 19 - 12400 = 7 \log 2 + \log 3$$
$$\log 19 = 147376 \equiv 13830 \!\!\!\mod p-1\; \Box$$

\section*{4.2}
We verify $S^e = D$ for each signature and document pair:\\
$$S^e = 772481 \neq D$$
$$(S')^e = 161153 = D'$$
$$(S'')^e = 586036 = D''$$

\section*{4.4}
The encrypted message $c$ is decrypted exactly as in standard RSA to yield $m$. Since $x^{d_Ae_A} \equiv x \mod N_A$ for all $x$, we can be certain $s^{e_A} = ((\mathrm{Hash}(m))^{d_A})^{e_A} = \mathrm{Hash}(m)$. Since Bob has $m$, he can easily calculate the hash and verify authenticity. However, if someone were to forge a message they would likely need to know Alice's private key or have a fast method to solve DLP in order for $s^{e_A}$ to match $\mathrm{Hash}(m)$. Both are unlikely.

\section*{4.11}
Using Shanks, we calculate $\log_{21947}(31377) = 602 \!\!\!\mod 103687$. Write $S_1 = (g^k \!\!\!\mod p)\!\!\!\mod q = 439$, $S_2 = (D+aS_1)k^{-1} \!\!\!\mod q = 1259$. The signature is thus $(439,1259)$.

\end{document}

% List of tex snippets:
%   - tex-header (this)
%   - R      --> \mathbb{R}
%   - Z      --> \mathbb{Z}
%   - B      --> \mathcal{B}
%   - E      --> \mathcal{E}
%   - M      --> \mathcal{M}
%   - m      --> \mathfrak{m}({#1})
%   - normlp --> \norm{{#1}}_{L^{{#2}}}
