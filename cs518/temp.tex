\documentclass{article}
\usepackage[utf8]{inputenc}
\usepackage[margin=1in]{geometry}

\title{Quasi Filters}
\author{Victor Zhang}
\date{February 22, 2021}

\usepackage[utf8]{inputenc}
\usepackage{amsmath}
\usepackage{amsfonts}
\usepackage{natbib}
\usepackage{graphicx}
% \usepackage{changepage}
\usepackage{amssymb}
\usepackage{xfrac}
% \usepackage{bm}
% \usepackage{empheq}
\usepackage{dirtytalk}

\newtheorem{definition}{Definition}
\newtheorem{lemma}{Lemma}


\newcommand{\contra}{\raisebox{\depth}{\#}}

\newenvironment{myindentpar}[1]
  {\begin{list}{}
          {\setlength{\leftmargin}{#1}}
          \item[]
  }
  {\end{list}}

\pagestyle{empty}

\begin{document}

\maketitle
% \begin{center}
% {\huge Econ 482 \hspace{0.5cm} HW 3}\
% {\Large \textbf{Victor Zhang}}\
% {\Large February 18, 2020}
% \end{center}

\section{Randomized to Deterministic}

\begin{definition}
A \textit{quasi-filter} $Q(\epsilon,\rho)$ is a filter with false positive rate $\epsilon$ and false negative rate $\rho$.
\end{definition}
Consider a string of $n$ insertions $S \in U^n$ to $Q$. WLOG we may suppose the false negative set $T \subset S$ is completely and uniformly shuffled after each insertion.
\begin{definition}
Let $r \in \{0,1\}^*$ be an infinite random string and $Q_r$ the resulting quasi-filter taking $r$ to be its internal randomness. For sequence $S \in U^n$ of $n$ insertions, denote by $\hat{S} \subseteq U$ the set of all elements for which $Q_r$ responds \say{yes} to.
\end{definition}
Crucially, $S \subseteq \hat{S}$, regardless of false negatives, for all random strings $r$. Define
$$S_{r,\epsilon} = \{S \in U^n \;:\; \mu(\hat{S}_r) \leq 4\epsilon \}$$

\begin{lemma}
(From Pagh, Segev, Weider) There exists $r^* \in \{0,1\}^*$ s.t. $|S_{r^*,\epsilon}| \geqslant u^n/2$.
\end{lemma}
The proof is exactly the same as the proof offered in the original paper.

\section{Compression Argument}
From now on, focus on $Q_{r^*,\epsilon}$. Let $S$ be a sequence taken from the set in lemma 1. Partition it into $C_1C_2\dots$ s.t. $C_i$ has length $\gamma^i$ insertions. Define $S_i = C_1c_2\dots C_i$. Note since $\hat{S}$ is cumulative, $\hat{S}_i \subseteq \hat{S}_{i+1}$ and thus $\mu(\hat{S}_i) \leqslant \mu(\hat{S}_{i+1}) \leqslant 4\epsilon$.

\begin{lemma}
(From Pagh, Segev, Weider) For any suitable sequence $S$, there exists integer $i$ s.t. $|S_i| \in [\alpha n,n]$ and
$$\mu(\hat{S}_i) - \mu(\hat{S}_{i-1}) \leqslant \frac{4\epsilon}{\log_\gamma(1/\alpha)-2}$$
\end{lemma}
The proof may also be taken exactly from the original paper.

Now we try to encode according to the paper. To be able to successfully make the same encoding, we must be able to characterize $\hat{S}_i$ given $Q_i$, which is only accomplishable if false positives are persistent. That is, once an element is a false positive, it stays a false positive forever.

\section*{If the Yes set may shrink}
Let us imagine everything besides the actual compression magically works, which it should. Then the compression scheme also works, since $Q_i$ can characterize $\hat{S}_i$ if we simply query everything in the universe to $Q_i$; $S_{i-1}$ and $Q_i$ characterizes $\hat{S}_{i-1}$, since we may recreate the data structure using the sequence $S_{i-1}$ and query everything in the universe to this new structure.

\end{document}

% List of tex snippets:
%   - tex-header (this)
%   - R      --> \mathbb{R}
%   - Z      --> \mathbb{Z}
%   - B      --> \mathcal{B}
%   - E      --> \mathcal{E}
%   - M      --> \mathcal{M}
%   - m      --> \mathfrak{m}({#1})
%   - normlp --> \norm{{#1}}_{L^{{#2}}}
