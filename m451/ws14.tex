\documentclass{article}
\usepackage[utf8]{inputenc}

\title{451 - Worksheet 14}
\author{Victor Zhang}
\date{October 30, 2020}

\usepackage[utf8]{inputenc}
\usepackage{amsmath}
\usepackage{amsfonts}
\usepackage{natbib}
\usepackage{graphicx}
% \usepackage{changepage}
\usepackage{amssymb}
\usepackage{xfrac}
% \usepackage{bm}
% \usepackage{empheq}

\newcommand{\contra}{\raisebox{\depth}{\#}}

\newenvironment{myindentpar}[1]
  {\begin{list}{}
          {\setlength{\leftmargin}{#1}}
          \item[]
  }
  {\end{list}}

\pagestyle{empty}

\begin{document}

\maketitle
% \begin{center}
% {\huge Econ 482 \hspace{0.5cm} HW 3}\
% {\Large \textbf{Victor Zhang}}\
% {\Large February 18, 2020}
% \end{center}

\section{}
\subsection{}
For all $n \in N$, $hnh^{-1} = n'$ for some $n' \in N$. Then $hn = n'h$ so $HN \subseteq NH$. Conversely, $hnh^{-1} = n'$ so $nh^{-1} = h^{-1}n'$. Since $H$ is a group, $h^{-1} \in H$ and we may conclude $NH \subseteq HN$. Then in fact $NH = HN$ $\Box$
\subsection{}
Since $H,N$ are both groups, $NH$ contains $ee = e$, the identity. For arbitrary $n_1h_1, n_2h_2 \in NH$, $(n_1h_1)(n_2h_2) = n_1h_1n_2h_2 = n_1h_1n_2h_1^{-1}h_1h_2 = n_1n_3h_1h_2 = n'h'$, where the second equality comes from normality of $N$. Then $NH$ is closed under multiplication. $(nh)^{-1} = h^{-1}n^{-1} = h^{-1}n^{-1}hh^{-1} = n'h^{-1} \in NH$, so $NH$ is closed in inverses as well. Thus $NH$ is a group, and in particular $NH < G$ $\Box$

\end{document}

% List of tex snippets:
%   - tex-header (this)
%   - R      --> \mathbb{R}
%   - Z      --> \mathbb{Z}
%   - B      --> \mathcal{B}
%   - E      --> \mathcal{E}
%   - M      --> \mathcal{M}
%   - m      --> \mathfrak{m}({#1})
%   - normlp --> \norm{{#1}}_{L^{{#2}}}
