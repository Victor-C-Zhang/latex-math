\documentclass{article}
\usepackage[utf8]{inputenc}

\title{451 - Worksheet 2}
% \author{Victor Zhang}
\date{September 3, 2020}

\usepackage[utf8]{inputenc}
\usepackage{amsmath}
\usepackage{amsfonts}
\usepackage{natbib}
\usepackage{graphicx}
% \usepackage{changepage}
\usepackage{amssymb}
% \usepackage{bm}
% \usepackage{empheq}

\newcommand{\contra}{\raisebox{\depth}{\#}}

\newenvironment{myindentpar}[1]
  {\begin{list}{}
          {\setlength{\leftmargin}{#1}}
          \item[]
  }
  {\end{list}}

\pagestyle{empty}

\begin{document}

\maketitle
% \begin{center}
% {\huge Econ 482 \hspace{0.5cm} HW 3}\
% {\Large \textbf{Victor Zhang}}\
% {\Large February 18, 2020}
% \end{center}

\section{}
We may consider the 5-permutation $\sigma$ as the product of a 3-permutation of $\{1,2,3\}$ and a 2-permutation of $\{4,5\}$. It is clear that the 3-permutation has an order of 3, and the 2-permutation has an order of 2. So $\sigma$ has order $\mathrm{lcm}(2,3) = 6$.

\section{}
We prove by induction on $j$. Take base case $j = 1$. $x^ix^1 = x^ix = x^{i+1}$, as desired. Now assume the statement is proved for arbitrary $j = n$. Then $x^ix^{j+1} = x^ix^jx = x^{i+j}x = x^{i+j+1}$, where the last equality is a consequence of the base case. By weak induction, we are done $\Box$

\end{document}
