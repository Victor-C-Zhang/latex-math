\documentclass{article}
\usepackage[utf8]{inputenc}

\title{451 - Weekly 6}
\author{Victor Zhang}
\date{October 16, 2020}

\usepackage[utf8]{inputenc}
\usepackage{amsmath}
\usepackage{amsfonts}
\usepackage{natbib}
\usepackage{graphicx}
% \usepackage{changepage}
\usepackage{amssymb}
\usepackage{xfrac}
% \usepackage{bm}
% \usepackage{empheq}

\newcommand{\contra}{\raisebox{\depth}{\#}}

\newenvironment{myindentpar}[1]
  {\begin{list}{}
          {\setlength{\leftmargin}{#1}}
          \item[]
  }
  {\end{list}}

\pagestyle{empty}

\begin{document}

\maketitle
% \begin{center}
% {\huge Econ 482 \hspace{0.5cm} HW 3}\
% {\Large \textbf{Victor Zhang}}\
% {\Large February 18, 2020}
% \end{center}

\section{}
As shown in class, every transformation $T \in O_n(\mathbb{R})$ can be represented as an orthogonal matrix $A_T \in M_n(\mathbb{R})$ with determinant $\pm 1$. $T$ thus preserves a line if and only if there exists some real-valued eigenvector associated with $A_T$. Consider the linear transformation whose matrix is
\begin{equation*}
A_T = \left(\begin{matrix}
\sfrac{1}{\sqrt{2}} & -\sfrac{1}{\sqrt{2}} & 0 & 0\\
\sfrac{1}{\sqrt{2}} & \sfrac{1}{\sqrt{2}} & 0 & 0\\
0 & 0 & \sfrac{1}{\sqrt{2}} & -\sfrac{1}{\sqrt{2}}\\
0 & 0 & \sfrac{1}{\sqrt{2}} & \sfrac{1}{\sqrt{2}}\\
\end{matrix}
\right)
\end{equation*}
Note this is the composition of two rotations in $\mathbb{R}^2$ so is an orthogonal transformation. Through calculation, it can be shown that this matrix has no real eigenvalues, and thus no real-valued eigenvectors. Thus, this transformation does not preserve any lines $\Box$

\section{}
Note the subgroups of $C_n$ are exactly $\{C_j : j | n\}$. We may list the subgroups beginning with pure cyclic subgroups, pure reflection subgroups, and mixed subgroups. Denote the generators of $D_4$ by $x,y$, where $x$ has order $n$ and $yx = x^{-1}y$. There are 2 pure cyclic subgroups, $\langle x \rangle, \langle x^2 \rangle$. There is one pure reflective subgroup, $\langle y \rangle$. $D_4$ is generated by $\{x,y\}$, there is one subgroup generated by $\{x^2,y\}$, another by $\langle xy \rangle$, and a last one by $\langle x^2y \rangle$. By the group structure of $D_n$, these are all the subgroups. If we were to find a generator with more than one power of $x$, it suffices to pick either $x$ or $x^2$. If we were to find a generator with more than one term $x^iy$, we note $(x^iy)(x^jy) = x^{i-j}$, so it suffices to pick a generator $\{x^k, y\}$ $\Box$

\section{}
All orthogonal transformations in the plane must be a rotation or a reflection. Reflections preserve a line, so must have real-valued eigenvalues. Rotations do not preserve any lines, so do not have real-valued eigenvalues. Thus, we may identify an orthogonal transformation with a reflection or a rotation based on whether it has a real-valued eigenvalue, and we are done. Additionally, if we compute the eigenvalues of a rotation by $\theta$, we get $e^{\pm i\theta}$, so we may recover the angle of rotation of a 2 by 2 orthogonal matrix via this calculation $\Box$

\end{document}

% List of tex snippets:
%   - tex-header (this)
%   - R      --> \mathbb{R}
%   - Z      --> \mathbb{Z}
%   - B      --> \mathcal{B}
%   - E      --> \mathcal{E}
%   - M      --> \mathcal{M}
%   - m      --> \mathfrak{m}({#1})
%   - normlp --> \norm{{#1}}_{L^{{#2}}}
