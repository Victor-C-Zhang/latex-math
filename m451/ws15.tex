\documentclass{article}
\usepackage[utf8]{inputenc}
\usepackage[margin=1in]{geometry}

\title{451 - Worksheet 15}
\author{Victor Zhang}
\date{November 18, 2020}

\usepackage[utf8]{inputenc}
\usepackage{amsmath}
\usepackage{amsfonts}
\usepackage{natbib}
\usepackage{graphicx}
% \usepackage{changepage}
\usepackage{amssymb}
\usepackage{xfrac}
% \usepackage{bm}
% \usepackage{empheq}

\newcommand{\contra}{\raisebox{\depth}{\#}}

\newenvironment{myindentpar}[1]
  {\begin{list}{}
          {\setlength{\leftmargin}{#1}}
          \item[]
  }
  {\end{list}}

\pagestyle{empty}

\begin{document}

\maketitle
% \begin{center}
% {\huge Econ 482 \hspace{0.5cm} HW 3}\
% {\Large \textbf{Victor Zhang}}\
% {\Large February 18, 2020}
% \end{center}

\section{}
Denote this subset by $F$. Consider 2 elements $x = a+b\sqrt{-2}$, $y = c+d\sqrt{-2}$. Clearly, $x + y = y + x = (a+c) + (b+d)\sqrt{-2} \in F$, so $F$ is closed under addition and commutative. Also, the inverse under addition, $x^{-1} = (-a) + (-b)\sqrt{-2} \in F$. The additive identity $0 \in F$ as well. So $(F,+)$ is an Abelian group.\\
Now consider $x,y \in F\setminus \{0\}$. $xy = yx = (ac-2bd) + (ad+bc)\sqrt{-2} \in F$, so $F$ is closed under multiplication and commutative. The inverse under multiplication $x^{-1} = \frac{a}{a^2+2b^2} + \frac{-b}{a^2+2b^2}\sqrt{-2} \in F$. The multiplicative inverse $1 \in F\setminus \{0\}$, so $(F\setminus\{0\}, \times)$ is an Abelian group. So $F$ is a field, in particular a subfield of $\mathbb{C}$ $\Box$

\end{document}

% List of tex snippets:
%   - tex-header (this)
%   - R      --> \mathbb{R}
%   - Z      --> \mathbb{Z}
%   - B      --> \mathcal{B}
%   - E      --> \mathcal{E}
%   - M      --> \mathcal{M}
%   - m      --> \mathfrak{m}({#1})
%   - normlp --> \norm{{#1}}_{L^{{#2}}}
