\documentclass{article}
\usepackage[utf8]{inputenc}
\usepackage[margin=1in]{geometry}

\title{451 - Weekly 9}
\author{Victor Zhang}
\date{November 23, 2020}

\usepackage[utf8]{inputenc}
\usepackage{amsmath}
\usepackage{amsfonts}
\usepackage{natbib}
\usepackage{graphicx}
% \usepackage{changepage}
\usepackage{amssymb}
\usepackage{xfrac}
% \usepackage{bm}
% \usepackage{empheq}

\newcommand{\contra}{\raisebox{\depth}{\#}}

\newenvironment{myindentpar}[1]
  {\begin{list}{}
          {\setlength{\leftmargin}{#1}}
          \item[]
  }
  {\end{list}}

\pagestyle{empty}

\begin{document}

\maketitle
% \begin{center}
% {\huge Econ 482 \hspace{0.5cm} HW 3}\
% {\Large \textbf{Victor Zhang}}\
% {\Large February 18, 2020}
% \end{center}

\section{}
The subset $F = \{a+b\sqrt{3}\}$ is a subfield. Similarly to the subfield $\{a+b\sqrt{2}\}$, $(F,+)$ is an Abelian group. For $a+b\sqrt{3}, c+d\sqrt{3} \in F \setminus \{0\}$, $(a+b\sqrt{3})(c+d\sqrt{3}) = ac+3bd + (ad+bc)\sqrt{3} \in F \setminus \{0\}$. $(a+b\sqrt{3})^{-1} = \frac{a}{a^2-3b^2} - \frac{b}{a^2-3b^2}\sqrt{3} \in F \setminus \{0\}$ so in fact, $(F \setminus \{0\}, \times)$ is an Abelian group. Then $F$ is a subfield of $\mathbb{C}$.\\
The subset $G = \{a+b\sqrt[3]{2}\}$ is not a field. $G \setminus \{0\}$ is not closed under multiplication. To see this, $(a+b\sqrt[3]{2})^2 = a^2 + 2ab\sqrt[3]{2} + b^2\sqrt[3]{2}\sqrt[3]{2} \notin G \setminus \{0\}$, where the last conclusion is a consequence of irrationality of $\sqrt[3]{2}$ $\contra$

\section{}
We claim it is isomorphic to the $n$th roots of unity $Z_n$. Clearly, $Z_n$ is closed under multiplication and inverses. $1 \in Z_n$ as well, so $Z_n < \mathbb{C}^{\times}$. There is an obvious isomorphism $\phi : Z_n \rightarrow C_n$ given by $\phi(z^k) = \sigma^k$ for some generator $z$ of $Z_n$. $\phi(z^i)\phi(z^j) = c^ic^j = c^{i+j} = \phi(z^{i+j}) = \phi(z^iz^j)$ so in fact this is an isomorphism. Since $\sfrac{\mathbb{Z}}{n \mathbb{Z}} \simeq C_n$, we are done $\Box$

\section{}
\subsection{}
Clearly, $\langle g \rangle < G$ for all $g \in G$. By Lagrange, $\lvert \langle g \rangle \rvert \Big\vert |G|$. In other words, $o(g) \big\vert n$. So since $g^{o(g)} = e$, $g^n = e$ as well $\Box$

\subsection{}
If $F$ is a field, $F\setminus \{0\}$ is a multiplicative group with order $n-1$. By 3.1, $x^{n-1} = 1$ for $x \in F\setminus \{0\}$, and $x^n = x$. Since $0^n = 0$, we are done $\Box$

\end{document}

% List of tex snippets:
%   - tex-header (this)
%   - R      --> \mathbb{R}
%   - Z      --> \mathbb{Z}
%   - B      --> \mathcal{B}
%   - E      --> \mathcal{E}
%   - M      --> \mathcal{M}
%   - m      --> \mathfrak{m}({#1})
%   - normlp --> \norm{{#1}}_{L^{{#2}}}
