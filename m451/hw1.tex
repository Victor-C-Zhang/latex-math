\documentclass{article}
\usepackage[utf8]{inputenc}

\title{451 - Weekly 1}
\author{Victor Zhang}
\date{September 10, 2020}

\usepackage[utf8]{inputenc}
\usepackage{amsmath}
\usepackage{amsfonts}
\usepackage{natbib}
\usepackage{graphicx}
% \usepackage{changepage}
\usepackage{amssymb}
\usepackage{xfrac}
% \usepackage{bm}
% \usepackage{empheq}

\newcommand{\contra}{\raisebox{\depth}{\#}}

\newenvironment{myindentpar}[1]
  {\begin{list}{}
          {\setlength{\leftmargin}{#1}}
          \item[]
  }
  {\end{list}}

\pagestyle{empty}

\begin{document}

\maketitle
% \begin{center}
% {\huge Econ 482 \hspace{0.5cm} HW 3}\
% {\Large \textbf{Victor Zhang}}\
% {\Large February 18, 2020}
% \end{center}

\section{}
Take $G = (\mathbb{Z}, +)$ and $H = \mathbb{Z}_{>0}$. $H$ is clearly closed under addition, but doesn't contain identity (zero) so is not a subgroup.

\section{}
Note any element of $G$ must have finite order. Otherwise, we could find a cyclic subgroup $\langle x \rangle < G$ with infinite cardinality. Thus, any closed subset $H$ must contain the identity. Moreover, by closure $H$ contains the cyclic subgroup of every element $a \in H$ so contains its inverse $a^{-1}$ as well. Thus $H$ is a subgroup $\Box$

\section{}
Trivially, $\{1\}$ is a finite subgroup under multiplication. The only other finite subgroup is $\{1,-1\}$. To see why, consider the cyclic subgroup generated by an arbitrary real $r$. Unless $|r| = 1$, $|\langle r \rangle| = \infty$. Thus, we see the only elements that can be included in a finite subgroup are $\pm 1$ $\Box$

\section{}
One possible subgroup is the subset representing quarter-turn rotations:
$$H = \left\{\left( \begin{matrix} 1 & 0 \\ 0 & 1 \end{matrix} \right), \left( \begin{matrix} 0 & -1 \\ 1 & 0 \end{matrix} \right), \left( \begin{matrix} -1 & 0 \\ 0 & -1 \end{matrix} \right), \left( \begin{matrix} 0 & 1 \\ -1 & 0 \end{matrix} \right)\right\}$$
It is easily verified that $H$ is closed, contains identity and all inverses of all elements.

\section{}
Denote the elements $e, c_1, c_2$.
\begin{center}\begin{tabular}{ c c c c }
     & e & $c_1$ & $c_2$\\
    \cline{2-4}
    e & \multicolumn{1}{|c}{e} & \multicolumn{1}{|c|}{$c_1$} & \multicolumn{1}{c|}{$c_2$}\\
    \cline{2-4}
    $c_1$ & \multicolumn{1}{|c}{$c_1$} & \multicolumn{1}{|c|}{$c_2$} & \multicolumn{1}{c|}{e}\\
    \cline{2-4}
    $c_2$ & \multicolumn{1}{|c}{$c_2$} & \multicolumn{1}{|c|}{e} & \multicolumn{1}{c|}{$c_1$}\\
    \cline{2-4}
\end{tabular}\end{center}

\section{}
We have already proved the statement for positive $i,j$. In the case $i,j < 0$ we may take $y = x^{-1}$ and apply the statement for positive exponents. WLOG suppose $i \geq 0$, $j \leq 0$. We may prove this case using induction.\\
Take base cases $j = 0$, $i = 0$. Since $x^0 = e$, the base cases are trivially true. Now take arbitrary $i,j$ and assume the statement holds for $i-1, j+1$. Then $x^ix^j = x^{i-1}xx^{-1}x^{j+1} = x^{i-1}(xx^{-1})x^{j+1} = x^{i-1}x^{j+1}$. By weak induction, we are done $\Box$

\end{document}