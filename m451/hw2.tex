\documentclass{article}
\usepackage[utf8]{inputenc}

\title{451 - Weekly 2}
\author{Victor Zhang}
\date{September 17, 2020}

\usepackage[utf8]{inputenc}
\usepackage{amsmath}
\usepackage{amsfonts}
\usepackage{natbib}
\usepackage{graphicx}
% \usepackage{changepage}
\usepackage{amssymb}
\usepackage{xfrac}
% \usepackage{bm}
% \usepackage{empheq}

\newcommand{\contra}{\raisebox{\depth}{\#}}
\newcommand{\cyclicsg}[1]{\langle {#1} \rangle}

\newenvironment{myindentpar}[1]
  {\begin{list}{}
          {\setlength{\leftmargin}{#1}}
          \item[]
  }
  {\end{list}}

\pagestyle{empty}

\begin{document}

\maketitle
% \begin{center}
% {\huge Econ 482 \hspace{0.5cm} HW 3}\
% {\Large \textbf{Victor Zhang}}\
% {\Large February 18, 2020}
% \end{center}

\section{}
Consider $\cyclicsg{g}$, the cyclic subgroup generated by $g$. Note $\cyclicsg{g}$ is isomorphic to $\mathbb{Z}_9$ with $\phi(g^k) = k$. Thus, the order of $g^2$ w.r.t. $\cyclicsg{g} < G$ is the same as the order of 2 w.r.t. $\mathbb{Z}_9$, which is trivially 9. Similarly, the order of $g^3$ w.r.t. $G$ is the same as the order of 3 w.r.t. $\mathbb{Z}_9$, which is 3.

\section{}
If $h \neq k$, $\phi_g(h) = ghg^{-1} \neq gkg^{-1} = \phi_g(k)$. Thus, $\phi_g$ is injective. For any $k \in G$, $g^{-1}kg \in G$ and $\phi_g(g^{-1}kg) = k$. Thus, $\phi_g$ is surjective. In particular, it is bijective. For arbitrary $h,k \in G$, $\phi_g(hk) = ghkg^{-1} = ghg^{-1}gkg^1 = \phi_g(h)\phi_g(k)$. Thus $\phi_g$ preserves multiplication as well and is thus an isomorphism from $G$ to $G$, in other words, an automorphism $\Box$

\end{document}

% List of tex snippets:
%   - R      --> \mathbb{R}
%   - Z      --> \mathbb{Z}
%   - B      --> \mathcal{B}
%   - E      --> \mathcal{E}
%   - M      --> \mathcal{M}
%   - m      --> \mathfrak{m}({#1})
%   - normlp --> \norm{{#1}}_{L^{{#2}}}
