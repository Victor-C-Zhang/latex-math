\documentclass{article}
\usepackage[utf8]{inputenc}

\title{451 - Worksheet 13}
\author{Victor Zhang}
\date{October 27, 2020}

\usepackage[utf8]{inputenc}
\usepackage{amsmath}
\usepackage{amsfonts}
\usepackage{natbib}
\usepackage{graphicx}
% \usepackage{changepage}
\usepackage{amssymb}
\usepackage{xfrac}
% \usepackage{bm}
% \usepackage{empheq}

\newcommand{\contra}{\raisebox{\depth}{\#}}

\newenvironment{myindentpar}[1]
  {\begin{list}{}
          {\setlength{\leftmargin}{#1}}
          \item[]
  }
  {\end{list}}

\pagestyle{empty}

\begin{document}

\maketitle
% \begin{center}
% {\huge Econ 482 \hspace{0.5cm} HW 3}\
% {\Large \textbf{Victor Zhang}}\
% {\Large February 18, 2020}
% \end{center}

\section{}
Consider the quotient group $\sfrac{G}{H}$. Define group homomorphism $\phi: G \rightarrow \frac{G}{H}$ by $g \mapsto gH$, according to the first isomorphism theorem. The order of $g$ in $G$, $o_G(g)$, must be a multiple of the order of $g$ in $\sfrac{G}{H}$, $o_{\sfrac{G}{H}}(g)$, because $\phi(g^{o_G(g)}) = g^{o_G(g)}H = H$. But since $g^p \in H$, $o_{\sfrac{G}{H}}(g) \big\vert p$. $g \notin H$ so $o_{\sfrac{G}{H}}(g) = p$. Then $p \big\vert o_G(g)$ $\Box$

\end{document}

% List of tex snippets:
%   - tex-header (this)
%   - R      --> \mathbb{R}
%   - Z      --> \mathbb{Z}
%   - B      --> \mathcal{B}
%   - E      --> \mathcal{E}
%   - M      --> \mathcal{M}
%   - m      --> \mathfrak{m}({#1})
%   - normlp --> \norm{{#1}}_{L^{{#2}}}
