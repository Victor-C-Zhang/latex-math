\documentclass{article}
\usepackage[utf8]{inputenc}

\title{451 - Worksheet 3}
\author{Victor Zhang}
\date{September 10, 2020}

\usepackage[utf8]{inputenc}
\usepackage{amsmath}
\usepackage{amsfonts}
\usepackage{natbib}
\usepackage{graphicx}
% \usepackage{changepage}
\usepackage{amssymb}
\usepackage{xfrac}
% \usepackage{bm}
% \usepackage{empheq}

\newcommand{\contra}{\raisebox{\depth}{\#}}

\newenvironment{myindentpar}[1]
  {\begin{list}{}
          {\setlength{\leftmargin}{#1}}
          \item[]
  }
  {\end{list}}

\pagestyle{empty}

\begin{document}

\maketitle
% \begin{center}
% {\huge Econ 482 \hspace{0.5cm} HW 3}\
% {\Large \textbf{Victor Zhang}}\
% {\Large February 18, 2020}
% \end{center}

\section{}
By definition, for isomorphism $\varphi$, $\varphi(gh) = \varphi(g)\varphi(h)$.
If there were such an isomorphism $\varphi: V \rightarrow C_4$, $\varphi(e) = \varphi(\sigma^2) = \varphi(\sigma)\varphi(\sigma)$, and similarly $\varphi(e) = \varphi(\tau)\varphi(\tau)$. But $C_4$ only contains one non-identity element who behaves this way, namely $c_2$. Thus, there must not exist any isomorphism from $V$ to $C_4$.

\end{document}