\documentclass{article}
\usepackage[utf8]{inputenc}

\title{451 - Weekly 7}
\author{Victor Zhang}
\date{October 25, 2020}

\usepackage[utf8]{inputenc}
\usepackage{amsmath}
\usepackage{amsfonts}
\usepackage{natbib}
\usepackage{graphicx}
% \usepackage{changepage}
\usepackage{amssymb}
\usepackage{xfrac}
% \usepackage{bm}
% \usepackage{empheq}

\newcommand{\contra}{\raisebox{\depth}{\#}}

\newenvironment{myindentpar}[1]
  {\begin{list}{}
          {\setlength{\leftmargin}{#1}}
          \item[]
  }
  {\end{list}}

\pagestyle{empty}

\begin{document}

\maketitle
% \begin{center}
% {\huge Econ 482 \hspace{0.5cm} HW 3}\
% {\Large \textbf{Victor Zhang}}\
% {\Large February 18, 2020}
% \end{center}

\section{}
We investigate the conjugacy classes of pure rotations $x^{i}$, pure reflection $y$ and combination elements $yx^{i}$:\\
$(yx^{j})(x^{i})(yx^{j})^{-1} = x^{-i}$ and $(x^{j})(x^{i})(x^{j})^{-1} = x^{i}$ so the conjugacy class of $x^{i}$ is $\{x^{i}, x^{-i}\}$.\\
$(yx^{j})(y)(yx^{j})^{-1} = yx^{2j}$ and $(x^{j})(y)(x^{j})^{-1} = yx^{-2j}$ so the conjugacy class of $y$ is $\{yx^{2i} : i \in \mathbb{Z}\}$.\\
$(yx^{j})(yx^{i})(yx^{j})^{-1} = yx^{2j-i}$ and $(x^{j})(yx^{i})(x^{j})^{-1} = yx^{i-2j}$ so the conjugacy class of $yx^{i}$ is $\{yx^{\pm(i-2j)} : j \in \mathbb{Z}\}$.\\
If $n$ is even, the conjugacy classes of $D_n$ are $\{x^i, x^{-i}\}, \{yx^{2k} : k \in \mathbb{Z}\}, \{yx^{2k+1} : k \in \mathbb{Z}\}$. If $n$ is odd, the conjugacy classes of $D_n$ are $\{x^i, x^{-i}\}, \{yx^{k} : k \in \mathbb{Z}\}$ $\Box$

\section{}
We know $a(2) = 3$, $a(3) = 5$, $a(4) = 6$ is a necessary constraint on $a$. Then $a = (2\, 3\, 5)(4\, 6)$ is a suitable permutation $\Box$

\section{}
As demonstrated in class, we may decompose $$(1\, 2\, 3 \dots m) = (1\, m)(1\, (m-1)) \dots (1\, 2)$$Then this permutation is even if $m$ is odd $\Box$

\section{}
Since $f$ is an automorphism, it must preserve multiplication. That is, for arbitrary $x,y$, $f(x)f(y)=f(xy)$. Denote by $H$ the set of $x \in G$ s.t. $f(x) = x^{-1}$. If $x,y \in H$ we have that $x^{-1}y^{-1} = {(xy)}^{-1} = y^{-1}x^{-1}$. Thus, $H^{-1} = \{h^{-1} : h \in H\}$ is an Abelian subset of $G$ with cardinality $> \frac{3}{4}|G|$. Observe that the centralizer of any group element is a subgroup, and we know the centralizer for $x \in H^{-1}$ has cardinality at least $\frac{3}{4}|G|$, so in fact, $Z(x) = G$ by Lagrange. But then $x \in Z(G)$ and $|Z(G)| > \frac{3}{4}|G|$, so since the centralizer of a group is a subgroup, $Z(G) = G$ by Lagrange. Then in fact, $G$ is Abelian.\\
Now we show $f(x) = x^{-1}$ for all $x$. Suppose not, that is, there is some $x$ s.t. $f(x) \neq x^{-1}$. But then for all $y$, we must have that $f(x)y^{-1} = f(xy)$. Further, $f(xy) \neq {(xy)}^{-1}$, or we arrive at a contradiction. But then there must be at least $\frac{3}{4}|G|$ unique elements $xy$ s.t. $f(xy) \neq {(xy)}^{-1}$ $\contra$

\end{document}

% List of tex snippets:
%   - tex-header (this)
%   - R      --> \mathbb{R}
%   - Z      --> \mathbb{Z}
%   - B      --> \mathcal{B}
%   - E      --> \mathcal{E}
%   - M      --> \mathcal{M}
%   - m      --> \mathfrak{m}({#1})
%   - normlp --> \norm{{#1}}_{L^{{#2}}}
