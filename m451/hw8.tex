\documentclass{article}
\usepackage[utf8]{inputenc}

\title{451 - Weekly 8}
\author{Victor Zhang}
\date{October 30, 2020}

\usepackage[utf8]{inputenc}
\usepackage{amsmath}
\usepackage{amsfonts}
\usepackage{natbib}
\usepackage{graphicx}
% \usepackage{changepage}
\usepackage{amssymb}
\usepackage{xfrac}
% \usepackage{bm}
% \usepackage{empheq}

\newcommand{\contra}{\raisebox{\depth}{\#}}

\newenvironment{myindentpar}[1]
  {\begin{list}{}
          {\setlength{\leftmargin}{#1}}
          \item[]
  }
  {\end{list}}

\pagestyle{empty}

\begin{document}

\maketitle
% \begin{center}
% {\huge Econ 482 \hspace{0.5cm} HW 3}\
% {\Large \textbf{Victor Zhang}}\
% {\Large February 18, 2020}
% \end{center}

\section{}
For arbitrary $g,h \in G$ we have
$$gh = (gh)^{-1} = h^{-1}g^{-1} = hg$$
So $g$ and $h$ commute and thus $G$ is Abelian $\Box$

\section{}
\subsection{}
We may say for all $n \in N < G$, $k \in K$ that $nkn^{-1} \in K$. Then $K \lhd N$. By the first isomorphism theorem, we may identify elements of $\sfrac{G}{K}$ with cosets $gK$, $g \in G$, and elements of $\sfrac{N}{K}$ with cosets $nK$, $n \in N$. Then since $N < G$, $n \in G$ so $nK \in \sfrac{G}{K}$. Thus $\sfrac{N}{K} < \sfrac{G}{K}$ as well $\Box$

\subsection{}
Again by the first isomorphism theorem, we may consider homomorphism $G \xrightarrow{\phi} \sfrac{G}{K}$ given by $g \mapsto gK$. Then we note $(gK)^{-1} = g^{-1}K$ and $(gK)(hK) = (gh)K$. So then $(gK)(nK)(gK)^{-1} = (gng^{-1})K$ and in fact, $\sfrac{N}{K} \lhd \sfrac{G}{K}$ iff $N \lhd G$ $\Box$

\subsection{}
By 2.2 we have that $\sfrac{N}{K} \lhd \sfrac{G}{K}$. We claim the map $\sfrac{G}{K} \xrightarrow{\phi} \sfrac{G}{N}$ given by $gK \mapsto gN$ is a homomorphism. Indeed, $\phi(gK)\phi(hK) = (gN)(hN) = (gh)N = \phi((gh)K)$ $\Box$

\subsection{}
We may represent the homomorphism in 2.3 by an equivalence relation, $\sim$, given by: $gK \sim hK$ if $gN = hN$. The equivalence class of $eK$ is then all elements $gK$ s.t. $gN = eN = N$. That is, $g \in N$ and thus $gK \in \sfrac{N}{K}$. Then the equivalence class of $eK$ is simply $\sfrac{N}{K}$. $\sfrac{N}{K}$ is the normal subgroup associated with homomorphism $\phi$, so by the first isomorphism theorem $\sfrac{\sfrac{G}{K}}{\sfrac{N}{K}} \simeq \sfrac{G}{N}$ $\Box$

\end{document}

% List of tex snippets:
%   - tex-header (this)
%   - R      --> \mathbb{R}
%   - Z      --> \mathbb{Z}
%   - B      --> \mathcal{B}
%   - E      --> \mathcal{E}
%   - M      --> \mathcal{M}
%   - m      --> \mathfrak{m}({#1})
%   - normlp --> \norm{{#1}}_{L^{{#2}}}
