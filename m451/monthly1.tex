\documentclass{article}
\usepackage[utf8]{inputenc}
\usepackage[margin=1in]{geometry}

\title{451 - Monthly 1}
\author{Victor Zhang}
\date{November 10, 2020}

\usepackage[utf8]{inputenc}
\usepackage{amsmath}
\usepackage{amsfonts}
\usepackage{natbib}
\usepackage{graphicx}
% \usepackage{changepage}
\usepackage{amssymb}
\usepackage{xfrac}
% \usepackage{bm}
% \usepackage{empheq}
\usepackage{dirtytalk}

\newcommand{\contra}{\raisebox{\depth}{\#}}

\newenvironment{myindentpar}[1]
  {\begin{list}{}
          {\setlength{\leftmargin}{#1}}
          \item[]
  }
  {\end{list}}

\pagestyle{empty}

\begin{document}

\maketitle
% \begin{center}
% {\huge Econ 482 \hspace{0.5cm} HW 3}\
% {\Large \textbf{Victor Zhang}}\
% {\Large February 18, 2020}
% \end{center}

\section{General groups}
\subsection{}
Note if $x,y \in H_1 \cap H_2$ then $x,y \in H_1, H_2$. Thus $xy \in H_1, H_2$ so $xy \in H_1 \cap H_2$ and $H_1 \cap H_2$ is closed in multiplication. Similarly, if $x \in H_1 \cap H_2$, $x^{-1} \in H_1, H_2$, so $x^{-1} \in H_1 \cap H_2$ and $H_1 \cap H_2$ is closed in inverses. $e \in H_1 \cap H_2$ so in fact $H_1 \cap H_2$ is a subgroup $\Box$

\subsection{}
For $n \in H_1 \cap H_2$, by normality of $H_1$, $H_2$, we have $gng^{-1} \in H_1, H_2$ for all $g \in G$. Then $gng^{-1} \in H_1 \cap H_2$ so $H_1 \cap H_2$ is normal $\Box$

\subsection{}
For $g \in H_1$, $h \in H_1 \cap H_2$, $ghg^{-1} \in H_2$. But since $h \in H_1$ we have that $ghg^{-1} \in H_1$ and in fact $ghg^{-1} \in H_1 \cap H_2$. So $H_1 \cap H_2 \lhd H_1$ $\Box$ 

\section{Actions}
\subsection{}
For all $hk \in HK$, the orbit $G(hk)$ is simply $Hk = \{gk : g \in H\}$. We may see this by first observing $G(hk) \subseteq Hk$. Then we note that any $h' \in H$ may be written $gh^{-1}$, where $g = h'h \in H$. So it follows that $h'hk = gh^{-1}hk = gk$ for $g \in H$ and $Hk \subseteq G(hk)$. Then we are done, and $|G(hk)| = |H|$ is the size of this and all other orbits $\Box$

\subsection{}
For any orbit $O$, pick some arbitrary element $g \in O \subset HK$. We may decompose $g = hk$ for some $h \in H$, $k \in K$. Then clearly $\phi(k) = O$ from the form derived in the previous section. It follows that $\phi$ is surjective $\Box$

\subsection{}
Note if $\phi(k) = \phi(k')$ there must be some $h \in H$ s.t. $hk = k'$ since $k' = ek' \in \phi(k') = \phi(k)$. But since $K$ is a subgroup, $h = k'h^{-1} \in K$ so in fact $h \in H \cap K$. By problem 1.1, $H \cap K$ is a subgroup. Then 
$$(H \cap K)k = \{gk : g \in H \cap K\} = \{g'hk : g' \in H \cap K\} = \{g'(hk) : g' \in H \cap K\} = (H \cap K)k'$$
This proves the forward equality. Since $e \in H \cap K$, $k,k' \in (H \cap K)k$. So there exists some $h$ s.t. $k' = hk$. Then $k'$ is in the orbit of $k$ and $\phi(k) = \phi(k')$ $\Box$

\subsection{}
If we again consider $\phi$, each $k$ is mapped to $|H|$ elements of $HK$. From the previous section, we recall that $\phi(k) = \phi(k')$ if there is some $h \in H$ s.t. $hk = k'$. Since $K$ is a subgroup, we may say further $h \in H \cap K$. Then there are $|H \cap K|$ elements $k'$ that share the same orbit as $k$ (including $k$ itself). By elementary counting, there are $\frac{|H||K|}{|H \cap K|}$ elements in $HK$ and the result follows immediately $\Box$

\section{Symmetries and conjugacy classes}
\subsection{}
Consider $v$ under action of $G$. The orbit of $v$ contains two points: the \say{top} and \say{bottom} of the solid. There are 5 rotations which stabilize $v$. If we count reflections as well, that gets us to 10 symmetries in $\mathrm{Stab}(v)$. Then by orbit stabilizer, $|G| = 20$ $\Box$

\subsection{}
$|G_v| = |\mathrm{Stab}(v)| = 10$. Any edge $e_i$ may be brought to any other edge $e_j$ by some rotation or combination of rotation and inversion (of top and bottom), so $|G_e| = 2$ by orbit-stabilizer. Every face may be brought to any other face similarly, so $|G_f| = 2$ as well $\Box$

\subsection{}
The group of symmetries which fix $v$ are exactly the group of rotations or reflections of the five faces which meet at $v$. This is the dihedral group $D_5$. For all $h \in H$ and abritrary $g \in G$, $ghg^{-1}(v) = v$ since $h$ fixes $v$. Thus, $ghg^{-1}$ fixes $v$ and thus $ghg^{-1} \in H$. Then $H \lhd G$ $\Box$

\subsection{}
Note that every symmetry may be considered the composition of (some or all of) a rotation, reflection, or inversion. Rotation and reflection are inherited from $D_5$, and we say \say{inversion} to mean a symmetry that switches the top and bottom while minimizing rotation. For instance, one possible inversion $\sigma_e$ would bring the face labelled 9 to the face labelled 8 and the face labelled 2 to the face labelled 9. Then $K = \langle \sigma_e \rangle \simeq C_{10}$. Moreover, $\sigma_e^2$ is equivalent to $\sigma_v$, a rotation by $\sfrac{2\pi}{5}$, so in fact $C_{10} \cap D_5 = C_5$. Then $K$ must be normalized by $C_5$, the group of rotations in $G$. Since $|K| = 10$ and all elements preserve orientation, it follows that reflection must also normalize $K$. Inversion trivially normalizes $K$, so in fact, any composition of the three also normalize $K$, so $K \lhd G$ $\Box$

\subsection{}
First we deal with all elements of $C_5$. The action of $\sigma_v$ is invariant under rotation and inversion, and is $\sigma_v^{-1}$ under reflection. So $C_G(\sigma_v) = \{\sigma_v, \sigma_v^4\}$. Similarly, the conjugacy class of $\sigma_v^2$ is $C_G(\sigma_v^2) = \{\sigma_v^2, \sigma_v^3\}$. The identity commutes with every element, so is in its own conjugacy class $\{e\}$.\\
Now we look at the elements of $D_5$. Denote a reflection by $\sigma_f$. $\sigma_f \sigma_v = \sigma_v^{-1}\sigma_f$ from the group structure of $D_5$ so $\sigma_f\sigma_v^i \in C_G(\sigma_f)$. $\sigma_e \sigma_f \sigma_e^{-1} = \sigma_f\sigma_v^{-1}$, so in fact $C_G(\sigma_f) = \{\sigma_f\sigma_v^{i}\}$.\\
Then consider the elements of $C_{10}$. $\sigma_e$ is invariant under rotation.
$$\sigma_f \sigma_e \sigma_f^{-1} = \sigma_e^{-1} = \sigma_e^{9}$$
so $C_G(\sigma_e) = \{\sigma_e, \sigma_e^{9} \}$. Similarly, $C_G(\sigma_e^3) = \{\sigma_e^3, \sigma_e^{7} \}$ and $C_G(\sigma_e^5) = \{\sigma_e^5\}$.\\
The only elements left to consider are those in the form $\sigma_v^i\sigma_e\sigma_f$.
$$(\sigma_e^k \sigma_f)(\sigma_e\sigma_f)(\sigma_e^k\sigma_f)^{-1} = \sigma_e^{2k-1}\sigma_f$$
Then in fact $C_G(\sigma_e\sigma_f) = \{\sigma_v^{i}\sigma_e\sigma_f\}$.\\
Summing the conjugacy classes we have found, we see that $2 + 2 + 1 + 5 + 2 + 2 + 1 + 5 = 20$, so we have found all the conjugacy classes of $G$. We may thus take $\{\sigma_v, \sigma_v^2, e, \sigma_f, \sigma_e, \sigma_e^3, \sigma_e^5, \sigma_e\sigma_f\}$ a set of representatives. $\Box$

\subsection{}
There are 2 singleton conjugacy classes. These elements comprise the center $|Z(G)|$. Since $(\sigma_e^5)^2 = e$, $Z(G) \simeq C_2$ $\Box$


\end{document}

% List of tex snippets:
%   - tex-header (this)
%   - R      --> \mathbb{R}
%   - Z      --> \mathbb{Z}
%   - B      --> \mathcal{B}
%   - E      --> \mathcal{E}
%   - M      --> \mathcal{M}
%   - m      --> \mathfrak{m}({#1})
%   - normlp --> \norm{{#1}}_{L^{{#2}}}
