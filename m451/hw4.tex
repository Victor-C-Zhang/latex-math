\documentclass{article}
\usepackage[utf8]{inputenc}

\title{451 - Weekly 4}
\author{Victor Zhang}
\date{October 1, 2020}

\usepackage[utf8]{inputenc}
\usepackage{amsmath}
\usepackage{amsfonts}
\usepackage{natbib}
\usepackage{graphicx}
% \usepackage{changepage}
\usepackage{amssymb}
\usepackage{xfrac}
% \usepackage{bm}
% \usepackage{empheq}

\newcommand{\contra}{\raisebox{\depth}{\#}}

\newenvironment{myindentpar}[1]
  {\begin{list}{}
          {\setlength{\leftmargin}{#1}}
          \item[]
  }
  {\end{list}}

\pagestyle{empty}

\begin{document}

\maketitle
% \begin{center}
% {\huge Econ 482 \hspace{0.5cm} HW 3}\
% {\Large \textbf{Victor Zhang}}\
% {\Large February 18, 2020}
% \end{center}

\section{}
It suffices to show $H_1 \cap H_2$ is a group, since $S \subset H_1 \cap H_2 \subset G$. Clearly, $e \in H_1$ and $e \in H_2$ so $e \in H_1 \cap H_2$. If $x,y \in H_1 \cap H_2$ then $x,y \in H_1$ and $x,y \in H_2$ so by the group structure, $xy \in H_1$, $xy \in H_2$, $xy \in H_1 \cap H_2$. Thus, $H_1 \cap H_2$ is closed in multiplcation since $H_1$ and $H_2$ are closed in multiplication. Similarly, since $H_1$, $H_2$ are closed in inverses, $H_1 \cap H_2$ is closed in inverses as well. Thus, $H_1 \cap H_2$ is a group $\Box$

\section{}
Consider the set $\mathcal{H}$ of groups $F < G$ that contain $S$. Put $H = \bigcap\limits_{\mathcal{H}} F$. By problem 1, $H$ is a group containing $S$, and $H \subseteq H'$ for all $H' \in \mathcal{H}$, so we are done $\Box$

\section{}
Note the smallest subgroup containing a single element is the cyclic subgroup generated by that element. Following the analysis of elements of $S_4$ in Weekly Worksheet 3, we note that elements of $S_3$ have order at most 3. Thus, no one element may generate $|S_3| = 6$ elements. However, take $\sigma$ to have order 2 and $\tau$ to have order 3. $\sigma \neq \tau$ so with some algebra we see that the set $\{\sigma, \tau\}$ generates 6 unique objects and thus $S_3$. For example, we may take $\sigma = \{(1,2), (2,1), (3,3)\}$ and $\tau = \{(1,2),(2,3),(3,1)\}$ $\Box$

\section{}
We claim $\langle \tau \rangle$ is such a normal subgroup. If $g \in \langle \tau \rangle$ it is clear that $g \langle \tau \rangle g^{-1} = \langle \tau \rangle$. Otherwise, $g$ can be written in the form $\sigma\tau^k$. We note that $\sigma\tau\sigma = \tau^2$ and $\sigma\tau^2\sigma = \tau$, so $(\sigma\tau^k) \tau^j (\sigma\tau^k)^{-1} = \sigma\tau^k\tau^j\tau^{-k}\sigma^{-1} = \sigma\tau^j\sigma \in \langle \tau \rangle$, so in fact $g \langle \tau \rangle g^{-1} = \langle \tau \rangle$ $\Box$


\end{document}

% List of tex snippets:
%   - tex-header (this)
%   - R      --> \mathbb{R}
%   - Z      --> \mathbb{Z}
%   - B      --> \mathcal{B}
%   - E      --> \mathcal{E}
%   - M      --> \mathcal{M}
%   - m      --> \mathfrak{m}({#1})
%   - normlp --> \norm{{#1}}_{L^{{#2}}}
