\documentclass{article}
\usepackage[utf8]{inputenc}

\title{451 - Worksheet 5}
\author{Victor Zhang}
\date{September 23, 2020}

\usepackage[utf8]{inputenc}
\usepackage{amsmath}
\usepackage{amsfonts}
\usepackage{natbib}
\usepackage{graphicx}
% \usepackage{changepage}
\usepackage{amssymb}
\usepackage{xfrac}
% \usepackage{bm}
% \usepackage{empheq}

\newcommand{\contra}{\raisebox{\depth}{\#}}

\newenvironment{myindentpar}[1]
  {\begin{list}{}
          {\setlength{\leftmargin}{#1}}
          \item[]
  }
  {\end{list}}

\pagestyle{empty}

\begin{document}

\maketitle
% \begin{center}
% {\huge Econ 482 \hspace{0.5cm} HW 3}\
% {\Large \textbf{Victor Zhang}}\
% {\Large February 18, 2020}
% \end{center}

\section{}
An element $x \in S_4$ has order 2 iff it is its own inverse. That is, if we regard $x$ as a permutation $\phi: [4] \rightarrow [4]$, $\phi(\alpha) = \beta \Longleftrightarrow \phi(\beta) = \alpha$. So $\phi$ either partitions $[4]$ into two pairs or into one pair and two singlets. In total, there are $\frac{1}{2}\binom{4}{2} + \binom{4}{2} = 9$ such $\phi$ $\Box$

\section{}
By Lagrange and its corollary, an element $x \in S_4$ may only have order a divisor of $|S_4| = 24$. Clearly, we may find elements of orders 1,2,3,4, so it suffices to show $x$ may not have order greater than 4. We may identify $x$ with some $\phi$ as defined in problem 1. Suppose $x$ has order $k > 6$. Note for $i \in [4]$, the set $\langle i \rangle = \{\phi^t(i) \, : \, t \in \mathbb{Z}\}$ has k distinct elements. Moreover, all elements $j$ in $\langle i \rangle$ have $\langle j \rangle = \langle i \rangle$ and thus the size of $\langle i \rangle$ is the order of each $j \in \langle i \rangle$. Thus, we may partition $[4]$ into cyclic subsets with sizes that are coprime and multiply to $k$. But if $k > 4$, any such decomposition of $k$ will have sum at greater than 4 $\contra$
\end{document}

% List of tex snippets:
%   - tex-header (this)
%   - R      --> \mathbb{R}
%   - Z      --> \mathbb{Z}
%   - B      --> \mathcal{B}
%   - E      --> \mathcal{E}
%   - M      --> \mathcal{M}
%   - m      --> \mathfrak{m}({#1})
%   - normlp --> \norm{{#1}}_{L^{{#2}}}
