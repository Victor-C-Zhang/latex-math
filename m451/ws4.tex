\documentclass{article}
\usepackage[utf8]{inputenc}

\title{451 - Worksheet 4}
\author{Victor Zhang}
\date{September 16, 2020}

\usepackage[utf8]{inputenc}
\usepackage{amsmath}
\usepackage{amsfonts}
\usepackage{natbib}
\usepackage{graphicx}
% \usepackage{changepage}
\usepackage{amssymb}
\usepackage{xfrac}
% \usepackage{bm}
% \usepackage{empheq}

\newcommand{\contra}{\raisebox{\depth}{\#}}

\newenvironment{myindentpar}[1]
  {\begin{list}{}
          {\setlength{\leftmargin}{#1}}
          \item[]
  }
  {\end{list}}

\pagestyle{empty}

\begin{document}

\maketitle
% \begin{center}
% {\huge Econ 482 \hspace{0.5cm} HW 3}\
% {\Large \textbf{Victor Zhang}}\
% {\Large February 18, 2020}
% \end{center}

\section{}
\subsection{}
Since $S_3$ consists of functions $\mathbb{R}^3 \rightarrow \mathbb{R}^3$, it is monomorphic to $GL_3(\mathbb{R})$. Thus to find the kernel of the action of $S_3$ on itself via conjugation, it suffices to find the kernel for the action of $GL_3(\mathbb{R})$ on itself via conjugation, restricted to the subgroup induced by $S_3$. The kernel of the unrestricted action is clearly $\{\lambda I \; | \; \lambda \in \mathbb{R}\}$, so the kernel of the restricted action is $\{I\}$. Thus, the kernel of $S_3$, and in fact, any permutation group $S_n$, acting on itself is simply $\{e\}$ $\Box$

\subsection{}
Since there are only 6 elements of $S_3$, we may brute-force the orbit easily. In this case, it is $\{\tau, \tau^{-1}\}$. Alternatively, note that $\tau$ maps every element to a different element. Conjugation simply changes the basis, and not the fact that every element is mapped to a different element. The only members of $S_3$ that have this property are $\tau$ and $\tau^{-1}$, so the result follows $\Box$

\end{document}

% List of tex snippets:
%   - tex-header (this)
%   - R      --> \mathbb{R}
%   - Z      --> \mathbb{Z}
%   - B      --> \mathcal{B}
%   - E      --> \mathcal{E}
%   - M      --> \mathcal{M}
%   - m      --> \mathfrak{m}({#1})
%   - normlp --> \norm{{#1}}_{L^{{#2}}}
