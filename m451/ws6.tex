\documentclass{article}
\usepackage[utf8]{inputenc}

\title{451 - Worksheet 6}
\author{Victor Zhang}
\date{September 24, 2020}

\usepackage[utf8]{inputenc}
\usepackage{amsmath}
\usepackage{amsfonts}
\usepackage{natbib}
\usepackage{graphicx}
% \usepackage{changepage}
\usepackage{amssymb}
\usepackage{xfrac}
% \usepackage{bm}
% \usepackage{empheq}

\newcommand{\contra}{\raisebox{\depth}{\#}}

\newenvironment{myindentpar}[1]
  {\begin{list}{}
          {\setlength{\leftmargin}{#1}}
          \item[]
  }
  {\end{list}}

\pagestyle{empty}

\begin{document}

\maketitle
% \begin{center}
% {\huge Econ 482 \hspace{0.5cm} HW 3}\
% {\Large \textbf{Victor Zhang}}\
% {\Large February 18, 2020}
% \end{center}

\section{}
By the corollary to the first isomorphism theorem, it suffices to find a homomorphism $\phi: \mathbb{Z} \rightarrow C_n$ with kernel $n\mathbb{Z}$. Indeed, $\phi(x) := \sigma^{(x\mod{n})}$ is such a map. $\phi(x)\phi(y) = \sigma^{(x\mod{n})}\sigma^{(y\mod{n})} = \sigma^{(x+y\mod{n})} = \phi(x+y)$ so $\phi$ preserves multiplication (+) and is a homomorphism. If $\phi(x) = e = \sigma^0$ then $x \equiv 0 \mod{n}$. In other words, $x \in n \mathbb{Z}$. Thus, the kernel of $\phi$ is $n \mathbb{Z}$ and we are done $\Box$

\end{document}

% List of tex snippets:
%   - tex-header (this)
%   - R      --> \mathbb{R}
%   - Z      --> \mathbb{Z}
%   - B      --> \mathcal{B}
%   - E      --> \mathcal{E}
%   - M      --> \mathcal{M}
%   - m      --> \mathfrak{m}({#1})
%   - normlp --> \norm{{#1}}_{L^{{#2}}}