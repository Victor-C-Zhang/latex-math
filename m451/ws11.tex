\documentclass{article}
\usepackage[utf8]{inputenc}

\title{451 - Worksheet 11}
\author{Victor Zhang}
\date{October 21, 2020}

\usepackage[utf8]{inputenc}
\usepackage{amsmath}
\usepackage{amsfonts}
\usepackage{natbib}
\usepackage{graphicx}
% \usepackage{changepage}
\usepackage{amssymb}
\usepackage{xfrac}
% \usepackage{bm}
% \usepackage{empheq}

\newcommand{\contra}{\raisebox{\depth}{\#}}

\newenvironment{myindentpar}[1]
  {\begin{list}{}
          {\setlength{\leftmargin}{#1}}
          \item[]
  }
  {\end{list}}

\pagestyle{empty}

\begin{document}

\maketitle
% \begin{center}
% {\huge Econ 482 \hspace{0.5cm} HW 3}\
% {\Large \textbf{Victor Zhang}}\
% {\Large February 18, 2020}
% \end{center}

\section{}
Suppose $h_1, h_2 \in C_G(g)$. Then $gh_1h_2 = h_1gh_2 = h_1h_2g$ and thus $h_1h_2 \in C_G(g)$. So $C_G(g)$ is closed under multiplication.
For $h \in C_G(g)$, $g = ghh^{-1} = hgh^{-1}$. But also note $g = hh^{-1}g$ so thus $gh^{-1} = h^{-1}g$ and $C_G(g)$ is closed in inverses.
$ge = eg$ so $C_G(g)$ also contains the identity. So $C_G(g)$ is a subgroup $\Box$

\section{}
By normality of $N$, $gng^{-1} \in N$ for all $g \in G$, $n \in N$. For any $h \in C_G(N)$ we may write
$$ghg^{-1}n = ghg^{-1}ngg^{-1} = gh(g^{-1}ng)g^{-1} = g(g^{-1}ng)hg^{-1} = nghg^{-1}$$
So in fact $ghg^{-1} \in C_G(N)$ and thus the centralizer is a normal subgroup $\Box$

\end{document}
