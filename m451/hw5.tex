\documentclass{article}
\usepackage[utf8]{inputenc}

\title{451 - Homework 5}
\author{Victor Zhang}
\date{October 9, 2020}

\usepackage[utf8]{inputenc}
\usepackage{amsmath}
\usepackage{amsfonts}
\usepackage{natbib}
\usepackage{graphicx}
% \usepackage{changepage}
\usepackage{amssymb}
\usepackage{xfrac}
% \usepackage{bm}
% \usepackage{empheq}

\newcommand{\contra}{\raisebox{\depth}{\#}}

\newenvironment{myindentpar}[1]
  {\begin{list}{}
          {\setlength{\leftmargin}{#1}}
          \item[]
  }
  {\end{list}}

\pagestyle{empty}

\begin{document}

\maketitle
% \begin{center}
% {\huge Econ 482 \hspace{0.5cm} HW 3}\
% {\Large \textbf{Victor Zhang}}\
% {\Large February 18, 2020}
% \end{center}

\section{}
\subsection{}
If $g$ is a reflection, this is trivially true. If $g$ is a translation, it is similarly trivially true. Thus, it is true for glide reflections, since for all glide reflections we may commute the translation $t$ and reflection $r$, that is, $tr = rt$. It suffices to show $gt_ag^{-1}$ is a translation when $g$ is a rotation and $a \in \mathbb{R}^2$.\\
Denote by $p$ the center of rotation, $\theta$ the angle of rotation. Denote by $l$ the line through $p$ parallel to $a$. Consider the line $m = g(l)$. The action of $gt_ag^{-1}$ on $m$ is equivalent to translating the line by some amount in a direction parallel to $m$. Thus, $gt_ag^{-1} \in T$, and we are done $\Box$

\subsection{}
In general, conjugation defines an action $G \times G \rightarrow G$. By the previous section, $\cdot$ is, in fact, an action $G \times T \rightarrow T$ $\Box$

\subsection{}
We identify a translation $t_a$ with some $a \in \mathbb{R}^2$.If we denote the origin with $o$, the action $g\cdot t_a$ may be identified by $g'(a)$, where $g' = g$ if $g$ is a translation and $g' = t_{-g(o)}g$ otherwise $\Box$

\section{}
\subsection{}
As derived on previous assignments, the number of orientation-preserving symmetries of a cube is 24. The orbit of a corner contains all the 8 corners of the cube, so by orbit-stabilizer, the order of the stabilizer of a corner is $24/8 = 3$ $\Box$

\subsection{}
The orbit of a face contains the 6 faces of the cube, so appealing once again to orbit-stabilizer, the order of the stabilizer of the face is $24/6 = 4$ $\Box$

\subsection{}
We take inspiration from section 2.1, noting rotation through a space diagonal by $\frac{2\pi}{3}$ is a symmetry with order 3. Thus, we may split the cake by first positioning the cake such that one of its space diagonals $l$ is perpendicular to some flat plane of reference (read: table), making an arbitrary cut perpendicular to the table, through the entire cake vertically, but stopping at the space diagonally horizontally. Turn the cake along the space diagonal by $\frac{2\pi}{3}$. Repeat until you have 3 pieces. By rotational symmetry, all pieces will be identical $\Box$

\end{document}

% List of tex snippets:
%   - tex-header (this)
%   - R      --> \mathbb{R}
%   - Z      --> \mathbb{Z}
%   - B      --> \mathcal{B}
%   - E      --> \mathcal{E}
%   - M      --> \mathcal{M}
%   - m      --> \mathfrak{m}({#1})
%   - normlp --> \norm{{#1}}_{L^{{#2}}}
