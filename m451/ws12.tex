\documentclass{article}
\usepackage[utf8]{inputenc}

\title{451 - Worksheet 12}
\author{Victor Zhang}
\date{October 23, 2020}

\usepackage[utf8]{inputenc}
\usepackage{amsmath}
\usepackage{amsfonts}
\usepackage{natbib}
\usepackage{graphicx}
% \usepackage{changepage}
\usepackage{amssymb}
\usepackage{xfrac}
% \usepackage{bm}
% \usepackage{empheq}

\newcommand{\contra}{\raisebox{\depth}{\#}}

\newenvironment{myindentpar}[1]
  {\begin{list}{}
          {\setlength{\leftmargin}{#1}}
          \item[]
  }
  {\end{list}}

\pagestyle{empty}

\begin{document}

\maketitle
% \begin{center}
% {\huge Econ 482 \hspace{0.5cm} HW 3}\
% {\Large \textbf{Victor Zhang}}\
% {\Large February 18, 2020}
% \end{center}

\section{}
Recall the definition of an element in the product algebra $N \times \overline{G}$ is a tuple $(n,g)$, $n \in N$, $g \in \overline{G}$. We define an equivalence relation $\sim$ as follows: $(n_1,g_1) \sim (n_2,g_2)$ if $g_1 = g_2$. Then the equivalence class $[e] = N \times \{e\} := N'$. Note $\sim$ defines a group homomorphism $\phi: G \rightarrow \overline{G}$ since $\phi$ preserves multiplication according to the group structure of $\overline{G}$. Then by the first isomorphism theorem, $G/N{'} \simeq \overline{G}$. The kernel of this isomorphism is $[e] = N' \simeq N$ $\Box$

\end{document}

% List of tex snippets:
%   - tex-header (this)
%   - R      --> \mathbb{R}
%   - Z      --> \mathbb{Z}
%   - B      --> \mathcal{B}
%   - E      --> \mathcal{E}
%   - M      --> \mathcal{M}
%   - m      --> \mathfrak{m}({#1})
%   - normlp --> \norm{{#1}}_{L^{{#2}}}
