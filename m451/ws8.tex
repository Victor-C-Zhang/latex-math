\documentclass{article}
\usepackage[utf8]{inputenc}

\title{451 - Worksheet 8}
\author{Victor Zhang}
\date{October 2, 2020}

\usepackage[utf8]{inputenc}
\usepackage{amsmath}
\usepackage{amsfonts}
\usepackage{natbib}
\usepackage{graphicx}
% \usepackage{changepage}
\usepackage{amssymb}
\usepackage{xfrac}
% \usepackage{bm}
% \usepackage{empheq}
\usepackage{dirtytalk}

\newcommand{\contra}{\raisebox{\depth}{\#}}

\newenvironment{myindentpar}[1]
  {\begin{list}{}
          {\setlength{\leftmargin}{#1}}
          \item[]
  }
  {\end{list}}

\pagestyle{empty}

\begin{document}

\maketitle
% \begin{center}
% {\huge Econ 482 \hspace{0.5cm} HW 3}\
% {\Large \textbf{Victor Zhang}}\
% {\Large February 18, 2020}
% \end{center}

\section{}
This is exactly $D_5$. Number the points of the pentagon consecutively with 1,2,3,4,5. Denote by $r_i$ the rotation of the pentagon about its center by $2\pi i/5$. Denote by $s_i$ the reflection through the line determined by point $i$ and the center. The symmetries of the pentagon may therefore be enumerated
$$\{r_0, r_1, r_2, r_3, r_4, s_1, s_2, s_3, s_4, s_5\}$$
Note the symmetry of \say{doing nothing} is represented in this notation by $r_0$ $\Box$

\section{}
Let $G$ be the group of rotational symmetries of the cube. Consider the action of this group on an individual vertex $v$ of the cube. The orbit of $v$ has size 8, since the cube can be rotated to put one vertex in any other's place. The stabilizer of $v$ has size 3, since the cube may be rotated along the great diagonal through $v$ in 3 distinct ways. Then by orbit-stabilizer theorem, $|G| = 8 \cdot 3 = 24$. That is, there are 24 rotational symmetries of the cube. If we wish to include reflection in 4 dimensions as being a \say{symmetry}, the stabilizer of $v$ also contains the 3 ways to reflect the cube about a plane through the great diagonal, so $|Stab(v)| = 6$ and thus $|G| = 8\cdot 6 = 48$ $\Box$

\end{document}

% List of tex snippets:
%   - tex-header (this)
%   - R      --> \mathbb{R}
%   - Z      --> \mathbb{Z}
%   - B      --> \mathcal{B}
%   - E      --> \mathcal{E}
%   - M      --> \mathcal{M}
%   - m      --> \mathfrak{m}({#1})
%   - normlp --> \norm{{#1}}_{L^{{#2}}}
