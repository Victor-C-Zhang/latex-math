\documentclass{article}
\usepackage[utf8]{inputenc}
\usepackage[margin=1in]{geometry}

\title{451 - Worksheet 16}
\author{Victor Zhang}
\date{November 27, 2020}

\usepackage[utf8]{inputenc}
\usepackage{amsmath}
\usepackage{amsfonts}
\usepackage{natbib}
\usepackage{graphicx}
% \usepackage{changepage}
\usepackage{amssymb}
\usepackage{xfrac}
% \usepackage{bm}
% \usepackage{empheq}

\newcommand{\contra}{\raisebox{\depth}{\#}}

\newenvironment{myindentpar}[1]
  {\begin{list}{}
          {\setlength{\leftmargin}{#1}}
          \item[]
  }
  {\end{list}}

\pagestyle{empty}

\begin{document}

\maketitle
% \begin{center}
% {\huge Econ 482 \hspace{0.5cm} HW 3}\
% {\Large \textbf{Victor Zhang}}\
% {\Large February 18, 2020}
% \end{center}

\section{}
By linearity of $b$ and the definition of the adjoint,
$$b((T_1+T_2)v,w) = b(T_1v,w) + b(T_2v,w) = b(v, T_1^bw) + b(v, T_2^bw) = b(v, (T_1^b+T_2^b)w)$$
So $(T_1+T_2)^b = T_1^b + T_2^b$.
Similarly,
$$b(T_1T_2v,w) = b(T_2v, T_1^bw) = b(v, T_2^bT_1^bw)$$
So $(T_1T_2)^b = T_2^bT_1^b$ $\Box$

\end{document}

% List of tex snippets:
%   - tex-header (this)
%   - R      --> \mathbb{R}
%   - Z      --> \mathbb{Z}
%   - B      --> \mathcal{B}
%   - E      --> \mathcal{E}
%   - M      --> \mathcal{M}
%   - m      --> \mathfrak{m}({#1})
%   - normlp --> \norm{{#1}}_{L^{{#2}}}
