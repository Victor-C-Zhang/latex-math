\documentclass{article}
\usepackage[utf8]{inputenc}

\title{451 - Weekly 3}
\author{Victor Zhang}
\date{September 24, 2020}

\usepackage[utf8]{inputenc}
\usepackage{amsmath}
\usepackage{amsfonts}
\usepackage{natbib}
\usepackage{graphicx}
% \usepackage{changepage}
\usepackage{amssymb}
\usepackage{xfrac}
% \usepackage{bm}
% \usepackage{empheq}

\newcommand{\contra}{\raisebox{\depth}{\#}}

\newenvironment{myindentpar}[1]
  {\begin{list}{}
          {\setlength{\leftmargin}{#1}}
          \item[]
  }
  {\end{list}}

\pagestyle{empty}

\begin{document}

\maketitle
% \begin{center}
% {\huge Econ 482 \hspace{0.5cm} HW 3}\
% {\Large \textbf{Victor Zhang}}\
% {\Large February 18, 2020}
% \end{center}

\section{}
\subsection{}
For all $x \in X$ we have $\Phi_\alpha(g) \circ \Phi_\alpha(h) = g(hx) = (gh)x = \Phi_\alpha(gh)x$ as desired. Also, $\Phi_\alpha(e)(x) = ex = x$. Thus, $\Phi_\alpha(e) = \mathrm{id}_X$ $\Box$

\subsection{}
Note $\exists$ $g^{-1} \in G$ s.t. $g^{-1}g = e$, that is, $\Phi_\alpha(g^{-1})$ is well-defined and $\Phi_\alpha(g^{-1}) \circ \Phi_\alpha(g) = \Phi_\alpha(g) \circ \Phi_\alpha(g^{-1}) = \mathrm{id}_X$. Thus, $\Phi_\alpha(g)$ is a bijection and $\Phi_\alpha(g^{-1})$ is its inverse $\Box$

\subsection{}
This follows directly from part (1).

\section{}
\subsection{}
Since $\phi$ is a homomorphism, $\phi(g)\phi(h) = \phi(gh)$ for all $g,h$. Notate $\mathcal{A}_\Phi(g,x) = g \circ x$. Then for all $x$, $g \circ (h \circ x) = \phi(g)(\phi(h)(x)) = \phi(gh)(x) = (gh) \circ x$. Furthermore, $e \circ x = \phi(e)(x) = e_{S_X} x = x$. Thus, $\mathcal{A}_\Phi$ represents an action of $G$ on $X$ $\Box$

\subsection{}
By definition, $\Phi_{\mathcal{A}_\phi}(g)(x) = \mathcal{A}_\phi(g,x) = \phi(g)(x)$ for all $g \in G$, $x \in X$. Thus, $\Phi_{\mathcal{A}_\phi} = \phi$. Similarly, $\mathcal{A}_{\Phi_\alpha}(g,x) = \Phi_\alpha(g)(x) = \alpha(g,x)$ for all $g \in G$, $x \in X$ and thus $\mathcal{A}_{\Phi_\alpha} = \alpha$ $\Box$

\section{}
Suppose $gN = Ng$, that is, for $n \in N$, $gn = n'g$ for some $n' \in N$. Then $gng^{-1} = n'gg^{-1} = n' \in N$. Thus, $gNg^{-1} \subseteq N$. Similarly, $ng = gn'$ for some $n' \in N$ and thus $n = gn'g^{-1} \in gNg^{-1}$. This shows $N \subseteq gNg^{-1}$ and the result follows.\\
Now suppose $gNg^{-1} = N$. Then $gng^{-1} = n'$ for $n' \in N$. Thus, $gn = gng^{-1}g = n'g$ and $gN = Ng$ $\Box$

\end{document}

% List of tex snippets:
%   - tex-header (this)
%   - R      --> \mathbb{R}
%   - Z      --> \mathbb{Z}
%   - B      --> \mathcal{B}
%   - E      --> \mathcal{E}
%   - M      --> \mathcal{M}
%   - m      --> \mathfrak{m}({#1})
%   - normlp --> \norm{{#1}}_{L^{{#2}}}
