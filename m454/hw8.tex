\documentclass{article}
\usepackage[utf8]{inputenc}
\usepackage[margin=1in]{geometry}

\title{454 - Homework 8}
\author{Victor Zhang}
\date{November 11, 2021}

\usepackage[utf8]{inputenc}
\usepackage{amsmath}
\usepackage{amsfonts}
\usepackage{natbib}
\usepackage{graphicx}
% \usepackage{changepage}
\usepackage{amssymb}
\usepackage{xfrac}
% \usepackage{bm}
% \usepackage{empheq}
\usepackage{hhline}

\newcommand{\contra}{\raisebox{\depth}{\#}}

\newenvironment{myindentpar}[1]
  {\begin{list}{}
          {
            \setlength{\leftmargin}{#1}
            \setlength{\rightmargin}{#1}
          }
          \item[]
  }
  {\end{list}}

\pagestyle{empty}

\begin{document}

\maketitle
% \begin{center}
% {\huge Econ 482 \hspace{0.5cm} HW 3}\
% {\Large \textbf{Victor Zhang}}\
% {\Large February 18, 2020}
% \end{center}

\section*{10.18}
$$\lambda = \frac{r(k-1)}{v-1} = \frac{15}{7}$$
But $\lambda$ must be integral for a BIBD. Thus, no such BIBD exists with these parameters $\contra$

\section*{10.21}
To build the complement, we simply complement each block:
$$B_1 = \{2,4,5,6\}, B_2 = \{0,3,5,6\}, B_3 = \{0,1,4,6\}, B_4 = \{0,1,2,5\}, B_5 = \{1,2,3,6\}, B_6 = \{0,2,3,4\}, B_7 = \{1,3,4,5\}$$

\section*{10.28}
We generate difference table
\begin{center}
\begin{tabular}{c||c|c|c|c}
- & 0 & 1 & 3 & 9\\
\hhline{=||=|=|=|=}
0 & 0 & 12 & 10 & 4\\
\hline
1 & 1 & 0 & 11 & 5\\
\hline
3 & 3 & 2 & 0 & 7\\
\hline
9 & 9 & 8 & 6 & 0
\end{tabular}
\end{center}
Since every nonzero number appears $\lambda = 1$ times, this is a difference set. We give the corresponding SBIBD in terms of its incidence matrix
\setcounter{MaxMatrixCols}{20}
$$\begin{bmatrix}
1 & 1 & 0 & 1 & 0 & 0 & 0 & 0 & 0 & 1 & 0 & 0 & 0 \\
0 & 1 & 1 & 0 & 1 & 0 & 0 & 0 & 0 & 0 & 1 & 0 & 0 \\
0 & 0 & 1 & 1 & 0 & 1 & 0 & 0 & 0 & 0 & 0 & 1 & 0 \\
0 & 0 & 0 & 1 & 1 & 0 & 1 & 0 & 0 & 0 & 0 & 0 & 1 \\
1 & 0 & 0 & 0 & 1 & 1 & 0 & 1 & 0 & 0 & 0 & 0 & 0 \\
0 & 1 & 0 & 0 & 0 & 1 & 1 & 0 & 1 & 0 & 0 & 0 & 0 \\
0 & 0 & 1 & 0 & 0 & 0 & 1 & 1 & 0 & 1 & 0 & 0 & 0 \\
0 & 0 & 0 & 1 & 0 & 0 & 0 & 1 & 1 & 0 & 1 & 0 & 0 \\
0 & 0 & 0 & 0 & 1 & 0 & 0 & 0 & 1 & 1 & 0 & 1 & 0 \\
0 & 0 & 0 & 0 & 0 & 1 & 0 & 0 & 0 & 1 & 1 & 0 & 1 \\
1 & 0 & 0 & 0 & 0 & 0 & 1 & 0 & 0 & 0 & 1 & 1 & 0 \\
0 & 1 & 0 & 0 & 0 & 0 & 0 & 1 & 0 & 0 & 0 & 1 & 1 \\
1 & 0 & 1 & 0 & 0 & 0 & 0 & 0 & 1 & 0 & 0 & 0 & 1 \\
\end{bmatrix}$$
Here $b = v = 13$, $k = r = 4$, $\lambda = 1$ $\Box$

\section*{10.37}
If we interchange rows $r_i$ and $r_j$, the rows themselves maintain the Latin property and all other rows are unaffected. All columns also maintain the Latin property, since row swapping simply maps the column $C = \{c_1, \dots, c_i, \dots, c_j, \dots c_n\} \to \{c_1, \dots, c_{i-1}, c_{j}, c_{i+1}, \dots, c_{j-1}, c_i, c_{j+1}, \dots, c_n\}$. Thus, the resulting structure is still a Latin square. Interchanging columns similarly maintains the Latin property for both columns and rows $\Box$

\section*{10.38}
$$A^5 = \begin{bmatrix}
0 & 1 & 2 & 3 & 4 & 5 \\
5 & 0 & 1 & 2 & 3 & 4 \\
4 & 5 & 0 & 1 & 2 & 3 \\
3 & 4 & 5 & 0 & 1 & 2 \\
2 & 3 & 4 & 5 & 0 & 1 \\
1 & 2 & 3 & 4 & 5 & 0 \\
\end{bmatrix}$$

\section*{10.41}
$$A^3 = \begin{bmatrix}
0 & 1 & 2 & 3 & 4 & 5 & 6 & 7 \\
3 & 4 & 5 & 6 & 7 & 0 & 1 & 2 \\
6 & 7 & 0 & 1 & 2 & 3 & 4 & 5 \\
1 & 2 & 3 & 4 & 5 & 6 & 7 & 0 \\
4 & 5 & 6 & 7 & 0 & 1 & 2 & 3 \\
7 & 0 & 1 & 2 & 3 & 4 & 5 & 6 \\
2 & 3 & 4 & 5 & 6 & 7 & 0 & 1 \\
5 & 6 & 7 & 0 & 1 & 2 & 3 & 4 \\
\end{bmatrix}$$

\end{document}

% List of tex snippets:
%   - tex-header (this)
%   - R      --> \mathbb{R}
%   - Z      --> \mathbb{Z}
%   - B      --> \mathcal{B}
%   - E      --> \mathcal{E}
%   - M      --> \mathcal{M}
%   - m      --> \mathfrak{m}({#1})
%   - normlp --> \norm{{#1}}_{L^{{#2}}}
