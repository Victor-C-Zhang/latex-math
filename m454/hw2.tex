\documentclass{article}
\usepackage[utf8]{inputenc}
\usepackage[margin=1in]{geometry}

\title{454 - Homework 2}
\author{Victor Zhang}
\date{September 23, 2021}

\usepackage[utf8]{inputenc}
\usepackage{amsmath}
\usepackage{amsfonts}
\usepackage{natbib}
\usepackage{graphicx}
% \usepackage{changepage}
\usepackage{amssymb}
\usepackage{xfrac}
% \usepackage{bm}
% \usepackage{empheq}

\newcommand{\contra}{\raisebox{\depth}{\#}}

\newenvironment{myindentpar}[1]
  {\begin{list}{}
          {\setlength{\leftmargin}{#1}}
          \item[]
  }
  {\end{list}}

\pagestyle{empty}

\begin{document}

\maketitle
% \begin{center}
% {\huge Econ 482 \hspace{0.5cm} HW 3}\
% {\Large \textbf{Victor Zhang}}\
% {\Large February 18, 2020}
% \end{center}

\section*{5}
\begin{equation*}
\begin{split}
(2x - y)^7 = & (2x)^7 + \binom{7}{1}(2x)^6(-y) + \binom{7}{2}(2x)^5(-y)^2 + \binom{7}{3}(2x)^4(-y)^3 + \binom{7}{4}(2x)^3(-y)^4 + \binom{7}{5}(2x)^2(-y)^5\\
& + \binom{7}{6}(2x)(-y)^6 + (-y)^7\\
= & 128x^7 - 448x^6y + 672x^5y^2 - 560x^4y^3 + 280x^3y^4 - 84x^2y^5 + 14xy^6 - y^7 \; \Box
\end{split}
\end{equation*}

\section*{11}
Suppose we have a set of $n$ objects, $n-3$ of which are indistinguishable, and 3 objects $a,b,c$ that are distinguishable from the others and from each other. We count the number of ways to pick $k$ objects out of $n$ such that at least one of $a,b,c$:
The complement of picking at least one of $a,b,c$ is picking none of them. There are $\binom{n-3}{k}$ ways to do so. Thus, there are $\binom{n}{k} - \binom{n-3}{k}$ ways to pick at least one of $a,b,c$.\\
If we want to count constructively, there are 6 different ways we can pick the subset $A$. $A \cap \{a,b,c\}$ can be one of
$$\{a\}, \{a,b\}, \{a,b,c\}, \{b\}, \{b,c\}, \{c\}$$
Note this is equivalent to the three cases
$$\{\text{contains } a\}, \{\text{contains } b \text{ but not } a\}, \{\text{contains } c \text{ but not } b \text{ or } c\}$$
Respectively, there are $\binom{n-1}{k-1}$, $\binom{n-2}{k-1}$, and $\binom{n-3}{k-1}$ ways of picking these cases. Thus there are $\binom{n-1}{k-1} + \binom{n-2}{k-1} + \binom{n-3}{k-1}$ ways of picking $k$-subsets with at least one of $a,b,c$ and by double-counting principle we are done $\Box$

\section*{25}
Suppose we have two sets with $m_1, m_2$ items respectively. We count the ways to pick a total of $n$ objects from both sets combined:\\
Clearly, we may combine the two sets into one large set with $m_1 + m_2$ items and simply pick $n$ from that set. There are $\binom{m_1 + m_2}{n}$ ways to do this.\\
Now suppose we pick $k$ items from the first set. Then to pick $n$ total items, we must pick $n-k$ items from the second set. There are $\binom{m_1}{k} \binom{m_2}{n-k}$ ways to do this. Since we may vary $k$ from 0 to $n$, the total number of ways to pick $n$ items is $\sum\limits_{k=0}^n \binom{m_1}{k}\binom{m_2}{n-k}$. We have double-counted the LHS and RHS of the desired identity, so we are done.\\
Note the identity 5.16 is simply the special case where $m_1 = m_2 = n$:
\begin{gather*}
\sum\limits_{k=0}^n \binom{m_1}{k}\binom{m_2}{n-k} = \binom{m_1 + m_2}{n}\\
\sum\limits_{k=0}^n \binom{n}{k} \binom{n}{n-k} = \binom{n + n}{n}\\
\sum\limits_{k=0}^n \binom{n}{k} \binom{n}{k} = \binom{2n}{n}\\
\sum\limits_{k=0}^n \binom{n}{k}^2 = \binom{2n}{n} \; \Box\\
\end{gather*}

\section*{28}
Suppose we have a set of $2n$ numbered balls, $n$ of which are red and $n$ of which are blue. We count the number of ways to pick $n$ balls if the first ball must be red and we don't care about the numbers on the balls other than the first:\\
There are $n$ ways to choose the first red ball. Then the other $n-1$ balls may be picked at will, for a total of $n \binom{2n-1}{n-1}$ combinations.\\
Now suppose we pick $k \geqslant 1$ red balls. There are $\binom{n}{k}$ ways to pick $k$ red balls and $\binom{n}{n-k} = \binom{n}{k}$ ways to pick (or exclude) the blue balls to get $n$ total. Now out of the $k$ red balls, we may choose any of them to be the first, so there are $k \binom{n}{k}^2$ ways to satisfy the condition given we pick $k$ red balls. Since we may pick anywhere from 1 to $n$ red balls, the total number of ways to satisfy the condition is $\sum\limits_{k=1}^n k\binom{n}{k}^2$. Since we've double-counted the RHS and LHS, we are done $\Box$

\section*{38}
\begin{equation*}
\begin{split}
(x_1 + x_2 + x_3)^4 = & \binom{4}{4,0,0}x_1^4 + \binom{4}{3,1,0}x_1^3x_2 + \binom{4}{3,0,1}x_1^3x_3 + \binom{4}{2,2,0}x_1^2x_2^2 + \binom{4}{2,1,1}x_1^2x_2x_3\\
& + \binom{4}{2,0,2}x_1^2x_3^2 + \binom{4}{1,3,0}x_1x_2^3 + \binom{4}{1,2,1}x_1x_2^2x_3 + \binom{4}{1,1,2}x_1x_2x_3^2 + \binom{4}{1,0,3}x_1x_3^3\\
& + \binom{4}{0,4,0}x_2^4 + \binom{4}{0,3,1}x_2^3x_3 + \binom{4}{0,2,2}x_2^2x_3^2 + \binom{4}{0,1,3}x_2x_3^3 + \binom{4}{0,0,4}x_3^4\\
= & x_1^4 + 4x_1^3x_2 + 4x_1^3x_3 + 6x_1^2x_2^2 + 12x_1^2x_2x_3 + 6x_1^2x_3^2 + 4x_1x_2^3 + 12x_1x_2^2x_3 + 12x_1x_2x_3^2\\
& + 4x_1x_3^3 + x_2^4 + 4x_2^3x_3 + 6x_2^2x_3^2 + 4x_2x_3^3 + x_3^4 \; \Box\\
\end{split}
\end{equation*}

\section*{42}
We take the LHS to be the typical interpretation of the multinomial: permuting a multiset with $n$ total items and $n_i$ copies of item type $i$. We may count the same set of permutations with casework on the first item in the permutation. The first item may be one of $t$ types. If the first item is of type $i$, we may focus on the other $n-1$ items in the permutation. Permuting these is equivalent to permuting $n-1$ items where there are $n_j$ copies of item $j$ for $j \neq i$ and $n_i - 1$ copies of item $i$. In other words, $\binom{n-1}{n_1, \dots n_{i-1}, (n_i - 1), n_{i+1}, \dots n_t}$. The RHS of 5.21 is the sum of all of these cases, so is exactly equal to the LHS, as desired $\Box$

\end{document}

% List of tex snippets:
%   - tex-header (this)
%   - R      --> \mathbb{R}
%   - Z      --> \mathbb{Z}
%   - B      --> \mathcal{B}
%   - E      --> \mathcal{E}
%   - M      --> \mathcal{M}
%   - m      --> \mathfrak{m}({#1})
%   - normlp --> \norm{{#1}}_{L^{{#2}}}
