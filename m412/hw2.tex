\documentclass{article}
\usepackage[utf8]{inputenc}

\title{412 - Homework 2}
\author{Victor Zhang }
\date{February 18, 2020}

\usepackage[utf8]{inputenc}
\usepackage{amsmath}
\usepackage{amsfonts}
\usepackage{natbib}
\usepackage{graphicx}
% \usepackage{changepage}
\usepackage{amssymb}
% \usepackage{bm}
% \usepackage{empheq}

\newcommand{\contra}{\raisebox{\depth}{\#}}

\newenvironment{myindentpar}[1]
  {\begin{list}{}
          {\setlength{\leftmargin}{#1}}
          \item[]
  }
  {\end{list}}

\pagestyle{empty}

\begin{document}

\maketitle
% \begin{center}
% {\huge Econ 482 \hspace{0.5cm} HW 3}\\
% {\Large \textbf{Victor Zhang}}\\
% {\Large February 18, 2020}
% \end{center}

\section{}
\subsection{}
Note every open set in $\mathbb{R}$ is a countable union of open intervals. Denote the set of open sets in $\mathbb{R}$ by $\mathcal{E}$. Then $\mathcal{E} \subseteq \mathcal{M}(\mathcal{E}_1)$, so by the lemma shown in class, the $\sigma$-algebra generated by $\mathcal{E}$, $\mathcal{B}_{\mathbb{R}} \subseteq \mathcal{M}(\mathcal{E}_1)$. But every open interval is an open set, so $\mathcal{M}(\mathcal{E}_1) \subseteq \mathcal{B}_\mathbb{R}$ and we have equality, as desired $\Box$
\subsection{}
Every open interval is a countable union of closed intervals, so every open set is a countable union of closed intervals. Then by the lemma, $\mathcal{B}_\mathbb{R} \subseteq \mathcal{M}(\mathcal{E}_2)$.\\
Now fix some closed interval $[a,b]$ and consider sequence of open intervals $\alpha_n = (a-{\frac{1}{2}}^n,b+{\frac{1}{2}}^n)$. Then $\left(\bigcup\limits^{\infty}{\alpha_n}^c\right)^c = [a,b]$. Then $\mathcal{E}_2 \subseteq \mathcal{B}_\mathbb{R}$ so $\mathcal{M}(\mathcal{E}_2) \subseteq \mathcal{B}_\mathbb{R}$ and we have equality $\Box$
\subsection{}
For any open interval $(a,b)$ we can take sequence $\alpha_n = (a,b-{\frac{1}{2}}^n]$ s.t. $(a,b) = \bigcup\limits^\infty \alpha_n$. Then $\mathcal{B}_\mathbb{R} \subseteq \mathcal{M}(\mathcal{E}_3)$. Now fix arbitrary half-open interval $(a,b]$ and consider sequence $\beta_n = (a,b+{\frac{1}{2}}^n)$. Then $(a,b] = \left(\bigcup\limits^\infty \beta_n^c\right)^c$ and $\mathcal{M}(\mathcal{E}_3) \subseteq \mathcal{B}_\mathbb{R}$ and we have equality.\\
We can show similarly that $\mathcal{B}_\mathbb{R} = \mathcal{M}(\mathcal{E}_4)$ by taking sequences that vary the other endpoint $\Box$
\subsection{}
Note every open ray is a countable union of overlapping open intervals, since we may center each open interval on some rational and $\mathbb{Q}$ is dense on $\mathbb{R}$. So $\mathcal{M}(\mathcal{E}_5) \subseteq \mathcal{B}_\mathbb{R}$ and $\mathcal{M}(\mathcal{E}_6) \subseteq \mathcal{B}_\mathbb{R}$.\\
Note we can form half-open intervals with open rays, for example $(a,b] = \left( (b,\infty) \cup (a,\infty)^c \right)^c$. Thus $\mathcal{E}_3 \subseteq \mathcal{M}(\mathcal{E}_5)$ and so $\mathcal{B}_\mathbb{R} \subseteq \mathcal{M}(\mathcal{E}_3) \subseteq \mathcal{M}(\mathcal{E}_5)$ and similarly $\mathcal{B}_\mathbb{R} \subseteq \mathcal{M}(\mathcal{E}_4) \subseteq \mathcal{M}(\mathcal{E}_6)$ and we are done $\Box$
\subsection{}
Each closed ray is the complement of an open ray, so $\mathcal{M}(\mathcal{E}_8) = \mathcal{M}(\mathcal{E}_5)$ and $\mathcal{M}(\mathcal{E}_7) = \mathcal{M}(\mathcal{E}_6)$ and the result follows $\Box$

\section{}
Note that if $E_\alpha \in \mathcal{M}_\alpha$ then $\pi^{-1}(E_\alpha) \in \mathcal{M}(\mathcal{F}_1)$ since $\{ E \subseteq X_\alpha | \pi^{-1}(E_\alpha) \in \mathcal{M}(\mathcal{F}_1) \}$ is a $\sigma$-algebra on $X_\alpha$ that contains $\mathcal{E}_\alpha$ and thus $\mathcal{M}_\alpha$. So $\bigotimes \mathcal{M}_\alpha \subseteq \mathcal{M}(\mathcal{F}_1)$.\\
But clearly $\mathcal{F}_1 \subseteq \bigotimes \mathcal{M}_\alpha$ since $\mathcal{E}_\alpha \subseteq \mathcal{M}_\alpha$ for all $\alpha \in A$. Then $\mathcal{M}(\mathcal{F}_1) \subseteq \bigotimes \mathcal{M}_\alpha$ and thus $\mathcal{M}(\mathcal{F}_1) = \bigotimes \mathcal{M}_\alpha$ $\Box$\\
If $A$ is countable, note $\pi^{-1}(E_\alpha) = \prod E_\beta$ where $E_\beta = E_\alpha$ if $\beta = \alpha$, and $E_\beta = X_\beta$ otherwise. Then $\bigotimes \mathcal{M}_\alpha \subseteq \mathcal{M}(\mathcal{F}_2)$. But note $\prod E_\alpha = \bigcap \pi^{-1}(E_\alpha) \in \bigotimes \mathcal{M}_\alpha$, so $\mathcal{M}(\mathcal{F}_2) \subseteq \bigotimes \mathcal{M}_\alpha$ and we are done $\Box$

\section{}
Note $\bigotimes \mathcal{B}_{X_j}$ is generated by $\mathcal{E} = \{ \prod\limits_{i=1}^n E_i | E_i \in \mathcal{E}_i \}$, where $\mathcal{E}_i$ is the set of open sets in $X_i$. Choose open $E_1, E_2, \dots E_n$ and note for every point $p_i \in E_i$ we can find some neighborhood $N_{r_i} (p_i) \subseteq E_i$. Then if we put $r = \min \{r_i\}$, for the corresponding point $p = (p_1,p_2,\dots p_n)$, $N_r(p) \subseteq \prod\limits_{i=0}^n E_i$. So $\prod\limits_{i=1}^n E_i$ is open in $X$ and so is an element of $\mathcal{B}_X$. Thus $\bigotimes \mathcal{B}_{X_j} \subset \mathcal{B}_X$ $\Box$\\
Note if the $X_j$'s are separable, for each $j$ we may pick a countable, dense subset $A_j \subseteq X_j$ and $\mathcal{E}_j$ the set of open balls of rational radius centered around points in $A_j$. Then each open set in $X_j$ is a countable union of elements in $\mathcal{E}_j$ and every open ball in $X$ is a product of open balls in each $X_j$. We may simiarly pick a dense sbset of $X$ and see that every open set in $X$ is the countable union of open balls in $X$ and thus the product of open balls in each $X_j$. Then $\mathcal{B}_X \subseteq \bigotimes \mathcal{B}_{X_j}$ and we have equality, as desired $\Box$

\section{}
Note if finite disjoint union $E = E_1 \cup E_2 \cup \dots \cup E_n \in \mathcal{A}$, then $E^c = E_1^c \cap E_2^c \cap \dots \cap E_n^c$. Each $E_i^c$ is a finite disjoint union, and $E^c$ is the intersection of finite disjoint unions, so is a finite disjoint union. Then $E^c \in \mathcal{A}$.\\
Now suppose $E,F \in \mathcal{A}$ and suppose $E = E_1 \cup \dots \cup E_n$. WLOG we may union each disjoint set in $F$ iteratively to get $E\cup F$. So it suffices to show $E\cup F \in \mathcal{A}$ when $F$ contains just one disjoint set:\\
If $F\cap E_i = \emptyset$ for all $i$ then we are done. Otherwise, suppose $F \cap E_k \neq \emptyset$. Then $E_k = E_k \cap F + E_k \cap F^c$, so $F \cup E$ = $E \cap F^c + F \in \mathcal{A}$ $\Box$ 

\section{}
\subsection{}
Note that $X$ must be infinite. Otherwise $\textrm{card}\, \mathcal{M} \leq 2^{\mathrm{card}\,X}$ would be finite. Let $\mathcal{A}$ be the set of disjoint unions in $\mathcal{M}$. The cardinality of $\mathcal{A}$ must be infinite:
\begin{myindentpar}{2em}
    Otherwise we can (partially) order the elements in $\mathcal{A}$ by set inclusion and get some "largest" element $E$. If this $E$ an infinite disjoint union, $\textrm{card}\,\mathcal{A}$ cannot be finite. If $E$ is finite, then there must be some $E^{'} \in \mathcal{M}$ s.t. $E^{'} \nsubseteq E$ since $\mathcal{M}$ is infinite. But then we can take $F = E^{'} \backslash E$ s.t. $E \cup F$ is a disjoint union larger than $E$ $\contra$
\end{myindentpar}
So thus $\textrm{card}\,\mathcal{A}$ is infinite and we can then pick an infinite sequence of disjoint sets in $\mathcal{M}$ $\Box$
\subsection{}
Part 5(a) implies that there is an element $E \in \mathcal{A} \subseteq \mathcal{M}$ with infinite cardinality. Since we may generate a power set on the individual disjoint elements in $E$ (which is wholly contained in $\mathcal{M}$), it follows that $\textrm{card}\,\mathcal{M} \geq 2^{\aleph_0} = \mathfrak{c}$ $\Box$

\section{}
Note $(A\cap S) \cup (B \cap S^c) = \left( A^c \cup S^c \right)^c \cup \left( B^c \cup S\right)^c \in \sigma (\mathcal{A} \cup \{S\})$ for all $A,B \in \mathcal{A}$. So $M = \{ (A\cap S) \cup (B \cap S^c) \} \subseteq \sigma (\mathcal{A} \cup \{S\})$. Now we show $M$ is a $\sigma$-algebra as well:\\
For countable $A_n, B_n \in \mathcal{A}$, $\bigcup\limits_{n \in \mathbb{N}} (A_n\cap S) \cup (B_n \cap S^c) = [\left( \bigcup\limits_{n \in \mathbb{N}} A_n \right) \cap S] \cup [\left( \bigcup\limits_{n \in \mathbb{N}} B_n \right) \cap S^c] \in M$ since $\bigcup\limits_{n \in \mathbb{N}} A_n, \bigcup\limits_{n \in \mathbb{N}} B_n \in \mathcal{A}$. So $M$ is closed under countable union.
\begin{equation*}
\begin{split}
    \left( (A\cap S) \cup (B \cap S^c) \right)^c &= (A\cap S)^c \cap (B \cap S^c)^c\\
    &= (A^c \cup S^c) \cap (B^c \cup S)\\
    &= (A^c \cap B^c) \cup [(B^c \cap S) \cup (A^c \cap S^c)]\\
\end{split}
\end{equation*}
Note for all $A \in \mathcal{A}$, $(A\cap S) \cup (A \cap S^c) = A \in M$. Since $(A^c \cap B^c) \in \mathcal{A}$, the above complement is a union of elements in $M$ so $\in M$. Then $M$ is a $\sigma$-algebra containing $\mathcal{A} \cup \{S\}$ which is a subset of $\sigma (\mathcal{A} \cup \{S\})$. Since $\sigma (\mathcal{A} \cup \{S\})$ is the smallest $\sigma$-algebra containing $\mathcal{A}\cup \{S\}$, $M = \sigma (\mathcal{A} \cup \{S\})$ $\Box$

\section{}
Denote the union $\mathcal{A} = \bigcup\limits_{\mathcal{F}} \mathcal{M}(\mathcal{F})$. First show $\mathcal{A}$ is a $\sigma$-algebra:\\
\begin{myindentpar}{2em}
    Note that if $E \in \mathcal{A}$ then there must exist some $\mathcal{F} \in \mathcal{E}$ s.t. $E \in \mathcal{M}(\mathcal{F})$. Then clearly $E^c \in \mathcal{M}(\mathcal{F}) \subseteq \mathcal{A}$ so $\mathcal{A}$ is closed under complements.\\
    For countable sequence $E_n \in \mathcal{A}$ we can find $\mathcal{F}_1, \mathcal{F}_2, \dots$ s.t. $E_n \in \mathcal{M}(\mathcal{F}_n)$. Then $\bigcup E_n \in \bigcup \mathcal{M}(\mathcal{F}) \subseteq \mathcal{A}$. So $\mathcal{A}$ is closed under countable unions and is thus a $\sigma$-algebra.
\end{myindentpar}
Clearly, $\mathcal{F} \subseteq \mathcal{E}$ for all countable $\mathcal{F}$, so $\mathcal{M}(\mathcal{F}) \subseteq \mathcal{M}(\mathcal{E})$ and thus $\mathcal{A} \subseteq \mathcal{M}(\mathcal{E})$. But then for any element $E \in \mathcal{E}$ we can put $\mathcal{F} = \{E\}$ s.t. $E \in \mathcal{M}(\mathcal{F}) \subseteq \mathcal{A}$. So $\mathcal{M}(\mathcal{E}) \subseteq \mathcal{A}$ and we have equality, as desired $\Box$

\section{}
Let $F_1 = E_1$, $F_n = E_1 \backslash E_n$ for $n>1$. Then put $F = \bigcup\limits_{n=1}^\infty F_n$ and note $E_1 = \bigcup\limits_{n=1}^\infty F_n + \bigcap\limits_{n=1}^\infty E_n$. Moreover, $F \cap \left(\bigcap\limits_{n = 1}^\infty E_n \right) = \emptyset$, so $\mu (E_1) = \mu (F) + \mu (\bigcap\limits_{n=1}^\infty E_n)$. Note we can create disjoint $G_n = F_n \backslash F_{n-1}$ s.t. $F = \bigcup\limits^\infty G_n$. But $\mu (G_n) = \mu (F_n) - \mu (F_{n-1})$ so by telescoping, $\mu (\bigcup\limits_{n = 1}^j G_n)= \mu (F_n)$. $\lim\limits_{n\rightarrow\infty} \mu (\bigcup\limits_{n = 1}^\infty G_n) = \mu (F) = \lim\limits_{n\rightarrow\infty} \mu (E_1 \backslash E_n) = \lim\limits_{n\rightarrow\infty} \mu (E_1) - \mu (E_n)$. The result follows $\Box$\\
Note that if we drop the finiteness condition, we can pick $X = \mathbb{N}$, $\mathcal{M} = \mathcal{P}(\mathbb{N})$, and $\mu$ to be the counting measure. Then the proposition fails if we consider $E_i = \{j \geq i\}$ since all $\mu (E_i) = \infty$ but $\bigcap\limits^\infty E_i = \emptyset$

\section{}
By nonatomicity we can always find some monotonically decreasing sequence $\{B_n\}$ s.t. $A \supset B_1 \supset B_2 \supset \dots$ and $\mu (A) > \mu (B_1) > \mu (B_2) > \dots$. Suppose there is some $\epsilon > 0$ s.t. there is no $B_n$ s.t. $\mu (B_n) < \epsilon$. Then by monotone convergence, there must be some $N$ s.t. for $n>N$, $\epsilon < \mu (B_n) < \frac{3}{2} \epsilon$. Then for any $n > N$ we can choose new $B_{n+1}^{'} = B_n \backslash B_{n+1}$ s.t. $\mu (B_{n+1}^{'}) < \frac{1}{2} \epsilon$ $\contra$

\section{}
The following is a counterexample to the current problem statement:\\
Let $X = \{ 0,1 \}$, $\mathcal{M} = \{ \emptyset, \{0\}, \{1\}, \{0,1\} \}$, $\mu (\{0\}) = \mu (\{1\}) = \frac{1}{2}$, $\mu (\{0,1\}) = 1$. Then $(X,\mathcal{M},\mu$ is a finite measure space. But for $A = \{0\}$, $\nexists \, B \subset A$ s.t. $\mu (B) = \frac{1}{3}$.\\
However, if we are to assume this problem follows from problem 9 and we include the condition of nonatomicity, the solution is tractable:\\
By problem 9, we may find $B_1 \subset A$ s.t. $\mu (B_1) < \mu (A) - \theta$. Put $H_1 = A \backslash B_1$ so that $\mu (H_1) = \mu (A) - \mu (B_1) > \theta$. Now iteratively pick $B_n \subset H_{n-1}$ s.t. $\mu (B_n) < \mu (H_{n-1}) - \theta$ and $H_n = H_{n-1} \backslash B_n$. Let $B = \bigcap\limits^\infty H_n$. Note that $A \supseteq H_1 \supseteq H_2 \supseteq \dots$ and $\lim\limits_{n\rightarrow\infty} \mu (H_n) = \theta$, so by problem 8, $\mu (B) = \theta$ $\Box$

\end{document}