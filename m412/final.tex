\documentclass{article}
\usepackage[utf8]{inputenc}

\title{412 - Final Exam}
\author{Victor Zhang}
\date{May 11, 2020}

\usepackage[utf8]{inputenc}
\usepackage{amsmath}
\usepackage{amsfonts}
\usepackage{natbib}
\usepackage{graphicx}
% \usepackage{changepage}
\usepackage{amssymb}
% \usepackage{bm}
% \usepackage{empheq}

\newcommand{\contra}{\raisebox{\depth}{\#}}

\newenvironment{myindentpar}[1]
  {\begin{list}{}
          {\setlength{\leftmargin}{#1}}
          \item[]
  }
  {\end{list}}

\pagestyle{empty}

\begin{document}

\maketitle
% \begin{center}
% {\huge Econ 482 \hspace{0.5cm} HW 3}\
% {\Large \textbf{Victor Zhang}}\
% {\Large February 18, 2020}
% \end{center}

$\Psi$ is a k-chain of class $C^2$ so we may write $\Psi$ as a collection $\Phi_1, \dots \Phi_n$ of oriented $C^2$ k-simplexes. By definition,
$$\int_\Psi \omega = \sum^n_{i=1} \int_{\Phi_i} \omega$$
and
$$\partial \Psi = \sum^n \partial \Phi_i$$
so it suffices to show
$$\int_\Phi \textrm{d}\omega = \int_{\partial\Phi}\omega$$
for arbitrary oriented $C^2$ k-simplexes $\Phi$ in $V$.\\
First we show this for the standard oriented affine k-simplex
$$\sigma = [0,\textbf{e}_1,\dots \textbf{e}_k]$$
with parameter domain $Q^k$. Let $E$ be an open set that contains $Q^k$. Then since $\omega$ is a (k-1)-form of class $C^1$ in $E$ it suffices to show
$$\int_\sigma \textrm{d}\lambda = \int_{\partial\sigma}\lambda$$
where $\lambda$ is a $C^1$ (k-1)-form given by
$$\lambda = f(\textbf{x})\textrm{d}_{\textbf{\textit{I}}} \textrm{, where } \textbf{\textit{I}} = \{1,\dots r-1,r+1,\dots k\} \textrm{ for some } 1\leq r \leq k$$
since any $\omega$ can be represented as the sum of some $\lambda$ terms, and the operations differentiation $\textrm{d}\omega$ and integration $\int_\sigma \omega$ are both linear.\\
Fix some $\lambda, f, r$. By definition, the boundary
$$\partial\sigma = \sum^k_{j=0} (-1)^j[\textbf{e}_0, \dots \textbf{e}_{j-1}, \textbf{e}_{j+1}, \dots \textbf{e}_k] = [\textbf{e}_1, \dots \textbf{e}_k] + \sum_{j=1}^k (-1)^j[\textbf{e}_0, \dots \textbf{e}_{j-1}, \textbf{e}_{j+1}, \dots \textbf{e}_k]$$
Define the following maps from $Q^{k-1} \to Q^k$
\begin{equation*}
    \delta (u_1,\dots u_{k-1}) = (1-\sum^{k-1}_{i=1} x_i, x_1, \dots x_{k-1})
\end{equation*}

\begin{equation}
\begin{split}
    \epsilon_0 (u_1,\dots u_{k-1}) &= (x_1, \dots x_{r-1}, 1-\sum^{k-1}_{i=1} x_i, x_{r+1}, \dots x_{k-1}) \\
    \epsilon_i (u_1,\dots u_{k-1}) &= (x_1, \dots x_{i-1}, 0, x_{i+1}, \dots x_{k-1}), \; \; 1\leq i\leq k
\end{split}
\end{equation}
and note
$$\partial \sigma = \delta + \sum_{j=1}^k(-1)^j\epsilon_j = (-1)^{r-1}\epsilon_0 + \sum_{j=1}^k(-1)^j\epsilon_j$$
with the second equality achieved by sequentially right-swapping the first element of $\delta$ and applying the appropriate number of negations.\\
Now if we restrict the maps to $Q^{k-1} \to Q^{k-1}$ given by
$$(u_1,\dots u_{k-1}) \to (x_1,\dots x_{r-1},x_{r+1},\dots x_k)$$
we note that $\epsilon_0 = \epsilon_r = I$ so the Jacobian $J_0 = J_r = 1$. Further, all other $\epsilon_i$ have the $i$th row 0, so $J_i = 0$. Then since integration is linear on a chain,

\begin{equation}
\begin{split}
    \int_{\partial\sigma} \lambda &= (-1)^{r-1}\int_{\epsilon_0} \lambda + \sum^k_{i=1}(-1)^i\int_{\epsilon_i} \lambda\\
    &= (-1)^{r-1}(\int_{\epsilon_0}\lambda - \int_{\epsilon_r} \lambda )\\
    &= (-1)^{r-1} \int\limits_{Q^{k-1}} \left[f(\epsilon_0(\textbf{u})) - f(\epsilon_r(\textbf{u})) \right] \textrm{ d}\textbf{u}
\end{split}
\end{equation}
The last equality utilizes the theorem 10.24 from Rudin:
\begin{myindentpar}{0em}
\textbf{10.24 \; Theorem.} Suppose $\omega$ is a k-form in an open set $E \subset R^n$, $\Phi$ is a k-surface in $E$ with parameter domain $D \subset R^k$, and $\Delta$ is the k-surface in $R^k$ with parameter domain $D$ defined by $\Delta(\textbf{u}) = \textbf{u}(\textbf{u} \in D)$. Then 
$$\int_\Phi \omega = \int_\Delta \omega_\Phi$$
\end{myindentpar}
On the other hand, by the definition of differentiation of k-forms and anti-commutativity of the wedge operator,
\begin{equation*}
\begin{split}
    \textrm{d}\lambda &= (D_r f)(\textbf{x}) \textrm{d}x_r \wedge \textrm{d}x_1 \wedge \dots \wedge \textrm{d}x_{r-1} \wedge \textrm{d}x_{r+1} \wedge \dots \wedge \textrm{d}x_k\\
    &= (-1)^{r-1} D_r f)(\textbf{x}) \textrm{d}x_1 \wedge \dots \wedge \textrm{d}x_{k}
\end{split}
\end{equation*}
Then
$$\int_\sigma \textrm{d}\lambda = (-1)^{r-1} \int_{Q^{k}} (D_r f)(\textbf{x}) \textrm{ d}\textbf{x}$$
We may evaluate this integral from the $r$th coordinate first, with bounds
$$[0,1-x_1-\dots -x_{r-1} - x_{r+1} - \dots - x_k]$$
If we put $(u_1,\dots u_{k-1}) = (x_1, \dots x_{r-1},x_{r+1},\dots x_k)$ we may apply equations (1) to transform the bounds into $[\epsilon_r(\textbf{u}), \epsilon_0(\textbf{u})]$. Then by simple application of the Fundamental Theorem of Calculus, we get exactly the integral (2). Thus we have shown
\begin{equation}
\int_\sigma \textrm{d}\lambda =\int_{\partial\sigma}\lambda
\end{equation}

To finish, we note that any oriented k-simplex $\Phi$ of class $C^2$ in $V$ can be written as $T\circ \bar{\sigma}$ for a $C^2$ mapping $T: E\to V$ and $\bar{\sigma}$ an oriented affine k-simplex with parameter domain $Q^k$. But $\Phi$ is defined on $Q^k$ so $Q^k \subseteq E$, and in fact $\Phi = T\circ \sigma$ for some $C^2$ mapping $T$.\\
Now we appeal to theorems 10.22 and 10.25 from Rudin about $C^1$ k-forms:
\begin{myindentpar}{0em}
\textbf{10.22c \, Thereom. } For $C^1$ k-form $\omega$ and $C^2$ mapping $T: E\to V$ (with $E$ and $V$ defined as in the main proof),
$$\textrm{d}(\omega_T) = (\textrm{d}\omega)_T$$
\end{myindentpar}

\begin{myindentpar}{0em}
\textbf{10.25 \; Theorem. } Suppose $T: E\to V$ is a $C^1$ mapping, $\Phi$ is a k-surface in $E$ and $\omega$ is a k-form in $V$. Then
$$\int_{T\Phi} \omega = \int_\Phi \omega_T$$

Applying these, we see that
$$\int_\Phi \textrm{d}\omega = \int_{T\sigma} \textrm{d}\omega = \int_\sigma (\textrm{d}\omega)_T = \int_\sigma \textrm{d}(\omega_T)$$
and similarly
$$\int_{\partial\Phi} \omega = \int_{\partial\sigma} \omega_T$$.

Since $\omega_T$ is a (k-1)-form of class $C^1$ in $E$, by (3)
$$\int_\Phi \textrm{d}\omega = \int_\sigma \textrm{d}(\omega_T) = \int_{\partial\sigma} \omega_T = \int_{\partial\Phi} \omega$$
and we are done $\Box$
\end{myindentpar}
\end{document}