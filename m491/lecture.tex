\documentclass{article}
\usepackage[utf8]{inputenc}
\usepackage[margin=1in]{geometry}

\title{491 - Maximum Principle}
\author{Victor Zhang}
\date{Oct. 6, 2021}

\usepackage[utf8]{inputenc}
\usepackage{amsmath}
\usepackage{amsfonts}
\usepackage{amsthm}
\usepackage{natbib}
\usepackage{graphicx}
% \usepackage{changepage}
\usepackage{amssymb}
\usepackage{xfrac}
% \usepackage{bm}
% \usepackage{empheq}
\usepackage{dirtytalk}

\newcommand{\contra}{\raisebox{\depth}{\#}}

\newenvironment{myindentpar}[1]
  {\begin{list}{}
          {\setlength{\leftmargin}{#1}}
          \item[]
  }
  {\end{list}}

\newtheorem{theorem}{Theorem}
\newtheorem{definition}{Defn}
\newtheorem{lemma}[theorem]{Lemma}

\pagestyle{empty}

\begin{document}

\maketitle
% \begin{center}
% {\huge Econ 482 \hspace{0.5cm} HW 3}\
% {\Large \textbf{Victor Zhang}}\
% {\Large February 18, 2020}
% \end{center}

\section{Introduction}
\begin{theorem}[Maximum Principle]
Let $f$ be a holomorphic function on a domain $U$. If there is a point $p$ s.t.
$$|f(p)| \geq |f(z)|$$
for all $z \in U$ then $f$ is \textbf{constant}.
\end{theorem}

\begin{theorem}
A holomorphic function on a domain achieves maximum only on its boundary.
\end{theorem}

\begin{definition}[Real Analogue]
If $f$ is continuously differentiable on open interval $I$; if $\exists$ $p \in I$ s.t.
$$|f(p)| \geq |f(x)|$$
for all $x \in I$, then $X$ is a local extremum.
\end{definition}

\begin{definition}
A set (or topological space) is \textbf{connected} if it cannot be separated into two disjoint nonempty open sets.
\end{definition}

For example, $[1,2] \in \mathbb{R}$ is connected. $\mathbb{R}^2 \setminus \{y = 0\}$ is NOT connected.

\begin{definition}
A set $S$ is \textbf{path-connected} if $\exists$ $f : [0,1] \to S$ for all $x,y \in S$ s.t. $f(0) = x, f(1) = y$, $f$ is continuous, and $f(x) \in S$ for $0 \leq x \leq 1$.
\end{definition}

\begin{lemma}
In $\mathbb{R}^n, \mathbb{C}^n$ connected is equivalent to path-connected.
\end{lemma}

\begin{definition}
A \textbf{domain} is a connected open set (in $\mathbb{C}$).
\end{definition}

For example, $\{i > 0\}, \{|z| < 1\}, \mathbb{C}$ are domains.


\section{Preliminaries}
Recall: If $f$ is holomorphic on domain $U$ then for all $p \in U$
$$\lim_{z \to p} \frac{f(z) - f(p)}{z - p}$$
exists and we call the limit $f'(p)$ or $\frac{\partial f}{\partial z} |_p$.

\begin{theorem}[Cauchy-Riemann]
If $f$ is holomorphic, $f = u + iv$
$$\frac{\partial u}{\partial x} = \frac{\partial v}{\partial y}, \frac{\partial u}{\partial y} = \frac{\partial v}{\partial x}, f' = \frac{\partial x}{\partial x} + i \frac{\partial v}{\partial x}$$
\end{theorem}

\begin{theorem}[Cauchy Integral Formula]
Let $f$ be holomorphic on disk $\bar{D}(p,r)$. Then
$$f(p) = \frac{1}{2\pi i} \oint\limits_{\partial D} \frac{f(z)}{z-p} \;\mathrm{d}z$$
\end{theorem}

Parametrize $z = p + re^{it}$ for $0 \leq t \leq 2\pi$.\\
Recall from calculus:\\
If $U \subseteq \mathbb{C}$ open subset, $L \subset U$ is a curve of finite length parametrized by cont. diff. $\gamma : [a,b] \to L$ given as $t \mapsto x(t) + iy(t)$,
$$\oint_L f(z) \;\mathrm{d}z = \int_a^b f(\gamma(t))\gamma'(t) \;\mathrm{d}t$$
where $\gamma'$ is the complex derivative.

Then we have
\begin{equation*}
\begin{split}
\frac{1}{2\pi i} \oint\limits_{\partial D} \frac{f(z)}{z-p} \;\mathrm{d}z &= \frac{1}{2\pi i} \oint\limits_0^{2\pi} \frac{f(z)}{z-p} \;\mathrm{d}z\\
&= \frac{1}{2\pi} \int\limits_0^{2\pi} \frac{f(p + re^{it})}{re^{it}} rie^{it} \;\mathrm{d}t\\
&= \frac{1}{2\pi} \int\limits_0^{2\pi} f(p + re^{it}) \;\mathrm{d}t
\end{split}
\end{equation*}

\begin{theorem}
Let $f$ be holomorphic on domain $U$. If there exists point $p \in U$ s.t.
$$|f(p)| \geq |f(z)|$$
for all $z \in U$ then $f$ is constant.
\end{theorem}

\begin{proof}[Proof: Topology]
WLOG suppose $f(p)$ is real and nonnegative. Let $S = \{z \in U \;:\; f(z) = f(p)\}$. (Want $S = U$). Obviously $S$ is nonempty, since $p \in S$. Since $f$ is continous, $S$ is closed in $U$, \say{trivially}.

Suppose not. That is, $\exists$ limit point $x_0 \in U$ s.t. $f(x_0) \neq f(p)$. By definition of continuity, $\forall \varepsilon > 0$, $\exists \delta > 0$ s.t. if $z \in U$ and $z \in D(x_0, \delta)$ then $|f(z) - f(x_0)| < \varepsilon$. By definition of limit point $\forall \delta > 0$, $\exists z \in S$ s.t. $z \in D(x_0, \delta)$ $\contra$

To see that $S$ is open, take $w \in S$ and suppose $D(w, r) \subseteq U$. Then for $0 < r' < r$
\begin{equation*}
\begin{split}
f(p) = f(w) = |f(w)| &= \left| \frac{1}{2\pi} \int\limits_0^{2\pi} f(w + re^{it}) \;\mathrm{d}t \right|\\
&\leq \frac{1}{2\pi} \int\limits_0^{2\pi} \left| f(w+re^{it}) \right| \;\mathrm{d}t\\
&\leq \frac{1}{2\pi} \int\limits_0^{2\pi} \left| f(p) \right| \;\mathrm{d}t = f(p)
\end{split}
\end{equation*}
So $S$ is non-empty, closed in $U$, and open. Since $U$ is connected, then in fact $S = U$
\end{proof}

\begin{proof}[Proof: Analysis]
Suppose $f$ is holomorphic. We find power series expansion of $f$ around $p$.
$$f(z) = \sum_{n \in \mathbb{N}} a_n (z-p)^n, a_0 = f(p)$$
Let $z = p + re^{it}$. Then $f(z) = \sum_{n \in \mathbb{N}} a_n r^n e^{int}$
\begin{equation*}
\begin{split}
|f(z)|^2 = f(z) \overline{f(z)} &= \left( \sum_n a_nr^ne^{int} \right) \overline{\left( \sum_m a_mr^me^{imt} \right)}\\
&= \left( \sum_n a_nr^ne^{int} \right) \left( \sum_m \overline{a_m}r^me^{-imt} \right)\\
&= \sum_{n,m} a_n \overline{a_m} r^{n + m} e^{i(n-m)t}
\end{split}
\end{equation*}

\begin{equation*}
\begin{split}
\frac{1}{2\pi} \int_\gamma |f(z)|^2 &= \frac{1}{2\pi} \int_0^{2\pi} a_n \overline{a_m} r^{n + m} e^{i(n-m)t} \;\mathrm{d}t\\
&= \sum_n |a_n|^2 r^{2n}\\
&= |a_0|^2 + \sum_{n=1}^\infty |a_n|^2 r^{2n}\\
\end{split}
\end{equation*}
where the integral is evaluated by
$$\int a_n \overline{a_m} r^{n + m} e^{i(n-m)t} \;\mathrm{d}t = a_n \overline{a_m} r^{n + m} \frac{e^{i(n-m)t}}{i(n-m)}\Big\vert_0^{2\pi} = 0$$
$$\int a_n \overline{a_n} r^{2n} e^0 \;\mathrm{d}t = |a_n|^2 r^{2n} \cdot 2\pi$$

\begin{equation*}
\begin{split}
|f(p)| \geq |f(z)| \to & \frac{1}{2\pi} \int_0^{2\pi} |f(z)|^2 \;\mathrm{d}z\\
&\leq \frac{1}{2\pi} \int_0^{2\pi} |f(p)|^2 \;\mathrm{d}z\\
&= \frac{1}{2\pi} \int_0^{2\pi} |a_0|^2 \;\mathrm{d}z = |a_0|^2
\end{split}
\end{equation*}
So then $\sum_{n=1}^\infty |a_n|^2 r^{2n} = 0$, $a_n = 0$.
$$f(z) = \sum_n a_nr^ne^{it} = a_0 = f(p)$$
\end{proof}

\section{Applications}
\begin{definition}[Harmonic Functions]
Let $U \in \mathbb{R}^2$ be an open subset, $f : U \to \mathbb{R}$ given $(x,y) \mapsto u$ is \textbf{harmonic} if it satisfies the Laplace equations
$$\frac{\partial^2 u}{\partial x^2} + \frac{\partial^2 u}{\partial y^2} = 0$$
\end{definition}

\begin{lemma}
Let $u$ be a nonconstant harmonic function on domain $U$. Then it does not achieve maximum on $U$.
\end{lemma}
\begin{proof}
If $u$ is nonconstant, then locally it is nonconstant. We can find a \say{harmonic conjugate}, a harmonic $v$ s.t. $u,v$ satisfy Cauchy-Riemann. Let $f = u + iv$ be holomorphic. Then $e^f$ is holomorphic,
$$|e^f| = |e^u|$$
and by the maximum principle we are done.
\end{proof}

As an example, take $f(x,y) = e^x \sin y$.
\begin{gather*}
\frac{\partial^2 f}{\partial x^2} = e^x \sin y, \frac{\partial^2 f}{\partial y^2} = - e^x \sin y, \frac{\partial^2 f}{\partial x^2} + \frac{\partial^2 f}{\partial y^2} = 0\\
\frac{\partial f}{\partial x} = e^x \sin y = 0 \to y = k\pi, k \in \mathbb{Z}\\
\frac{\partial f}{\partial y} = e^x \cos y = 0 \to y = k\pi + \frac{\pi}{2}, k \in \mathbb{Z}
\end{gather*}
$f$ has no critical points, so no local extrema $\Box$

\end{document}

% List of tex snippets:
%   - tex-header (this)
%   - R      --> \mathbb{R}
%   - Z      --> \mathbb{Z}
%   - B      --> \mathcal{B}
%   - E      --> \mathcal{E}
%   - M      --> \mathcal{M}
%   - m      --> \mathfrak{m}({#1})
%   - normlp --> \norm{{#1}}_{L^{{#2}}}
