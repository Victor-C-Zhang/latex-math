\documentclass{article}
\usepackage[utf8]{inputenc}
\usepackage[margin=1in]{geometry}

\title{480 - Homework 8}
\author{Victor Zhang}
\date{December 13, 2021}

\usepackage[utf8]{inputenc}
\usepackage{amsmath}
\usepackage{amsfonts}
\usepackage{natbib}
\usepackage{graphicx}
% \usepackage{changepage}
\usepackage{amssymb}
\usepackage{xfrac}
% \usepackage{bm}
% \usepackage{empheq}

\newcommand{\contra}{\raisebox{\depth}{\#}}

\newenvironment{myindentpar}[1]
  {\begin{list}{}
          {
            \setlength{\leftmargin}{#1}
            \setlength{\rightmargin}{#1}
          }
          \item[]
  }
  {\end{list}}

\pagestyle{empty}

\begin{document}

\maketitle
% \begin{center}
% {\huge Econ 482 \hspace{0.5cm} HW 3}\
% {\Large \textbf{Victor Zhang}}\
% {\Large February 18, 2020}
% \end{center}

\section{}
The payoff matrix is given
\begin{center}
\begin{tabular}{c c c}
 & $l$ & $r$\\
$U$ & 13,6 & 18,1\\
$D$ & 10,5 & 17,7
\end{tabular}
\end{center}

\subsection{}
Player 1 should always play $U$ over $D$ since $U$ strictly dominates $D$ in payoff.

\subsection{}
The Nash equilibrium is thus $(U,l)$.

\section{}
\subsection{}
For each player, the marginal benefit of contributing $\;\mathrm{d}z$ is
$$\;\mathrm{d}u = -\frac{1}{2}\;\mathrm{d}z + 3\alpha(\frac{1}{2}\;\mathrm{d}z) = \frac{3\alpha - 1}{2}\;\mathrm{d}z$$
So long as $\frac{\mathrm{d}u}{\mathrm{d}z}$ is positive, each player will contribute the maximal amount of 10 dollars. This happens when $\alpha > \tfrac{1}{3}$. When $\frac{\mathrm{d}u}{\mathrm{d}z} = 0$, each player is indifferent to contributing more or less. So a NE with maximal contribution is possible when $\alpha \geq \tfrac{1}{3}$ $\Box$

\subsection{}
Note that marginal benefit of contributing is not dependent of amount contributed. So when a player's $\alpha < \tfrac{1}{3}$ they will contribute nothing, when $\alpha > \tfrac{1}{3}$ they will contribute everything, and when $\alpha = \tfrac{1}{3}$ they are indifferent to any contribution level. Thus, players 1 and 2 will contribute all 10 dollars and players 3 and 4 will contribute nothing. Then the monetary payoffs are (10, 10, 20, 20) and the utilities are (28.75, 28.75, 30, 20). So player 3 enjoys the highest equilibrium utility $\Box$

\section{}
\subsection{}
Note the responder's payoff to any split is constant at 1. So the responder will always accept the proposal. The proposer then only needs to worry about maximizing their own utility function $u_p(x) = \sqrt{x} + \frac{1}{2}\sqrt{1-x}$, which happens at $x = \frac{4}{5}$. So the SPE is $(\frac{4}{5}, \frac{1}{5})$ $\Box$

\subsection{}
From an algebraic calculation, the responder enjoys positive utility when $y > \frac{4}{13}$ so will accept splits with $y \geq \frac{4}{13}$, in other words $x \leq \frac{9}{13}$. Given this restriction, the proposer should propose $x = \frac{9}{13}$ since their utility function is increasing on the interval $[0,\frac{4}{5})$. So the SPE is $(\frac{9}{13}, \frac{4}{13})$ $\Box$

\end{document}

% List of tex snippets:
%   - tex-header (this)
%   - R      --> \mathbb{R}
%   - Z      --> \mathbb{Z}
%   - B      --> \mathcal{B}
%   - E      --> \mathbb{E}
%   - M      --> \mathcal{M}
%   - m      --> \mathfrak{m}({#1})
%   - normlp --> \norm{{#1}}_{L^{{#2}}}
%   - dd     --> \;\mathrm{d}{#1}
