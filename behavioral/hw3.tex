\documentclass{article}
\usepackage[utf8]{inputenc}
\usepackage[margin=1in]{geometry}

\title{480 - Homework 3}
\author{Victor Zhang}
\date{October 18, 2021}

\usepackage[utf8]{inputenc}
\usepackage{amsmath}
\usepackage{amsfonts}
\usepackage{natbib}
\usepackage{graphicx}
% \usepackage{changepage}
\usepackage{amssymb}
\usepackage{xfrac}
% \usepackage{bm}
% \usepackage{empheq}
\usepackage{dirtytalk}

\newcommand{\contra}{\raisebox{\depth}{\#}}

\newenvironment{myindentpar}[1]
  {\begin{list}{}
          {\setlength{\leftmargin}{#1}}
          \item[]
  }
  {\end{list}}

\pagestyle{empty}

\begin{document}

\maketitle
% \begin{center}
% {\huge Econ 482 \hspace{0.5cm} HW 3}\
% {\Large \textbf{Victor Zhang}}\
% {\Large February 18, 2020}
% \end{center}

\section{}
\subsection{}
First note the probability of two consecutive good years is
$$Pr(\text{two good years } | \text{ normal}) = \left(\frac{5}{10}\right)^2 = \frac{1}{4},\; Pr(\text{two good years } | \text{ extraordinary}) = \left(\frac{8}{10}\right)^2 = \frac{16}{25}$$
Then by Bayes
\begin{equation*}
\begin{split}
Pr(\text{extraordinary } | \text{ two good years}) &= \frac{Pr(\text{two good years } | \text{ extraordinary}) Pr(\text{extraordinary})}{Pr(\text{two good years})}\\
&= \frac{\frac{16}{25} \cdot \frac{1}{100}}{\frac{16}{25} \cdot \frac{1}{100} + \frac{1}{4} \cdot \frac{99}{100}}\\
&= 0.0252
\end{split}
\end{equation*}
So indeed, the probability Golden Fleece is extraordinary is less than 3\% $\Box$

\subsection{}
This is confirmation bias. Since Susan and her friends are already invested in Golden Fleece, they come in with the belief Golden Fleece is a good firm. Beating the market is thus seen as evidence Golden Fleece is a good (extraordinary) firm $\Box$

\section{}
This is an example of the representativeness heuristic. Since runs are not seen as \say{representative} of randomness, people tend not to give them within examples of random strings.

\section{}
\subsection{}
The expected utility is $\frac{1}{2} \left( \sqrt{120} + \sqrt{80} \right) \approx 9.95$. The certainty equivalent is thus $9.95^2 \approx 98.99$. Since the expected value of this gamble is 100, the risk premium is $100 - 98.99 \approx 1.01$ $\Box$

\subsection{}
The expected utility is $\frac{1}{2} \left( \ln 120 + \ln 80 \right) \approx 4.58$. The certainty equivalent is $e^{4.58} \approx 97.98$. So the risk premium is $100 - 97.98 \approx 2.02$ $\Box$

\end{document}

% List of tex snippets:
%   - tex-header (this)
%   - R      --> \mathbb{R}
%   - Z      --> \mathbb{Z}
%   - B      --> \mathcal{B}
%   - E      --> \mathcal{E}
%   - M      --> \mathcal{M}
%   - m      --> \mathfrak{m}({#1})
%   - normlp --> \norm{{#1}}_{L^{{#2}}}
