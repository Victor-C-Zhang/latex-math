\documentclass{article}
\usepackage[utf8]{inputenc}
\usepackage[margin=1in]{geometry}

\title{480 - Homework 1}
\author{Victor Zhang}
\date{September 20, 2021}

\usepackage[utf8]{inputenc}
\usepackage{amsmath}
\usepackage{amsfonts}
\usepackage{natbib}
\usepackage{graphicx}
% \usepackage{changepage}
\usepackage{amssymb}
\usepackage{xfrac}
% \usepackage{bm}
% \usepackage{empheq}
\usepackage{dirtytalk}

\newcommand{\contra}{\raisebox{\depth}{\#}}

\newenvironment{myindentpar}[1]
  {\begin{list}{}
          {\setlength{\leftmargin}{#1}}
          \item[]
  }
  {\end{list}}

\pagestyle{empty}

\begin{document}

\maketitle
% \begin{center}
% {\huge Econ 482 \hspace{0.5cm} HW 3}\
% {\Large \textbf{Victor Zhang}}\
% {\Large February 18, 2020}
% \end{center}

\section{}
By completeness, Coke and Pepsi are at least comparable. Since $\text{Jolt} = \text{Pepsi}$ and $\text{Jolt} \succ \text{7Up}$, by transitivity $\text{Pepsi} \succ \text{7Up}$ and similarly since $\text{7Up} = \text{Coke}$ we have that $\text{Pepsi} \succ \text{Coke}$ $\Box$

\section{}
\subsection{}
Since $\leq$ between prices and $\geq$ between mpg are both transitive and $\succeq$ requires both $\leq$ and $\geq$ relations to hold, $\succeq$ between cars is also transitive $\Box$

\subsection{}
$\succeq$ is not complete. For instance, two cars $x,y$ with $p(x) < p(y)$ and $mpg(x) < mpg(y)$ are incomparable $\contra$

\subsection{}
$\succeq$ is not negative transitive. As a counterexample, take 3 cars $x,y,z$ with $p(x) = 1$, $p(y) = 3$, $p(z) = 2$, $mpg(x) = 2$, $mpg(y) = 3$, $mpg(z) = 1$. Then $p(x) < p(y)$ and $mpg(x) < mpg(y)$ so $x \nsucceq y$. $p(y) > p(z)$ and $mpg(y) > mpg(z)$ so $y \nsucceq z$. But $p(x) < p(z)$ and $mpg(x) > mpg(z)$ so $x \succeq z$ $\contra$

\subsection{}
Since $\succeq$ is not complete, it is not a rational preference relation $\Box$

\section{}
We propose preferences $J = P$, $C = P$, $J \succ C$. This is clearly not rational, since transitivity of $\succeq$ requires $J = C$. But this preference ordering is complete and vacuously transitive under $\succ$ so is semi-rational $\Box$

\section{}
One example of irrational behavior is when customers happily choose the second-most expensive item when presented with the most expensive item as a decoy. This is explained by Angner as the \textit{compromise effect}. In this case, consumers are \say{compromising} between a very nice, but very expensive, entree and a cheap, but not as nice, entree.

\end{document}

% List of tex snippets:
%   - tex-header (this)
%   - R      --> \mathbb{R}
%   - Z      --> \mathbb{Z}
%   - B      --> \mathcal{B}
%   - E      --> \mathcal{E}
%   - M      --> \mathcal{M}
%   - m      --> \mathfrak{m}({#1})
%   - normlp --> \norm{{#1}}_{L^{{#2}}}
