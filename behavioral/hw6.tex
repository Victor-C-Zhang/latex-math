\documentclass{article}
\usepackage[utf8]{inputenc}
\usepackage[margin=1in]{geometry}

\title{480 - Homework 6}
\author{Victor Zhang}
\date{November 16, 2021}

\usepackage[utf8]{inputenc}
\usepackage{amsmath}
\usepackage{amsfonts}
\usepackage{natbib}
\usepackage{graphicx}
% \usepackage{changepage}
\usepackage{amssymb}
\usepackage{xfrac}
% \usepackage{bm}
% \usepackage{empheq}

\newcommand{\contra}{\raisebox{\depth}{\#}}

\newenvironment{myindentpar}[1]
  {\begin{list}{}
          {
            \setlength{\leftmargin}{#1}
            \setlength{\rightmargin}{#1}
          }
          \item[]
  }
  {\end{list}}

\pagestyle{empty}

\begin{document}

\maketitle
% \begin{center}
% {\huge Econ 482 \hspace{0.5cm} HW 3}\
% {\Large \textbf{Victor Zhang}}\
% {\Large February 18, 2020}
% \end{center}

\section{}
\subsection{}
At time 0, the value of option A is $\delta^2 \cdot 6 = \tfrac{8}{3}$ and the value of option B is $\delta \cdot 3 = 2$.

\subsection{}
At time 1, the value of option A is $\delta \cdot 6 = 4$ and the value of option B is 3.

\subsection{}
At time 0, the value of option A is $\beta\delta^2 \cdot 6 = 2$ and the value of option B is $\beta \delta \cdot 3 = 1$.

\subsection{}
At time 1, the value of option A is $\beta \delta \cdot 6 = 2$ and the value of option B is 3.

\subsection{}
Hip will end up happy and healthy, since they are time consistent and value option A higher than option B. If no commitments can be made, Hop will experience preference reversal at time 1 and end up watching television and not be happy or healthy $\Box$

\subsection{}
In this case, since both value option A higher than option B at time 0, commitment will ensure both of them are happy and healthy. It doesn't matter if Hop is naive or sophisticated since a commitment forces Hop to act on time 0 preferences regardless $\Box$

\section{}
\subsection{}
At time 0, the utility of option A is $\beta \delta^2 \cdot 18 = \tfrac{108}{49}$ and the utility of option B is $\beta \delta \cdot 6 + \beta \delta^2 \cdot 1 = \tfrac{48}{49}$. So if she can make a firm commitment she will pick option A $\Box$

\subsection{}
At time 0, the utility of option A is $\beta \delta \cdot 18 = \tfrac{18}{7}$ and the utility of option B is $6 + \beta \delta = \tfrac{43}{7}$. So at time 1 she will choose to watch TV. There is nothing she can do at time 0 to prevent preference reversal, so she will not go to the gym at time 1 $\Box$

\section{}
\subsection{}
At time 0, the utility of $\alpha$ is 3, the utility of $\beta$ is 2, and the utility of $\gamma$ is 3.5. So we will choose to skip $\alpha$ in hopes of seeing $\gamma$. At time 1, the utility of $\beta$ is 4 and the utility of $\gamma$ is 3.5, so we will choose to see $\beta$ $\Box$

\subsection{}
If we are a sophisticated hyperbolic discounter, we realize we will chose $\beta$ at time 1. That is, if we watch $\alpha$ we enjoy utility of 3 and if we skip $\alpha$ we will watch $\beta$ and thus enjoy a discounted utility of 2. Thus, we will choose to watch $\alpha$ $\Box$

\subsection{}
Since at time 0 the utility of $\gamma$ is the highest, we will commit to seeing that movie.

\end{document}

% List of tex snippets:
%   - tex-header (this)
%   - R      --> \mathbb{R}
%   - Z      --> \mathbb{Z}
%   - B      --> \mathcal{B}
%   - E      --> \mathcal{E}
%   - M      --> \mathcal{M}
%   - m      --> \mathfrak{m}({#1})
%   - normlp --> \norm{{#1}}_{L^{{#2}}}
