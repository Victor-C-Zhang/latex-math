\documentclass{article}
\usepackage[utf8]{inputenc}
\usepackage[margin=1in]{geometry}

\title{480 - Homework 7}
\author{Victor Zhang}
\date{November 29, 2021}

\usepackage[utf8]{inputenc}
\usepackage{amsmath}
\usepackage{amsfonts}
\usepackage{natbib}
\usepackage{graphicx}
% \usepackage{changepage}
\usepackage{amssymb}
\usepackage{xfrac}
% \usepackage{bm}
% \usepackage{empheq}
\usepackage{tikz}

\newcommand{\contra}{\raisebox{\depth}{\#}}

\newenvironment{myindentpar}[1]
  {\begin{list}{}
          {
            \setlength{\leftmargin}{#1}
            \setlength{\rightmargin}{#1}
          }
          \item[]
  }
  {\end{list}}

\pagestyle{empty}

\begin{document}

\maketitle
% \begin{center}
% {\huge Econ 482 \hspace{0.5cm} HW 3}\
% {\Large \textbf{Victor Zhang}}\
% {\Large February 18, 2020}
% \end{center}

\section{}
\subsection{}
The strategy $(M,m)$ is the only Nash equilibrium. Best responses for players 1 and 2 are marked with \textbf{bold} and \underline{underline}, respectively.
\begin{center}
\begin{tabular}{c c c c}
 & $l$ & $m$ & $r$\\
$U$ & \textbf{3},2 & 5,0 & 3,\underline{9}\\
$M$ & 1,1 & \textbf{6},\underline{4} & 0,2\\
$D$ & 2,7 & 4,\underline{9} & \textbf{9},8
\end{tabular}
\end{center}

\subsection{}
The Nash equilibrium found is not trembling-hand perfect, since if player 2 switched from $m$ to $r$, player 1's best response would be $D$ not $M$.

\subsection{}
No, the Nash equilibrium is not Pareto optimal since $(D,r)$ yields strictly better payoffs for both players.

\subsection{}
For player 2, $l$ is strictly dominated by $r$. Player 1 has no strictly dominated strategies.

\section{}
\subsection{}
% Set the overall layout of the tree
\tikzstyle{level 1}=[level distance=1cm, sibling distance=3cm]
\tikzstyle{level 2}=[level distance=1cm, sibling distance=1.5cm]

% Define styles for internals and leaves
\tikzstyle{internal} = [circle,inner sep = 0cm, fill=black, text width = 1.5mm]
\tikzstyle{leaf} = [text width=4em, text centered]
\begin{center}
\begin{tikzpicture}[
    grow=down,
    norm/.style={edge from parent/.style={black, thin, draw}},
    spe/.style={edge from parent/.style={red, very thick, draw}}
  ]
\node[internal] {}
child {
  node[internal] {}
  child[spe] {
    node[leaf] {1,3,7}
    edge from parent[->>>>] node[left]{L}
  }
  child {
    node[internal] {}
    child {
      node[leaf] {8,8,8}
      edge from parent node[left]{A}
    }
    child[spe] {
      node[leaf] {8,2,9}
      edge from parent[->>>>] node[right]{B}
    }
    edge from parent node[right]{R}
  }
  edge from parent node[left]{L}
}
child[spe] {
  node[internal] {}
  child[norm] {
    node[leaf] {0,3,0}
    edge from parent node[left]{C}
  }
  child[spe] {
    node[leaf] {(2,0,1)}
    edge from parent[->>>>] node[right]{D}
  }
  edge from parent[->>>>] node[right]{R}
};
\end{tikzpicture}
\end{center}

\subsection{}
This is clearly not Pareto-optimal. Everybody is strictly better off if the players choose $(L,L,A)$ with payoffs $(8,8,8)$.

\section{}
\subsection{}
Note it is never optimal for people to contribute different amounts, since the public good is taken as the minimum contribution. So the Pareto-optimal solution is for everybody to contribute the maximum 10 dollars for payoff 20 each $\Box$

\subsection{}
Yes, this is a Nash equilibrium. Since everybody contributes the maximum, any other response would decrease the payoff. That is, if player $i$ contributes $x_i < 10$ and all others contribute 10, their payoff would be
$$2y + (10 - x_i) = 2x_i + (10 - x_i) = 10 + x_i < 20 \; \Box$$

\end{document}

% List of tex snippets:
%   - tex-header (this)
%   - R      --> \mathbb{R}
%   - Z      --> \mathbb{Z}
%   - B      --> \mathcal{B}
%   - E      --> \mathcal{E}
%   - M      --> \mathcal{M}
%   - m      --> \mathfrak{m}({#1})
%   - normlp --> \norm{{#1}}_{L^{{#2}}}
