\documentclass{article}
\usepackage[utf8]{inputenc}
\usepackage[margin=1in]{geometry}

\title{452 - Homework 10}
\author{Victor Zhang}
\date{April 19, 2021}

\usepackage[utf8]{inputenc}
\usepackage{amsmath}
\usepackage{amsfonts}
\usepackage{natbib}
\usepackage{graphicx}
% \usepackage{changepage}
\usepackage{amssymb}
\usepackage{xfrac}
% \usepackage{bm}
% \usepackage{empheq}

\newcommand{\contra}{\raisebox{\depth}{\#}}

\newenvironment{myindentpar}[1]
  {\begin{list}{}
          {\setlength{\leftmargin}{#1}}
          \item[]
  }
  {\end{list}}

\pagestyle{empty}

\begin{document}

\maketitle
% \begin{center}
% {\huge Econ 482 \hspace{0.5cm} HW 3}\
% {\Large \textbf{Victor Zhang}}\
% {\Large February 18, 2020}
% \end{center}

\section*{4.}
Suppose there are finitely many irreducible polynomials. Then there are finitely many irreducible polynomials of positive degree $p_1, p_2, \dots p_n$. In particular, we are guaranteed to have at least two, since $x-a$ is irreducible for $a \in \mathbb{F}$. Since each of these polynomials has finite degree, simply multiply all of these together and add $1_F$. We claim this new polynomial is irreducible. Indeed, since $\mathbb{F}[x]$ is PID, it is UFD and thus every reducible polynomial must have factorization to irreducibles. But clearly none of $p_i$ can be a factor of this new polynomial, since otherwise it would divide $1_F$, a polynomial with degree 0. So in fact we have generated a new irreducible polynomial, a contradiction $\contra$

\end{document}

% List of tex snippets:
%   - tex-header (this)
%   - R      --> \mathbb{R}
%   - Z      --> \mathbb{Z}
%   - B      --> \mathcal{B}
%   - E      --> \mathcal{E}
%   - M      --> \mathcal{M}
%   - m      --> \mathfrak{m}({#1})
%   - normlp --> \norm{{#1}}_{L^{{#2}}}
