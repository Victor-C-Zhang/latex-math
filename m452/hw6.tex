\documentclass{article}
\usepackage[utf8]{inputenc}
\usepackage[margin=1in]{geometry}

\title{452 - Homework 6}
\author{Victor Zhang}
\date{}

\usepackage[utf8]{inputenc}
\usepackage{amsmath}
\usepackage{amsfonts}
\usepackage{natbib}
\usepackage{graphicx}
% \usepackage{changepage}
\usepackage{amssymb}
\usepackage{xfrac}
% \usepackage{bm}
% \usepackage{empheq}

\newcommand{\contra}{\raisebox{\depth}{\#}}

\newenvironment{myindentpar}[1]
  {\begin{list}{}
          {\setlength{\leftmargin}{#1}}
          \item[]
  }
  {\end{list}}

\pagestyle{empty}

\begin{document}

\maketitle
% \begin{center}
% {\huge Econ 482 \hspace{0.5cm} HW 3}\
% {\Large \textbf{Victor Zhang}}\
% {\Large February 18, 2020}
% \end{center}

\section*{5.}
Suppose $ab \in \varphi^{-1}(P)$. Then by homomorphism, $\varphi(ab) = \varphi(a)\varphi(b) \in P$. By primality, at least one of $\varphi(a),\varphi(b) \in P$. Then at least one of $a,b \in \varphi^{-1}(P)$. Moreover, $\varphi^{-1}(P)$ cannot be the entirety of $R$, since that would imply $1 \in \varphi^{-1}(P)$ and $1 \in P$ by homomorphism. But since $P$ is a proper ideal, so is $\varphi^{-1}(P)$. The desired conclusion follows immediately $\Box$

\end{document}

% List of tex snippets:
%   - tex-header (this)
%   - R      --> \mathbb{R}
%   - Z      --> \mathbb{Z}
%   - B      --> \mathcal{B}
%   - E      --> \mathcal{E}
%   - M      --> \mathcal{M}
%   - m      --> \mathfrak{m}({#1})
%   - normlp --> \norm{{#1}}_{L^{{#2}}}
