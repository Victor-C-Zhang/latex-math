\documentclass{article}
\usepackage[utf8]{inputenc}
\usepackage[margin=1in]{geometry}

\title{452 - Homework 1}
\author{Victor Zhang}

\usepackage[utf8]{inputenc}
\usepackage{amsmath}
\usepackage{amsfonts}
\usepackage{natbib}
\usepackage{graphicx}
% \usepackage{changepage}
\usepackage{amssymb}
\usepackage{xfrac}
% \usepackage{bm}
% \usepackage{empheq}

\newcommand{\contra}{\raisebox{\depth}{\#}}

\newenvironment{myindentpar}[1]
  {\begin{list}{}
          {\setlength{\leftmargin}{#1}}
          \item[]
  }
  {\end{list}}

\pagestyle{empty}

\begin{document}

\maketitle
% \begin{center}
% {\huge Econ 482 \hspace{0.5cm} HW 3}\
% {\Large \textbf{Victor Zhang}}\
% {\Large February 18, 2020}
% \end{center}

\section*{2.a}
No, since it violates distributivity. Take $f(x) = x$, $g(x) = x$, $h(x) = x^2$. Then $h(f+g) = 4x^2$ but $hf + hg = 2x^2$.
\section*{2.b}
Yes. It contains $1 = x \mapsto 1$ and $0 = x \mapsto 0$. It is closed and associative under addition and has inverses $(-f)(x) = -(f(x))$, so it is a group under addition. It is closed and associative under multiplication, so is a monoid under multiplication. Unlike (a), this also follows distributivity $\Box$

\section*{4}
All elements of $Z(R)$ commute with every other element. Thus, for $x,y \in Z(R)$ we have
$$(x+y)r = xr + yr = rx + ry = r(x+y)$$
So $Z(R)$ is closed under addition. Similarly
$$xyr = xry = rxy$$
So $Z(R)$ is closed under multiplcation. Clearly, $1 \in Z(R)$ so we are done $\Box$

\section*{6}
Suppose the finite ring $R$ has size $r$. We may construct unique homomorphism $\phi: \mathbb{Z}/r\mathbb{Z} \rightarrow R$, so it suffices to prove the claim for $\mathbb{Z}/r\mathbb{Z}$. Indeed, all $x$ relatively prime to $r$ have inverses $\mod r$, so are units. All other elements have nontrivial common factor, so must be expressible as $ax = br$ for integral $a,b < r$.\\
A domain is simply a ring with no zero divisors. A finite domain must then be entirely composed of units, in other words, closed under multiplicative inverses. Thus, it is a field $\Box$

\section*{7}
Take ring $\mathbb{Z}$ and element $7 \in \mathbb{Z}$. Clearly, since $\mathbb{Z}$ is not closed under inverses, $7$ is not a unit. However, there is no finite $x \in \mathbb{Z}$ s.t. $7x = 0$, so it is not a zero divisor either $\Box$

\section*{8.a}
Note by commutativity $(rx)^n = rxrxrx\dots rx = r^nx^n = r^n \cdot 0 = 0$. So $rx$ is nilpotent $\Box$
\section*{8.b}
By construction, $(1-x)(1+x+x^2+\dots x^n-1) = 1 - x^n = 1$, so $(1-x)$ is a unit $\Box$
\section*{8.c}
Since $y$ is a unit, we may note $\frac{x}{y} \in R$ is well defined and nilpotent, by (a). Then $1-\frac{x}{y}$ is a unit. Since $y$ is a unit, so is $-y$. Then $1 - \frac{x}{-y} = 1 + \frac{x}{y}$ is a unit. Then $y(1+\frac{x}{y}) = y + x = x + y$ is also a unit, since units are a multiplicative group $\Box$


\end{document}

% List of tex snippets:
%   - tex-header (this)
%   - R      --> \mathbb{R}
%   - Z      --> \mathbb{Z}
%   - B      --> \mathcal{B}
%   - E      --> \mathcal{E}
%   - M      --> \mathcal{M}
%   - m      --> \mathfrak{m}({#1})
%   - normlp --> \norm{{#1}}_{L^{{#2}}}
