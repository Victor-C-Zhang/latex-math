\documentclass{article}
\usepackage[utf8]{inputenc}
\usepackage[margin=1in]{geometry}

\title{452 - Exam 3}
\author{Victor Zhang}
\date{May 10, 2021}

\usepackage[utf8]{inputenc}
\usepackage{amsmath}
\usepackage{amsfonts}
\usepackage{natbib}
\usepackage{graphicx}
% \usepackage{changepage}
\usepackage{amssymb}
\usepackage{xfrac}
% \usepackage{bm}
% \usepackage{empheq}

\newcommand{\contra}{\raisebox{\depth}{\#}}

\newenvironment{myindentpar}[1]
  {\begin{list}{}
          {\setlength{\leftmargin}{#1}}
          \item[]
  }
  {\end{list}}

\pagestyle{empty}

\begin{document}

\maketitle
% \begin{center}
% {\huge Econ 482 \hspace{0.5cm} HW 3}\
% {\Large \textbf{Victor Zhang}}\
% {\Large February 18, 2020}
% \end{center}

\section{}
\subsection{}
$V_k$ is closed to addition since if $r \in I^k$, $v_1,v_2 \in V_k$ then $r(v_1 + v_2) = rv_1 + rv_2 = 0 + 0 = 0$. It is also closed to inverses, since $0 = r0 = r(v + -v) = rv + r(-v)$. So it is a subgroup of $V$. Since $I^k$ is an ideal, for all scalars $s \in R$, $r(sv_1) = r'v_1 = 0$ for some $r' = rs \in I^k$. So $V_k$ is also closed to scalar multiplication and is thus a submodule $\Box$
\subsection{}
Consider homomorphism $\phi : R \to \sfrac{R}{I^k}$ induced by ideal $I^k$. We will refer to elements of $\sfrac{R}{I^k}$ canonically as $\overline{r} = r + I^k$. Similarly, define scalar multiplication canonically as $\overline{r} \cdot x = (r + I^k) \cdot x$. We have already shown $V_k$ closed to addition in $R$ and we note $(I^k) \cdot x = 0$ for all $x \in V_k$, so by a similar argument $V_k$ is closed to addition in $\sfrac{R}{I^k}$. $V_k$ is also closed to scalar multiplication, since $(r + I^k) \cdot x = r \cdot x + I^k \cdot x = r \cdot x$ and we inherit closure from the structure under $R$. Now it suffices to show the module axioms, which we can do by borrowing the axioms from multiplication under $R$.\\
$$\overline{r} \cdot (x+y) = (r + I^k) \cdot (x+y) = r \cdot (x + y) + I^k \cdot (x+y) = r \cdot x + r \cdot y = (r + I^k) \cdot x + (r + I^k) \cdot y = \overline{r} \cdot x + \overline{r} \cdot y$$
$$(\overline{r} + \overline{s}) \cdot x = (r + I^k + s + I^k) \cdot x = (r + s) + (I^k + I^k) \cdot x = r \cdot x + s \cdot x = (r + I^k) \cdot x + (s + I^k) \cdot x = \overline{r} \cdot x + \overline{s} \cdot x$$
$$(\overline{r}\overline{s}) \cdot x = ((r + I^k)(s + I^k)) \cdot x = (r + I^k) ((s + I^k) \cdot x) = \overline{r} (\overline{s} \cdot x)$$
$$\overline{1} \cdot x = (1 + I^k) \cdot x = 1 \cdot x = x$$

\section{}
\subsection{}
We note $3^2 = 4^2 = 2$ mod 7. So we may construct scalar multiplications
\begin{equation*}
\begin{cases}
zv = (z \!\!\!\mod 7)v\\
\sqrt{2}v = 3v
\end{cases}
\begin{cases}
zv = (z \!\!\!\mod 7)v\\
\sqrt{2}v = 4v
\end{cases}
\Box
\end{equation*}

\subsection{}
No. Suppose there is some isomorphism $\phi: V_a \to V_b$ from the first structure to the second. Then by the definition of module homomorphism,
$$\phi(6) = \phi(2\sqrt{2}\cdot_a 1) = (2\sqrt{2}) \cdot_b \phi(1) = 1 \cdot_b \phi(1) = \phi(1)$$
which contradicts the bijection condition for isomorphism $\contra$

\subsection{}
By a similar reasoning to 1.2 it suffices to find $Ann_R(V)$, which conveniently is just $\langle 7, 3 - \sqrt{2} \rangle$ for the first structure and $\langle 7, 4 - \sqrt{2} \rangle$ for the second. Then $V = \frac{\mathbb{Z}[\sqrt{2}]}{\langle 7, 3 - \sqrt{2} \rangle}, \frac{\mathbb{Z}[\sqrt{2}]}{\langle 7, 4 - \sqrt{2} \rangle}$ $\Box$

\section{}
\subsection{}
$n = 6$. The first submodule has Jordan form $J_1(0) = \begin{pmatrix}0\end{pmatrix}$. The second is $J_1(3) = \begin{pmatrix}3\end{pmatrix}$. The last is $J_4(0) = \begin{pmatrix}0 & 1 & 0 & 0\\ 0 & 0 & 1 & 0\\ 0 & 0 & 0 & 1\\ 0 & 0 & 0 & 0\end{pmatrix}$. Putting all these together gives
$$A \sim \begin{pmatrix}
0 & 0 & 0 & 0 & 0 & 0 \\ 0 & 3 & 0 & 0 & 0 & 0 \\ 0 & 0 & 0 & 1 & 0 & 0 \\ 0 & 0 & 0 & 0 & 1 & 0 \\ 0 & 0 & 0 & 0 & 0 & 1 \\ 0 & 0 & 0 & 0 & 0 & 0
\end{pmatrix}$$
The companion matrices are $C_x = \begin{pmatrix}0\end{pmatrix}$, $C_{x-3} = \begin{pmatrix}3\end{pmatrix}$, $C_{x^4} = \begin{pmatrix}0 & 0 & 0 & 0\\ 1 & 0 & 0 & 0\\ 0 & 1 & 0 & 0\\ 0 & 0 & 1 & 0\end{pmatrix}$
$$A \sim \begin{pmatrix}
0 & 0 & 0 & 0 & 0 & 0 \\ 0 & 3 & 0 & 0 & 0 & 0 \\ 0 & 0 & 0 & 0 & 0 & 0 \\ 0 & 0 & 1 & 0 & 0 & 0 \\ 0 & 0 & 0 & 1 & 0 & 0 \\ 0 & 0 & 0 & 0 & 1 & 0
\end{pmatrix} \;\Box$$

\subsection{}
Since $\mathbb{C}$ is a field, we have isomorphism of $\mathbb{C}[x]$ modules $\mathbb{C}^n \simeq \sfrac{\mathbb{C}[x]^n}{\langle xI - A^2 \rangle \mathbb{C}[x]^n}$. We apply the standard procedure to find Smith normal form, keeping in mind the units of $\mathbb{C}[x]$ are exactly the units of $\mathbb{C}$. The matrix is
$$xI - A^2 = \begin{pmatrix}
x & 0 & 0 & 0 & 0 & 0 \\ 0 & x - 9 & 0 & 0 & 0 & 0 \\ 0 & 0 & x & 0 & -1 & 0 \\ 0 & 0 & 0 & x & 0 & -1 \\ 0 & 0 & 0 & 0 & x & 0 \\ 0 & 0 & 0 & 0 & 0 & x
\end{pmatrix}$$
We first focus on the bottom right corner matrix:
\begin{equation*}
\begin{split}
\begin{pmatrix}
x & 0 & -1 & 0\\
0 & x & 0 & -1\\
0 & 0 & x & 0 \\
0 & 0 & 0 & x \\
\end{pmatrix}
&\xrightarrow[]{R_3 += xR_1}
\begin{pmatrix}
x & 0 & -1 & 0\\
0 & x & 0 & -1\\
x^2 & 0 & 0 & 0 \\
0 & 0 & 0 & x \\
\end{pmatrix}
\xrightarrow[]{R_4 += xR_2}
\begin{pmatrix}
x & 0 & -1 & 0\\
0 & x & 0 & -1\\
x^2 & 0 & 0 & 0 \\
0 & x^2 & 0 & 0 \\
\end{pmatrix}\\
&\xrightarrow[]{C_1 += xC_3}
\begin{pmatrix}
0 & 0 & -1 & 0\\
0 & 0 & 0 & -1\\
x^2 & 0 & 0 & 0 \\
0 & x^2 & 0 & 0 \\
\end{pmatrix}
\rightarrow
\begin{pmatrix}
1 & 0 & 0 & 0\\
0 & 1 & 0 & 0\\
0 & 0 & x^2 & 0 \\
0 & 0 & 0 & x^2 \\
\end{pmatrix}
\end{split}
\end{equation*}
Then we may write $xI - A^2$ as
\begin{equation*}
\begin{split}
\begin{pmatrix}
x & 0 & 0 & 0 & 0 & 0 \\ 0 & x - 9 & 0 & 0 & 0 & 0 \\ 0 & 0 & x & 0 & -1 & 0 \\ 0 & 0 & 0 & x & 0 & -1 \\ 0 & 0 & 0 & 0 & x & 0 \\ 0 & 0 & 0 & 0 & 0 & x
\end{pmatrix}
&\rightarrow
\begin{pmatrix}
x & 0 & 0 & 0 & 0 & 0 \\ 0 & x - 9 & 0 & 0 & 0 & 0 \\ 0 & 0 & 1 & 0 & 0 & 0 \\ 0 & 0 & 0 & 1 & 0 & 0 \\ 0 & 0 & 0 & 0 & x^2 & 0 \\ 0 & 0 & 0 & 0 & 0 & x^2
\end{pmatrix}\\
&\rightarrow
\begin{pmatrix}
1 & 0 & 0 & 0 & 0 & 0 \\ 0 & 1 & 0 & 0 & 0 & 0 \\ 0 & 0 & x & 0 & 0 & 0 \\ 0 & 0 & 0 & x^2 & 0 & 0 \\ 0 & 0 & 0 & 0 & x^2 & 0 \\ 0 & 0 & 0 & 0 & 0 & x-9
\end{pmatrix}\\
&\rightarrow
\begin{pmatrix}
1 & 0 & 0 & 0 & 0 & 0 \\ 0 & 1 & 0 & 0 & 0 & 0 \\ 0 & 0 & 1 & 0 & 0 & 0 \\ 0 & 0 & 0 & x & 0 & 0 \\ 0 & 0 & 0 & 0 & x^2 & 0 \\ 0 & 0 & 0 & 0 & 0 & x^2(x-9)
\end{pmatrix}
\end{split}
\end{equation*}
Then $\mathbb{C}^6 \simeq \frac{\mathbb{C}[x]}{\langle 1 \rangle} \oplus \frac{\mathbb{C}[x]}{\langle 1 \rangle} \oplus \frac{\mathbb{C}[x]}{\langle 1 \rangle} \oplus \frac{\mathbb{C}[x]}{\langle x \rangle} \oplus \frac{\mathbb{C}[x]}{\langle x^2 \rangle} \oplus \frac{\mathbb{C}[x]}{\langle x^2(x-9) \rangle} \simeq \frac{\mathbb{C}[x]}{\langle x \rangle} \oplus \frac{\mathbb{C}[x]}{\langle x^2 \rangle} \oplus \frac{\mathbb{C}[x]}{\langle x^2(x-9) \rangle}$ $\Box$

\end{document}

% List of tex snippets:
%   - tex-header (this)
%   - R      --> \mathbb{R}
%   - Z      --> \mathbb{Z}
%   - B      --> \mathcal{B}
%   - E      --> \mathcal{E}
%   - M      --> \mathcal{M}
%   - m      --> \mathfrak{m}({#1})
%   - normlp --> \norm{{#1}}_{L^{{#2}}}
