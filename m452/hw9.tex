\documentclass{article}
\usepackage[utf8]{inputenc}
\usepackage[margin=1in]{geometry}

\title{452 - Homework 9}
\author{Victor Zhang}
\date{April 18, 2021}

\usepackage[utf8]{inputenc}
\usepackage{amsmath}
\usepackage{amsfonts}
\usepackage{natbib}
\usepackage{graphicx}
% \usepackage{changepage}
\usepackage{amssymb}
\usepackage{xfrac}
% \usepackage{bm}
% \usepackage{empheq}

\newcommand{\contra}{\raisebox{\depth}{\#}}

\newenvironment{myindentpar}[1]
  {\begin{list}{}
          {\setlength{\leftmargin}{#1}}
          \item[]
  }
  {\end{list}}

\pagestyle{empty}

\begin{document}

\maketitle
% \begin{center}
% {\huge Econ 482 \hspace{0.5cm} HW 3}\
% {\Large \textbf{Victor Zhang}}\
% {\Large February 18, 2020}
% \end{center}

\section*{3.}
Let $I$ be an ideal of localization $S^{-1}R$. Then by correspondence, there is $J \lhd R$ s.t. $I = S^{-1}J$. $J$ is principle, so let it be generated as $\langle x \rangle$. Then we claim $I = \langle \bar{x} \rangle$. Indeed, if $a \in I$ we may write $a = \frac{r}{s}$ for $r \in J$, $s \in S$. But then $r = bx$ for some $b$, so in fact $a = \frac{b}{s}\frac{x}{1} \in \langle \bar{x} \rangle$. Similarly, if $a \in \langle \bar{x} \rangle$ then clearly $a = \frac{r}{s}\frac{x}{1} = \frac{y}{s} \in S^{-1}J$ for $y \in J$, since $J$ is an ideal $\Box$\\
$\mathbb{F}$ is PID so clearly $\mathbb{F}[x]$ is PID. Then $\mathbb{F}[x,x^{-1}]$ is simply the localization of $\mathbb{F}[x]$ by $\{x^i\}_{i \in \mathbb{N}}$, which as we just showed is PID $\Box$

\end{document}

% List of tex snippets:
%   - tex-header (this)
%   - R      --> \mathbb{R}
%   - Z      --> \mathbb{Z}
%   - B      --> \mathcal{B}
%   - E      --> \mathcal{E}
%   - M      --> \mathcal{M}
%   - m      --> \mathfrak{m}({#1})
%   - normlp --> \norm{{#1}}_{L^{{#2}}}
