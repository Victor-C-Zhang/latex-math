\documentclass{article}
\usepackage[utf8]{inputenc}
\usepackage[margin=1in]{geometry}

\title{452 - Homework 7}
\author{Victor Zhang}
\date{}

\usepackage[utf8]{inputenc}
\usepackage{amsmath}
\usepackage{amsfonts}
\usepackage{natbib}
\usepackage{graphicx}
% \usepackage{changepage}
\usepackage{amssymb}
\usepackage{xfrac}
% \usepackage{bm}
% \usepackage{empheq}

\newcommand{\contra}{\raisebox{\depth}{\#}}

\newenvironment{myindentpar}[1]
  {\begin{list}{}
          {\setlength{\leftmargin}{#1}}
          \item[]
  }
  {\end{list}}

\pagestyle{empty}

\begin{document}

\maketitle
% \begin{center}
% {\huge Econ 482 \hspace{0.5cm} HW 3}\
% {\Large \textbf{Victor Zhang}}\
% {\Large February 18, 2020}
% \end{center}

\section*{5.a}
For every prime ideal $0 \in P$. Then by nilpotence $x \cdot x^{n-1} \in P$ so one of $x, x^{n-1} \in P$. If $x \in P$ we are done. Otherwise, take $x^{n-1} = x \cdot x^{n-2}$ and we may inductively reason that $x \in P$ by iteratively shedding factors of $x$ until we get $x^2 = x \cdot x \in P$ $\Box$

\section*{5.b}
Clearly, $\mathcal{N} \subseteq \bigcap\limits_{P \text{ prime}} P$. Suppose there is some $x \in (\bigcap P) \setminus \mathcal{N}$. Then since $x^j \neq 0$ for all $j$, we may find nontrivial localization $R_x$. $R_x$ has unique maximal ideal $M$, which is prime. Then by correspondence and universal property of localization, the pre-image of $M$ is a prime ideal $P' \lhd R$. But $P'$ must not contain $x$, since otherwise $1 \in M$ and $M$ is not maximal. Then $x \notin \bigcap P$, a contradiction. So in fact, $\mathcal{N} = \bigcap P$ $\Box$


\end{document}

% List of tex snippets:
%   - tex-header (this)
%   - R      --> \mathbb{R}
%   - Z      --> \mathbb{Z}
%   - B      --> \mathcal{B}
%   - E      --> \mathcal{E}
%   - M      --> \mathcal{M}
%   - m      --> \mathfrak{m}({#1})
%   - normlp --> \norm{{#1}}_{L^{{#2}}}
