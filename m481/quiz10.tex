\documentclass{article}
\usepackage[utf8]{inputenc}
\usepackage[margin=1in]{geometry}

\title{481 - Quiz 10}
\author{Victor Zhang}
\date{November 23, 2020}

\usepackage[utf8]{inputenc}
\usepackage{amsmath}
\usepackage{amsfonts}
\usepackage{natbib}
\usepackage{graphicx}
% \usepackage{changepage}
\usepackage{amssymb}
\usepackage{xfrac}
% \usepackage{bm}
% \usepackage{empheq}
\usepackage{multirow}

\newcommand{\contra}{\raisebox{\depth}{\#}}

\newenvironment{myindentpar}[1]
  {\begin{list}{}
          {\setlength{\leftmargin}{#1}}
          \item[]
  }
  {\end{list}}

\pagestyle{empty}

\begin{document}

\maketitle
% \begin{center}
% {\huge Econ 482 \hspace{0.5cm} HW 3}\
% {\Large \textbf{Victor Zhang}}\
% {\Large February 18, 2020}
% \end{center}

\section{}
From the example, we see the likelihood ratio estimator $\lambda$ is given by the table
\begin{center}\begin{tabular}{ c c c c c c c }
     1 & 2 & 3 & 4 & 5 & 6 & 7 \\
     \cline{1-7}
     $\sfrac{1}{4}$ & $\sfrac{1}{4}$ & $\sfrac{1}{4}$ & $\sfrac{3}{8}$ & 1 & 1 & 1 \\
\end{tabular}\end{center}
To get size 0.1, we cannot include more than one of the values of $X$ in our rejection region. This means we cannot include $\lambda = \frac{1}{4}$ in our rejection region, since then the size would be $0.25$. We are forced to reject when $\lambda < \frac{1}{4}$, in other words not at all. Obviously, this test will have no power.\\
Constrast this with another region $C : \{x = 3\}$. This has size $\alpha = \frac{1}{12} < 0.1$. The probability of a type II error is $\beta = \frac{1}{3}$, so the power is $\frac{2}{3}$, greater than the power of the likelihood ratio test $\Box$

\end{document}

% List of tex snippets:
%   - tex-header (this)
%   - R      --> \mathbb{R}
%   - Z      --> \mathbb{Z}
%   - B      --> \mathcal{B}
%   - E      --> \mathcal{E}
%   - M      --> \mathcal{M}
%   - m      --> \mathfrak{m}({#1})
%   - normlp --> \norm{{#1}}_{L^{{#2}}}
