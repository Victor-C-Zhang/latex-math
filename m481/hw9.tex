\documentclass{article}
\usepackage[utf8]{inputenc}

\title{481 - Homework 9}
\author{Victor Zhang}
\date{November 2, 2020}

\usepackage[utf8]{inputenc}
\usepackage{amsmath}
\usepackage{amsfonts}
\usepackage{natbib}
\usepackage{graphicx}
% \usepackage{changepage}
\usepackage{amssymb}
\usepackage{xfrac}
% \usepackage{bm}
% \usepackage{empheq}

\newcommand{\contra}{\raisebox{\depth}{\#}}

\newenvironment{myindentpar}[1]
  {\begin{list}{}
          {\setlength{\leftmargin}{#1}}
          \item[]
  }
  {\end{list}}

\pagestyle{empty}

\begin{document}

\maketitle
% \begin{center}
% {\huge Econ 482 \hspace{0.5cm} HW 3}\
% {\Large \textbf{Victor Zhang}}\
% {\Large February 18, 2020}
% \end{center}

\section*{11.17}
$$\theta < \frac{1}{2n}\chi^2_{\alpha,2(x+1)} = \frac{1}{400} \chi^2_{0.01,8} = \frac{20.090}{400} \approx 0.050 \;\Box$$

\section*{11.19}
$s \sim N(\sigma,\frac{\sigma^2}{2n})$, so $\frac{s-\sigma}{\sqrt{\sigma^2}{2n}} \sim Z$.
\begin{gather*}
Pr(-z_{\alpha/2} < \frac{s-\sigma}{\sfrac{\sigma}{\sqrt{2n}}} < z_{\alpha/2}) \approx 1-\alpha\\
Pr(\sigma - \frac{\sigma}{\sqrt{2n}}z_{\alpha/2} < s < \sigma + \frac{\sigma}{\sqrt{2n}}z_{\alpha/2}) \approx 1-\alpha\\
Pr(\sigma(1-\frac{z_{\alpha/2}}{\sqrt{2n}}) < s < \sigma(1+\frac{z_{\alpha/2}}{\sqrt{2n}})) \approx 1-\alpha\\
Pr(\frac{s}{1+\frac{z_{\alpha/2}}{\sqrt{2n}}} < \sigma < \frac{s}{1-\frac{z_{\alpha/2}}{\sqrt{2n}}}) \approx 1-\alpha \;\Box
\end{gather*}

\section*{12.1}
Simple; composite; composite; composite.
\section*{12.2}
Simple; composite; composite; composite.

\section*{12.3}
Given $k = 2$, $\alpha = Pr(X = 2) = \frac{\binom{2}{2}\binom{5}{0}}{\binom{7}{2}} = \frac{1}{21}$.\\
Given $k = 4$, $Pr(X = 2) = \frac{\binom{4}{2}\binom{3}{0}}{\binom{7}{2}} = \frac{2}{7}$, so $\beta = 1-\frac{2}{7} = \frac{5}{7}$ $\Box$

\section*{12.4}
Given $\theta = 0.9$, $\alpha = Pr(X \leqslant 16) = 0.133$.\\
Given $\theta = 0.6$, $\beta = Pr(X > 16) = 0.016$ $\Box$

\section*{12.5}
Note for the geometric distribution that $Pr(X > k) = (1-\theta)^k$.\\
Given $\theta = \theta_0$, $\alpha = Pr(X \geqslant k) = (1-\theta_0)^{k-1}$.\\
Given $\theta = \theta_1$, $\beta = Pr(X < k) = 1 - (1-\theta_1)^{k-1}$ $\Box$

\section*{12.6}
For an exponential variable $X$ with parameter $\lambda$
$$Pr(X \geq k) = \int_k^\infty \frac{1}{\lambda}e^{-\sfrac{x}{\lambda}} \;\mathrm{d}x = e^{-\sfrac{k}{\lambda}}$$
Given $\lambda = 2$, $\alpha = Pr(X \geqslant 3) = e^{-\sfrac{3}{2}}$.\\
Given $\lambda = 5$, $\beta = Pr(X < 3) = 1 - e^{-\sfrac{3}{5}}$ $\Box$

\section*{12.7}
$\overline{X} \sim N(\mu, \frac{1}{2})$. If $\mu = \mu_0$
$$Pr(\overline{X} > \mu_0 + 1) = Pr(\frac{\overline{X}-\mu_0}{\sqrt{1/2}} > \sqrt{2}) = Pr(Z > \sqrt{2}) \approx 0.0793 \;\Box$$

\section*{12.8}
Since it is impossible for the sample to take values $> \beta_0$ when $\beta = \beta_0$, we will never reject $H_0$ when it is true. Thus, $\alpha = 0$.\\
If $\beta = \beta_0 + 2$, $Pr(X \leqslant \beta_0 + 1) = \frac{\beta_0 + 1}{\beta_0 + 2}$ is the probability of a type II error $\Box$

\end{document}

% List of tex snippets:
%   - tex-header (this)
%   - R      --> \mathbb{R}
%   - Z      --> \mathbb{Z}
%   - B      --> \mathcal{B}
%   - E      --> \mathcal{E}
%   - M      --> \mathcal{M}
%   - m      --> \mathfrak{m}({#1})
%   - normlp --> \norm{{#1}}_{L^{{#2}}}
