\documentclass{article}
\usepackage[utf8]{inputenc}
\usepackage[margin=1in]{geometry}

\title{481 - Homework 10}
\author{Victor Zhang}
\date{November 11, 2020}

\usepackage[utf8]{inputenc}
\usepackage{amsmath}
\usepackage{amsfonts}
\usepackage{natbib}
\usepackage{graphicx}
% \usepackage{changepage}
\usepackage{amssymb}
\usepackage{xfrac}
% \usepackage{bm}
% \usepackage{empheq}

\newcommand{\contra}{\raisebox{\depth}{\#}}

\newenvironment{myindentpar}[1]
  {\begin{list}{}
          {\setlength{\leftmargin}{#1}}
          \item[]
  }
  {\end{list}}

\pagestyle{empty}

\begin{document}

\maketitle
% \begin{center}
% {\huge Econ 482 \hspace{0.5cm} HW 3}\
% {\Large \textbf{Victor Zhang}}\
% {\Large February 18, 2020}
% \end{center}

\section*{12.10}
$$L(\vec{x}, \mu) = \prod \frac{1}{\sqrt{2\pi}} e^{-\frac{\sum(x_i-\mu)^2}{2}} = (2\pi)^{-n/2} e^{-\frac{\sum(x_i-\mu)^2}{2}}$$
$$\frac{L_0}{L_1} = \exp\left[-\frac{1}{2}\sum (x_i-\mu_0)^2 - (x_i-\mu_1)^2\right] = \exp\left[-\frac{1}{2}\left(n(\mu_0^2-\mu_1^2) + 2(\mu_1-\mu_0)\overline{x}\right)\right]$$
Since $\mu_1 < \mu_0$, $\frac{L_0}{L_1} = c_1 e^{c_2\overline{x}}$ for $c_1,c_2 > 0$. Then $\frac{L_0}{L_1}$ is increasing in $\overline{x}$ and a critical region of the form $C = \{\overline{x} \leqslant K\}$ is most powerful. It follows that the region given is most powerful, from looking at tail probabilities of the Normal distribution $\Box$

\section*{12.11}
$$L(\vec{x}, \lambda) = \prod \frac{1}{\lambda}e^{-x/\lambda} = \frac{1}{\lambda^n}e^{-\sfrac{\sum x_i}{\lambda}}$$
$$\frac{L_0}{L_1} = \left(\frac{\theta_1}{\theta_0}\right)^n e^{\left(1/\theta_1 - 1/\theta_0\right)\sum x_i}$$
Since $\theta_1 > \theta_0$, $\frac{1}{\theta_1}-\frac{1}{\theta_0} < 0$. So $\frac{L_0}{L_1} = c_1 e^{-c_2 \sum x_i}$, where $c_1,c_2 > 0$. Then a region of the form $C = \{\sum x_i \geqslant K\}$ is most powerful by Neyman-Pearson. Note the sum of i.i.d. exponentials is a gamma RV. That is, under assumption of $H_0$, $\sum x_i \sim Gamma(n,\theta_0)$. Then the critical value $K = Gamma_{n,\theta_0,\alpha}$ $\Box$

\section*{12.12}
We may consider the binomial RV as a sum of i.i.d. Bernoulli variables $X_i$.
$$L(\vec{x}, p) = \prod p^{x_i}(1-p)^{1-x_i} = p^{\sum x_i}(1-p)^{n-\sum x_i}$$
$$\frac{L_0}{L_1} = \left(\frac{\theta_0}{\theta_1}\right)^{\sum x_i} \left(\frac{1-\theta_0}{1-\theta_1}\right)^{n-\sum x_i} = (1+\delta_1)^{\sum x_i} (1-\delta_2)^{n-\sum x_i}$$
This is an increasing function in $\sum x_i$ so a most powerful region is of the form $C = \{\sum x_i \leqslant K\}$. Then $K = Bin_{n,\theta_0,1-\alpha}$ $\Box$

\section*{12.13}
For large $n$, $Bin(n,\theta_0) \rightarrow N(n\theta_0, n\theta_0(1-\theta_0))$. Then $Pr(\sum x_i \leqslant K) = Pr(Z \leqslant \frac{K - n\theta_0}{\sqrt{n\theta_0(1-\theta_0)}})$. For $\alpha = 0.05$, $K = 31.94$, so our critical region is $C = \{\sum x_i \leqslant 31\}$. Now assuming $p = 0.3$
$$\beta = Pr(\sum x_i \geqslant K) = Pr(Z \geqslant \frac{K - n\theta_1}{\sqrt{n\theta_1(1-\theta_1)}}) = 0.337 \;\Box$$

\section*{12.14}
$$L(x, p) = p(1-p)^{x-1}$$
$$\frac{L_0}{L_1} = \frac{\theta_0}{\theta_1} \left(\frac{1-\theta_0}{1-\theta_1}\right)^{x-1}$$
Since $\theta_1 > \theta_0$ this is a decreasing function in $x$. Then a region of the form $C = \{x \geqslant K\}$ is most powerful. $Pr(x \geq K) = (1-\theta_0)^{k-1} = \alpha$, so $K = 1 + \frac{\ln\alpha}{\ln(1-\theta_0)}$ $\Box$

\section*{12.15}
$$L(\vec{x}, \sigma) = (2\pi\sigma^2)^{-n/2} e^{-\sfrac{\sum x_i^2}{2\sigma^2}}$$
$$\frac{L_0}{L_1} = c \cdot \exp\left[\left(\frac{1}{2\sigma_1^2} - \frac{1}{2\sigma_0^2}\right)\sum x_i^2\right]$$
Since $\sigma_1 > \sigma_0$, $\frac{1}{\sigma_1^2} - \frac{1}{\sigma_0^2} < 0$ so this is a decreasing function in $\sum x_i^2$. Then a most powerful region is of the form $C = \{\sum x_i^2 \geqslant K\}$. $Pr(\sum x_i^2 \geqslant K) = Pr(\sum \left(\frac{x_i}{\sigma_0}\right)^2 \geqslant \frac{K}{\sigma_0^2}) = Pr(\chi^2_n \geqslant \frac{K}{\sigma_0^2}) = \alpha$. Then $K = \sigma_0^2 \chi^2_{n,\alpha}$ $\Box$

\section*{12.16}
\subsection*{12.16.1}
$Pr(X \leqslant 14 \;|\; p = 0.9) = 0.0113$, $Pr(X > 14 \;|\; p = 0.6) = 0.1256$.
\begin{equation*}
\begin{split}
  Pr(\text{failure}) &= Pr(X \leqslant 14 \;|\; p = 0.9) Pr(p = 0.9) + Pr(X > 14 \;|\; p = 0.6) Pr(p = 0.6)\\
  &= 0.0113\cdot 0.8 + 0.1256\cdot 0.2 = 0.0342 \;\Box
\end{split}
\end{equation*}
\subsection*{12.16.2}
$Pr(X \leqslant 15 \;|\; p = 0.9) = 0.0432$, $Pr(X > 15 \;|\; p = 0.6) = 0.0510$.
\begin{equation*}
\begin{split}
  Pr(\text{failure}) &= Pr(X \leqslant 15 \;|\; p = 0.9) Pr(p = 0.9) + Pr(X > 15 \;|\; p = 0.6) Pr(p = 0.6)\\
  &= 0.0432\cdot 0.8 + 0.0510\cdot 0.2 = 0.0448 \;\Box
\end{split}
\end{equation*}
\subsection*{12.16.3}
$Pr(X \leqslant 16 \;|\; p = 0.9) = 0.1330$, $Pr(X > 16 \;|\; p = 0.6) = 0.0160$.
\begin{equation*}
\begin{split}
  Pr(\text{failure}) &= Pr(X \leqslant 16 \;|\; p = 0.9) Pr(p = 0.9) + Pr(X > 16 \;|\; p = 0.6) Pr(p = 0.6)\\
  &= 0.1330\cdot 0.8 + 0.0160\cdot 0.2 = 0.1096 \;\Box
\end{split}
\end{equation*}

\end{document}

% List of tex snippets:
%   - tex-header (this)
%   - R      --> \mathbb{R}
%   - Z      --> \mathbb{Z}
%   - B      --> \mathcal{B}
%   - E      --> \mathcal{E}
%   - M      --> \mathcal{M}
%   - m      --> \mathfrak{m}({#1})
%   - normlp --> \norm{{#1}}_{L^{{#2}}}
