\documentclass{article}
\usepackage[utf8]{inputenc}

\title{481 - Homework 1}
\author{Victor Zhang}
\date{September 11, 2020}

\usepackage[utf8]{inputenc}
\usepackage{amsmath}
\usepackage{amsfonts}
\usepackage{natbib}
\usepackage{graphicx}
% \usepackage{changepage}
\usepackage{amssymb}
\usepackage{xfrac}
% \usepackage{bm}
% \usepackage{empheq}

\newcommand{\contra}{\raisebox{\depth}{\#}}

\newenvironment{myindentpar}[1]
  {\begin{list}{}
          {\setlength{\leftmargin}{#1}}
          \item[]
  }
  {\end{list}}

\pagestyle{empty}

\begin{document}

\maketitle
% \begin{center}
% {\huge Econ 482 \hspace{0.5cm} HW 3}\
% {\Large \textbf{Victor Zhang}}\
% {\Large February 18, 2020}
% \end{center}

\section*{4.7}
$E[Y] = \int\limits_\mathbb{R} y f(y) \textrm{d}y = \int\limits_2^4 y\frac{1}{8}(y+1)\textrm{d}y = \frac{37}{12}$

\section*{4.8}
$E[X] = \int\limits_0^1 x \cdot x \textrm{d}x + \int\limits_1^2 x(2-x) \textrm{d}x = 1$

\section*{4.47}
\subsection*{a.}
$\textrm{cov}(U,V) = E[UV] - E[U]E[V] = E[X^3] - E[X]E[X^2]$. Note $E[X^n]$ is odd when $n$ is odd. Thus, $E[X] = E[X^3] = 0$ and $\textrm{cov}(U,V) = 0$ $\Box$

\subsection*{b.}
$P(U < -\frac{1}{2}) > 0$ and $P(V < \frac{1}{4}) > 0$, but $P(U < -\frac{1}{2} \cap V < \frac{1}{4}) = 0$. So $U$ and $V$ are not independent $\Box$

\section*{4.49}
\subsection*{a.}
$E[2X_1 - 3X_2 + 4X_3] = 2E[X_1] - 3E[X_2] + 4E[X_3] = -7$\\
$\mathrm{var}(2X_1 - 3X_2 + 4X_3) = 4 \mathrm{var}(X_1) + 9 \mathrm{var}(X_2) + 16 \mathrm{var}(X_3) = 155$

\subsection*{b.}
$E[X_1 + 2X_2 - X_3] = E[X_1] + 2E[X_2] - E[X_3] = 19$\\
$\mathrm{var}(X_1 + 2X_2 - X_3) = \mathrm{var}(X_1) + 4 \mathrm{var}(X_2) + \mathrm{var}(X_3) = 36$

\section*{4.54}
$\textrm{cov}(X_1 - 2X_2 + 3X_3, -2X_1 + 3X_2 + 4X_3) = -2\mathrm{var}(X_1) + 3\textrm{cov}(X_1,X_2) + 4\textrm{cov}(X_1,X_3) + 4\textrm{cov}(X_1,X_2) - 6 \mathrm{var}(X_2) - 8\textrm{cov}(X_2,X_3) - 6\textrm{cov}(X_1,X_3) + 9\textrm{cov}(X_2,X_3) + 12\mathrm{var}(X_3) = -2\cdot 5 + 3\cdot 3 + 4 \cdot (-2) + 4 \cdot 3 - 6\cdot 4 -8 \cdot 0 - 6 \cdot (-2) + 9 \cdot 0 + 12 \cdot 7 = 75$

\section*{5.23}
Let $X \sim \textrm{Geom}(p)$ s.t. $P(X = x) = p(1-p)^{x-1}$. Then $P(X > n) = (1-p)^n$, $P(X = x + n) = p(1-p)^{x+n-1}$, $P(X = x+n \;|\; X > n) = \frac{p(1-p)^{x+n-1}}{(1-p)^n} = p(1-p)^{x-1} \; \Box$

\section*{6.3}
By definition, the distribution function $f_x$ is
\begin{equation*}
    f_x(x) =
    \begin{cases}
        \frac{1}{\beta - \alpha} & \alpha \leq x \leq \beta\\
        0 & \textrm{otherwise}\\
    \end{cases}
\end{equation*}

\section*{6.16}
Let $X \sim \mathrm{Exp}(\lambda)$ with distribution function $f_x(x) = \frac{1}{\lambda}e^{-\sfrac{x}{\lambda}}$.
$$P[(X\geq t + T)|(X\geq T)] = \frac{\int\limits_{t+T}^\infty \frac{1}{\lambda}e^{-\sfrac{x}{\lambda}}\mathrm{d}x}{\int\limits_T^\infty \frac{1}{\lambda}e^{-\sfrac{x}{\lambda}}\mathrm{d}x} = \frac{e^{-\frac{t+T}{\lambda}}}{e^{-\frac{T}{\lambda}}} = e^{-\sfrac{t}{\lambda}} = P(X \geq t)$$

\section*{6.20}
\subsection*{a.}
\begin{equation*}
\begin{split}
    \mu &= \int\limits_0^\infty 2\alpha x^2e^{-\alpha x^2} \mathrm{d}x\\
    &= \int\limits_0^\infty x \cdot 2\alpha xe^{-\alpha x^2} \mathrm{d}x\\
    &= -xe^{-\alpha x^2} \Big\rvert_0^\infty - \int\limits_0^\infty -e^{-\alpha x^2} \mathrm{d}x\\
    &= 0 + \sqrt{\frac{\pi}{\alpha}}\int\limits_0^\infty \frac{1}{\sqrt{2\pi \frac{1}{2\alpha}}}e^{-\frac{x}{2\cdot\sfrac{1}{2\alpha}}} \mathrm{d}x\\
    &= \frac{1}{2}\sqrt{\frac{\pi}{\alpha}}
\end{split}
\end{equation*}

\subsection*{b.}
\begin{equation*}
\begin{split}
    \sigma^2 &= E[X^2] - E[X]^2\\
    &= \int\limits_0^\infty x^2 \cdot 2\alpha x e^{-\alpha x^2} \mathrm{d}x - \mu^2\\
    &= -x^2e^{-\alpha x^2} \Big\rvert_0^\infty + \int\limits_0^\infty 2xe^{-\alpha x^2} \mathrm{d}x - \mu^2\\
    &= 0 - \frac{1}{\alpha}e^{-\alpha x^2} \Big\rvert_0^\infty  - \frac{1}{4}\frac{\pi}{\alpha}\\
    &= \frac{1}{\alpha}\left(1 - \frac{\pi}{4}\right)
\end{split}
\end{equation*}

\section*{6.37}
$\mathrm{cov}(X,Y) = E[XY] - E[X]E[Y] = E[X^3] - E[X]E[X^2]$.\\
$E[X^n] = \int\limits_\mathbb{R} \frac{x^n}{\sqrt{2\pi}}e^{-\sfrac{x^2}{2}} \mathrm{d}x$. Note when $n$ is odd, the integrand is odd, so $E[X] = E[X^3] = 0$, and thus $\mathrm{cov}(X,Y) = 0$.

\section*{7.16}
Note since $Y = \frac{2X}{1+2X}$ we may write $X = \frac{Y}{2-2Y}$. This is an increasing function on the range of $Y$, so we may derive the pdf of $Y$, $g$ to be
$$g(y) = \frac{k\left(\frac{y}{2(1-y)}\right)^3}{\left(1+\frac{2y}{2(1-y)}\right)^6}\;\frac{\mathrm{d}}{\mathrm{d}y}\frac{y}{2(1-y)} = \frac{ky^3(1-y)^3}{8}\frac{1}{2(1-y)^2}=\frac{ky^3(1-y)}{16}$$
This corresponds to a beta distribution with parameters $\alpha = 4$ and $\beta = 2$. Thus, $\frac{k}{16} = 20$ and $k = 320$.

\section*{7.18}
We may rewrite the relation as $X = e^{-{Y}/{2}}$. On the range of $Y$, this function is mono. decreasing, so we may derive the pdf of $Y$, $g$ to be
$$\left\lvert\frac{\mathrm{d}}{\mathrm{d}y}e^{-\frac{y}{2}}\right\rvert = \frac{1}{2}e^{-\frac{y}{2}} = \frac{1}{\Gamma(1)2^1}x^{1-1}e^{-\frac{y}{2}}$$
This corresponds with a gamma distribution with parameters $\alpha = 1$ and $\beta = 2$.

\end{document}
