\documentclass{article}
\usepackage[utf8]{inputenc}
\usepackage[margin=1in]{geometry}

\title{530 - Debate Notes}
\author{Victor Zhang}
\date{October 25, 2021}

\usepackage[utf8]{inputenc}
\usepackage{amsmath}
\usepackage{amsfonts}
\usepackage{natbib}
\usepackage{graphicx}
% \usepackage{changepage}
\usepackage{amssymb}
\usepackage{xfrac}
% \usepackage{bm}
% \usepackage{empheq}
\usepackage{dirtytalk}

\newcommand{\contra}{\raisebox{\depth}{\#}}

\newenvironment{myindentpar}[1]
  {\begin{list}{}
          {
            \setlength{\leftmargin}{#1}
            \setlength{\rightmargin}{#1}
          }
          \item[]
  }
  {\end{list}}

\pagestyle{empty}

\begin{document}

\maketitle
% \begin{center}
% {\huge Econ 482 \hspace{0.5cm} HW 3}\
% {\Large \textbf{Victor Zhang}}\
% {\Large February 18, 2020}
% \end{center}

\section{Pros}
- The concept of actors being below omniscience necessitates that there be a potential for omniscience. This is baked into the definition of bounded rationality presented by Simon et al. This also assumes objective environments regardless of objectivity of subjects.

The problem is that the best solution for vision is not necessarily objectivity.
\begin{myindentpar}{1em}
For instance, in economics, expected value is not as good of a predictor of behavior as expected utility, a less \say{objectively} correct measure.
\end{myindentpar}

- All-seeing eye assumes the purpose of vision is to filter the world into the mind one-to-one. This is not true. For instance, visual illusions would not exist if vision were truly objective. On this, many visual illusions occur because we try to interoperate 2D and 3D images and then act surprised when people perceive 2D as 3D and vice versa. This calls into question the existence of an \say{objective} truth for visual stimuli.

- All-seeing eye assumes vision is independent of the viewer. But this is not very true either. Different organisms and indeed different people perceive different things based on prior knowledge or purpose of vision.

\section{Cons}
High-level cons:
- Simplification is necessary when making models... All models are \say{wrong} but some models are more useful than others.

\end{document}

% List of tex snippets:
%   - tex-header (this)
%   - R      --> \mathbb{R}
%   - Z      --> \mathbb{Z}
%   - B      --> \mathcal{B}
%   - E      --> \mathcal{E}
%   - M      --> \mathcal{M}
%   - m      --> \mathfrak{m}({#1})
%   - normlp --> \norm{{#1}}_{L^{{#2}}}
