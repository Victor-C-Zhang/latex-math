\documentclass{article}
\usepackage[utf8]{inputenc}
\usepackage[margin=1in]{geometry}

\title{530 - Preliminary Proposal}
\author{Victor Zhang}
\date{September 21, 2021}

\usepackage[utf8]{inputenc}
\usepackage{amsmath}
\usepackage{amsfonts}
\usepackage{natbib}
\usepackage{graphicx}
% \usepackage{changepage}
\usepackage{amssymb}
\usepackage{xfrac}
% \usepackage{bm}
% \usepackage{empheq}
\usepackage{dirtytalk}

\newcommand{\contra}{\raisebox{\depth}{\#}}

\newenvironment{myindentpar}[1]
  {\begin{list}{}
          {\setlength{\leftmargin}{#1}}
          \item[]
  }
  {\end{list}}

\pagestyle{empty}

\begin{document}

\maketitle
% \begin{center}
% {\huge Econ 482 \hspace{0.5cm} HW 3}\
% {\Large \textbf{Victor Zhang}}\
% {\Large February 18, 2020}
% \end{center}

I would like to work on a term project focused on \say{intuitive} classification models. In particular, I would like to build an English handwriting decoder that can interpret and guess context. Many of the current approaches to handwriting recognition follow a well-defined process involving individual character extraction and matching. What happens afterwards is less well-defined. In humans, a good amount of intuition and guesswork is required to parse potential words from a jumble of characters. This intuition is built by exposure to a broad range of topics and contexts. I would like my system to be able to mimic this learned intuition by tapping into the Internet and the vast stores of context it hosts.

A core goal of AI is to be able to intuit and guess in the same way a human might. I expect that this project will teach me a lot about the tractability of modeling intuition. It may very well turn out that the system I build does no better than a plain old RNN or only does better with a very niche context. Ultimately, I hope that I will be able to approximate human intuition with statistics and engineering.

I plan on splitting my project into \textit{discovery}, \textit{design}, and \textit{implementation} phases.
\begin{itemize}
  \item An initial discovery phase will involve a literature review and data gathering to scope the project and establish solid project expectations. This will take an estimated 2 weeks.
  \item Based on literature review, a design iteration will be generated. It may involve additional reading and will take approximately a week.
  \item The design will be implemented and tested. This will take between a few days and a few weeks depending on the technical challenge of the design.
  \item Additional iterations will be made based on the results of testing, time permitting.
\end{itemize}
Overall, there will be two meta-phases of the project. The first involves just getting a handwriting classifier to work. The second and more interesting meta-phase is where the model is augmented with contextual hints. This is also the \say{make or break} meta-phase where any interesting conclusions will be made.

Evaluation of the final project should take into consideration the scope of the model and its functionality. A minimum viable product shall include a handwriting recognizer that does better than random chance and a module that can augment the recognizer based on context. Demonstrated human ingenuity on the part of the implementor should be looked upon favorably.

I expect for the going to be tough in the first few iterations of the project as I wrestle with just the basic character extraction and classification. When I move into the augmentation part of the project, I expect few gains to be made immediately, but the pace of progress to increase as problem points are identified and worked on. Ultimately, I expect no improvement over the standard classification strategies for general or random text but a marked improvement for topical text.

\end{document}

% List of tex snippets:
%   - tex-header (this)
%   - R      --> \mathbb{R}
%   - Z      --> \mathbb{Z}
%   - B      --> \mathcal{B}
%   - E      --> \mathcal{E}
%   - M      --> \mathcal{M}
%   - m      --> \mathfrak{m}({#1})
%   - normlp --> \norm{{#1}}_{L^{{#2}}}

look at tracking and deep learning
problem needs to be less abstract; specific problem and specific results
